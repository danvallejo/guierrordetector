\section{Empirical Evaluation}
\label{sec:evaluation}

%Our experiment investigate the followinig four research questions:

%\begin{itemize}
%\Item can our approach detect real bugs in multithreaded GUI applications?

%\Item is the reflection-aware call graph construction algorithm useful?

%\Item how effective is the proposed heuristic? is the reflection-aware call graph construction algorithm useful?
%\end{itemize}

Our experiment objective is three-fold: to demonstrate the effectiveness
of our approach in detecting real bugs in multithreaded GUI applications, to 
compare the reflection-aware call graph construction algorithms
with existing ones , and to evaluate the usefulness of the proposed 
 heuristics.  

First, we describe our subject programs (Section~\ref{sec:subjects}) and the experimental procedural (Section~\ref{sec:procedural}).
We then show that our technique detects bugs in real-world GUI applications on
four supported frameworks (Section~\ref{sec:errors}). We also compare our technique with
a straightfoward approach in Section~\ref{sec:straightforward}. We further compare the reflection-aware call graph
construction algorithms with existing algorithms, showing its usefulness (Section~\ref{sec:reflectionaware}).
 Finally, we show that the set of proposed heuristics are effective
in removing redundant warnings and eliminate false positivies (Section~\ref{sec:filters}). 



\subsection{Subject Programs}
\label{sec:subjects}

We use a set of open-source projects from Sourceforge, Google code and Eclipse plugin
market as evaluation subjects. We choose subjects for each supported GUI framework
as follows. We search the framework keywords (e.g., ``Java Swing'' or ``Java SWT'')
in the above three source repositories, and select subjects based on the following
criterias: first, the subject must be a multithread GUI application, not
an open library. Second, the subject must undergone at least two years of active development,
and is listed in the first 3 results pages based on popularity. This permits us
to exclude all pre-mature subjects, which may contain obvious errors.

The subjects used in our experiment are shown in Table~\ref{table:subjects}. Our
subjects range from end-user applications, programming tools, to games. Specifically:

\begin{itemize}
\Item CelarFTP\footnote{\url{http://celerftp.sourceforge.net/}}
and FileBunker\footnote{\url{http://filebunker.sourceforge.net/}} are two SWT desktop applications.
CelarFTP implements a simple FTP client based on Java and SWT UI library, and
FileBunker is a file backup application which uses one or more GMail accounts as its backup repository.

\Item  AnyEdit\footnote{\url{http://andrei.gmxhome.de/anyedit/}}, FileSync4Eclipse\footnote{\url{andrei.gmxhome.de/filesync/}}, and
EclipseRunner\footnote{\url{http://eclipserunnerplugin.googlecode.com/}} are three popular eclipse plugins.
AnyEdit adds several new tools to  Eclipse editors, output consoles, and main menu. 
FileSync4Eclipse is a file synchronisation tool to keep files outside of Eclipse projects in synchronized
with Eclipse project files. EclipseRunner extends capability of running launch configurations in Eclipse IDE.

\Item  S3dropbox\footnote{\url{http://s3dropbox.googlecode.com}} and SudokuPuzzleSolver\footnote{\url{http://sudokupuzzlesol.sourceforge.net/}}
are two Swing desktop applications. S3dropbox implements a client that allows users
to drag and drop files, which are then uploaded to their Amazon S3 accounts. SodokuPuzzleSolver
computes Sodoku solutions efficiently using mutlithreading execution. 

\Item  MyTracks\footnote{\url{http://code.google.com/p/mytracks/}}, Fennec\footnote{\url{http://www.mozilla.org/en-US/mobile/}},
and SGTPuzzles\footnote{\url{http://chris.boyle.name/projects/android-puzzles}} are three Android applications.
MyTracks, developed by Google, records users' GPS tracks in outdoor activities, and provides
interfaces to visualize them on Google Maps. Fennec, developed by Mozilla, is the Mozilla
Firefox web browser for mobile devices. SGTPuzzles is a small logic single-player game.

\end{itemize}


\begin{table}[t]
\begin{center}
 \fontsize{9pt}{\baselineskip}\selectfont
\hspace*{-0.2cm}
\begin{tabular}{|p{4.4cm}||c|c|c|c|}
\hline
 Program (version) & LOC & Classes & Methods \\
\hline \hline
\multicolumn{4}{|l|}{SWT desktop applications}   \\
 \hline
 CelarFTP (3.0 alpha) &  2293 &  20 &  165  \\
 \hline
 FileBunker (1.1.2)&  14237 &  150 &  1106  \\
 \hline
 \hline
\multicolumn{4}{|l|}{Eclipse plugins}   \\
 \hline
 EclipseRunner (1.1.0) &  3101 &  48 &  354\\
 \hline
 FileSync (1.4.0)&  7374 &  36 &  510 \\
 \hline
 AnyEdit (2.4.0)&  11077 &  74 &  649 \\
 \hline
 \hline
\multicolumn{4}{|l|}{Swing desktop applications}   \\
 \hline
 S3dropbox (1.7) &  2353 &  42  &  224 \\
 \hline
 Sudoku Solver (1.06)&  3555 &  10 &  62 \\
 \hline
 \hline
\multicolumn{4}{|l|}{Android mobile applications}   \\
 \hline
 SGTPuzzler (v9306.11)&  2220 &  16 &  148 \\
 \hline
 Fennec (10.0.a.rc3)&  8577 &  51 &  620 \\
 \hline
 MyTracks (1.1.13.rc4)&  20297 &  143 &  1374 \\
\hline
\multicolumn{4}{l}{}   \\
\hline
 GUI framework (version) & LOC & Classes & Methods  \\
\hline \hline
 SWT (3.6.2)&  20297 &  143 &  1374  \\
\hline
 Eclipse plugin development (3.6.2)&  20297 &  143 &  1374  \\
\hline
Swing (1.6)&  20297 &  143 &  1374  \\
\hline
 Android (3.6)&  20297 &  143 &  1374  \\
\hline
\end{tabular}

\end{center}
\vspace{-15pt}
\Caption{{\label{table:subjects} Open source programs
used in our evaluation. Column ``LOC'' is the number of lines
of code, as counted by LOCC~\cite{}.  Each program is analyzed
together with its dependent GUI framework, as listed in
the bottom table.} }
\end{table}


\subsection{Procedural}
\label{sec:procedural}

We run our tool on each subject with three call graph construction
algorithms: RTA~\cite{}, 0-CFA, and 1-CFA~\cite{}.  When running
each call graph construction algorithm on Android applications, we
use two configurations: with and without taming reflection (Section~\ref{}).
$k$-CFA ($k >$ 1) algorithm
does not scale to large programs, and thus is not used in this experiment

SGTPuzzler is the only subject that uses native methods to interact with
underlying operating systems. Thus, we manually added XXX \CodeIn{@CalledByNative}
annotations after inspecting its source code. In addition, two subjects XXXX and XXX use a self-defined
way to interact with the GUI framework. In this experiment, none of the paper
authors was faimilar with the subjects; but we found it is quite easy
to add extra those annotations and heuristic filters and finished these
manual parts in less than 20 minutes.

After provided with necessary annotations and heuristic filters,
our tool works in a fully-automatic, push-bottom manner. For each output
warning, we manually determine its validity by either searching the
bug repository to check whether the same bug has been reported before,
or submitting a new bug report, or writing a test driver to reproduce
the error.



\begin{table*}[ht]
\begin{center}
 \fontsize{9pt}{\baselineskip}\selectfont
\hspace*{-0.2cm}
\setlength{\tabcolsep}{.65\tabcolsep}
\begin{tabular}{|l||c|c|c||c|c|c||c|c|c||c|c|}
\hline
 Subject&  \multicolumn{9}{|c||}{Our Technique} & \multicolumn{2}{|c|}{The Approach in}  \\
\cline{2-10}
 Program  &  \multicolumn{3}{|c|}{RTA / reflection aware}& \multicolumn{3}{|c|}{0-CFA / reflection aware} & \multicolumn{3}{|c||}{1-CFA / reflection aware} & \multicolumn{2}{|c|}{Section~\ref{sec:straightforward}}  \\
\cline{2-12}
 & CG Size & \#Warning & \#Bug & CG Size & \#Warning & \#Bug & CG Size & \#Warning & \#Bug & \#Warning & \#Bug\\
\hline \hline
\multicolumn{12}{|l|}{SWT desktop applications}   \\
 \hline
 xxx &  xxxx &  xx &  xxx & xx & xx & xx& xx & xx & xx& xx & xx \\
 \hline
 xxx &  xxxx &  xx &  xxx & xx & xx & xx& xx & xx & xx& xx & xx \\
 \hline
 xxx &  xxxx &  xx &  xxx & xx & xx & xx& xx & xx & xx& xx & xx \\
 \hline
 \hline
\multicolumn{12}{|l|}{Eclipse plugins}   \\
 \hline
 xxx &  xxxx &  xx &  xxx & xx & xx & xx& xx & xx & xx& xx & xx \\
 \hline
 xxx &  xxxx &  xx &  xxx & xx & xx & xx& xx & xx & xx& xx & xx \\
 \hline
 xxx &  xxxx &  xx &  xxx & xx & xx & xx& xx & xx & xx& xx & xx \\
 \hline
 \hline
\multicolumn{12}{|l|}{Swing desktop applications}   \\
 \hline
 S3dropbox &  xxxx &  xx &  xxx & xx & xx & xx& xx & 1 & 1 & 210 & xx \\
 \hline
  Sudoku Solver&  xxxx &  xx &  xxx & xx & xx & xx& xx & xx & xx& 356 & xx \\
 \hline
 \hline
\multicolumn{12}{|l|}{Android mobile applications}   \\
 \hline
 xxx &  xxxx / xxxx &  xx / xxxx &  xxx / xxxx & xx /xxx & xx / xxxx  & xx / xxxx& xx / xxx & xxxx / xxx  & xxxx / xxx& xx & xx \\
 \hline
 xxx &  xxxx &  xx &  xxx & xx & xx & xx& xx & xx & xx& xx & xx \\
 \hline
 xxx &  xxxx &  xx &  xxx & xx & xx & xx& xx & xx & xx& xx & xx \\
\hline
\end{tabular}
\end{center}
\vspace{-15pt}
\Caption{{\label{table:results}Experimental results in finding errors
in multithreaded GUI programs. Column ``CG Size'' shows the
call graph size in terms of node number. Column ``\#Warning''
shows the number of warnings issued by our tool. Column ``\#Bug'' shows
the actual bugs found. Columns ``RTA / reflection aware '', ``0-CFA / reflection aware'',
and ``1-CFA / reflection aware'' show the results of using different
call graph construction algorithms with (before ``/'') or without (after ``/'')
taming the reflection on Android applications. In each cell of the table
for Android applications, a slash ``/'' separates the result of
using normal call graph construction algorithm and the reflection-aware
call graph construction algorithm. As a comparison, the
results of a straightforward approach (Section~\ref{sec:straightforward}) are show at the far right.} }
\end{table*}

\subsection{Results}
\label{sec:results}

Table~\ref{table:results} show the errors in
our subject programs, and Table~\ref{table:filters} show
the effect of applying each heuristic filter.

\subsubsection{Errors in Multithreaded GUI}
\label{sec:errors}


characteristic of errors in various subjects

False positive rate, and why?

redundant rate

response

Swing

Swing is a single-threaded GUI toolkit, and is not thread-safe. 
As the official documentation states, all Swing code must be executed in the UI-thread and
invoking Swing code from multiple threads risks thread interference or memory-consistency errors. 

Since Swing does not provide any run-time checks for invalid thread access and, most of the time,
programmers can get away with an apparently ``well behaved'' Swing GUI that actually breaks the
single-thread rule.

Example in Swing
\begin{figure}[t]
\begin{CodeOut}
\begin{alltt}
/* In class: com.tomczarniecki.s3.gui.DeleteBucketAction */
59.private void deleteBucket() \{
60.    executor.execute(new Runnable() \{
61.        public void run() \{
62.            try \{
63.                controller.deleteCurrentBucket();
64.            \} catch (Exception e) \{
65.                logger.info("Delete failed", e);
66.                deleteError(); 
67.            \}
68.       \}
69.    \});
70.\}

77.private void deleteError() \{
78.    String text = "Cannot delete folder .....";
79.    display.showErrorMessage("Delete failed", 
        String.format(text, controller.getSelectedBucketName())); 
80.\}
\end{alltt}
\end{CodeOut}
\vspace*{-2.0ex} \Caption{{\label{fig:swingerror} A
potential invalid thread error reported by our tool
in the S3dropbox Swing application. The error occurs when
the invocation of \CodeIn{deleteCurrentBucket} at line 63 throws
an exception, which lead to the \CodeIn{deleteError} method
access a Swing GUI object \CodeIn{display} at line 79 from
a non-UI thread. This error has been confirmed by S3dropbox developers.
}} %\vspace{-5mm}
\end{figure}


Example in  Eclipse plugin
\begin{figure}[t]
\begin{CodeOut}
\begin{alltt}
179.private void initializeResourceChangeListener() \{
180.  ResourcesPlugin.getWorkspace().addResourceChangeListener(
        new IResourceChangeListener() \{
181.      public void resourceChanged(IResourceChangeEvent event) \{
182.        refresh();
183.      \}
184.  \}, IResourceChangeEvent.POST\_CHANGE);
185.\}

414.public void refresh() \{
415.  getViewer().refresh();
416.\}
\end{alltt}
\end{CodeOut}
\vspace*{-2.0ex} \Caption{{\label{fig:pluginerror} A
potential invalid thread error reported by our tool
in the EclipseRunner plugin. The \CodeIn{IResourceChangeListener} object
added at line 180 is invoked by a non-UI thread, and the
callback method \CodeIn{resourceChanged} calls the \CodeIn{refresh}
method at line 414. However, the \CodeIn{refresh} method
directly accesses the GUI objects (at line 415) and triggers the error.
This error has been reported by other users 13 months after the
buggy code was checked in, and fixed by developers.
}} %\vspace{-5mm}
\end{figure}


\vspace{1mm}

\noindent \textbf{\textit{Summary.}}Our technique finds real errors
in multithreaded GUI applications on four different frameworks with
an acceptable false positive rate.

\subsubsection{Comparison with a Straightforward Approach}
\label{sec:straightforward}

As we mentioned in Section~\ref{sec:finding}, a standard way to avoid
invalid thread access errors is to wrap every GUI access code
with message passing methods (i.e., the safe methods). Thus, a
straightforward way to detect potential invalid error
is to check every method to determine whether all GUI-accessing code
is probably wrapping by safe methods. This approach
issues a warning for every violation. 
%If not, this approach issues warnings.

We implemented this straightforward approach and experimentally compared its
effectiveness with the technique presented in Section~\ref{}.
The far right column in Table~\ref{table:results} shows its result.
The results show that this straightforward approach can identified every
error as our technique does, but issues a tremendous number of false
positives. The primary reason is that this simple approach
does not globally reason about calling relationship between
threads, UI-accessing methods, and safe methods, thus it often incorrectly
classifies GUI accessing code which will never be executed
in a non-UI thread as errenous. Furthermore, our technique
outputs a method call chain in the error report, which can help
developers understand how an invalid thread access error is triggered.


\vspace{1mm}

\noindent \textbf{\textit{Summary.}} Our technique provides
richer contextually information for a potential error, and 
is significantly more precise than the straightforward approach.

\subsubsection{Call Graph Construction Algorithms}
\label{sec:reflectionaware}

We compare three call graph construction algorithms, RTA, 0-CFA, and 1-CFA,
in our experiments. As shown in Table~\ref{table:results},  XXX found
more errors with less warnings than the XXX and XXXX algorithms. This
is because RTA and 0-CFA do not consider the context XXX, and it may
mix the same methods called by different callers. Although the call graphs
built by 1-CFA are significantly larger, XXX, the algorithm still runs
in a practical amouont of time as presented in Section~\ref{}. For example,
our tool finished building the call graph for the largest subject (xxx)
within xxx minutes.

The results in Table~\ref{table:results} also shown that using a reflection-aware
call graph construction algorithm is critical in finding errors in Android
applications. None of the errors in the studied Android applications
can be found by using normal call graph construction algorithms, because their 
limitation in handling reflection.

\vspace{1mm}

\noindent \textbf{\textit{Summary.}} XX The 1-CFA call graph construction
algorithm is more effective in finding errors. XXX. Our reflection-aware call
graph construction algorithm is useful to find errors in
Android applications.

\subsubsection{Heuristic Filters}
\label{sec:filters}

We measure the effectiveness of the proposed heuristic filters of
Section~\ref{} via the number of removed warnings.

Table~\ref{table:filters} tabulates the results. The results show that
our technique outputs thousnads of warnings for each subject, due to
the conservative nature of static analysis.  However, the
set of proposed heuristic filters are highly effectively. They
 reduce the false positive numbers and redundant warnings by up to a factor of
XXX. Specifically, filters $XX_i$ and $XX_j$  identified xx\% of
the warnings as likely false positives, and filters $YY_i$
and $YY_j$ identified YY\% of the warnings as redundant warnings.
We further manually check each filtered warning, and found no
real errors have been removed.

\vspace{1mm}

\noindent \textbf{\textit{Summary.}} Although a static analysis may
issue false postivies or redundant reports, a set of well-designed
heuristics can still remove most of them.
Heuristic filter, effective.

\begin{table}[t]
\begin{center}
 \fontsize{9pt}{\baselineskip}\selectfont
\hspace*{-0.2cm}
\begin{tabular}{|l||c|c|c|c|}
\hline
 Program  & \#Warning & \#Warning & \#Warning & \#Warning \\
 (version) & $F_1$ & $F_2$ & $F_3$&$F_4$ \\
\hline \hline
\multicolumn{5}{|l|}{SWT desktop applications}   \\
 \hline
 xxx &  xxxx &  xx &  xxx & xx \\
 \hline
 xxx &  xxxx &  xx &  xxx & xx \\
 \hline
 xxx &  xxxx &  xx &  xxx & xx \\
 \hline
 \hline
\multicolumn{5}{|l|}{Eclipse plugins}   \\
 \hline
 xxx &  xxxx &  xx &  xxx & xx \\
 \hline
 xxx &  xxxx &  xx &  xxx & xx \\
 \hline
 xxx &  xxxx &  xx &  xxx & xx \\
 \hline
 \hline
\multicolumn{5}{|l|}{Swing desktop applications}   \\
 \hline
 xxx &  xxxx &  xx &  xxx & xx \\
 \hline
 xxx &  xxxx &  xx &  xxx & xx \\
 \hline
 xxx &  xxxx &  xx &  xxx & xx \\
 \hline
 \hline
\multicolumn{5}{|l|}{Android mobile applications}   \\
 \hline
 xxx &  xxxx &  xx &  xxx & xx \\
 \hline
 xxx &  xxxx &  xx &  xxx & xx \\
 \hline
 xxx &  xxxx &  xx &  xxx & xx \\
\hline
\end{tabular}
\end{center}
\vspace{-15pt}
\Caption{{\label{table:filters}Number of warnings after applying a set of
heuristic filters (Section~\ref{}). Column ``\#Warning $F_i$''
represents the number of remaining warnings after applying
filters 1 -- $i$, and the number of last column is taken
from Table~\ref{} (Subcolumn ``\#Warning'' under column ``1-CFA / with taming reflection'').} }
\end{table}


\subsection{Discussions}

We next discuss several important issues in our evaluation.

\subsubsection{Performance and Scalability}

Our tool runs in a practical amount of time. Our evaluations
were conducted on a 2.67GHz Intel Core PC with 4GB
physical memory (1GB is allocated for JVM), running Fedora 16.
For the largest subject, xxx, our tool finished the whole analysis
within 286 seconds using the most expensive call graph construction
algorithm (the reflection-aware 1-CFA analysis). Analyzing
other subjects using different algorithms took less time.
The performance has not been tuned, and we believe
that it could be improved further.

Our experiments also demonstrate that our approach has good scalability.
Our tool has analyzed xxx LOC in total, and together with all dependent libraries,
the largest subject our approach can analyze and detect errors has
over XXX LOC, xx classes, and xxx methods.



\subsubsection{Threats to Validity}

There are two major threats to validity in our evaluation. 
The major threat is the degree to which the subject programs
used in our experiment are representative of true practice.
In our evaluation, the studied subjects were selected with a bias towards popularity from
three open-source repositories. Another threat is that we only employed three
well-known call graph construction algorithms (i.e., RTA, 0-CFA, and 1-CFA) 
in our implementation. It is still unknown whether using other 
call graph construction algorithms such XFA~\cite{} may achieve better results.

\subsubsection{Limitations}

Our technique is limited in two aspects. First, it only considers
non-UI threads that are spawned by the UI-thread after the GUI
is initialized, and ignores other possible non-UI threads
that are created during the pre-initialization GUI work. One way
to remedy this limitation is to design an analysis to identify
such threads. Second, our tool cannot compute call relationships
for inter-process communication between components. It also
requires users to manually add annotations to characterize
call relationships that involve native methods. This limitation
may lead to false negatives. Investigating the false positive
rate is ongoing work.

%Future work could evaluate our technique on more subjects, and
%investigate the false negative rate.

%our technique requires developers to manually
%add annotations to provide call relationships
%Second, 

%Third, for subject program
%that uses native methods, 

\subsubsection{Experimental Conclusions}
Findings: programmers have noted this problem, but often overlook special cases like error handle, subtype

 Reporting on violations is fine but what is more important are the pracitces and patterns that need to be adopted to make it easier to avoid the issue in the first place. That is why there is a Worker interface in this project
