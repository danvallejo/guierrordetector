\section{Empirical Evaluation}
\label{sec:evaluation}

%Our experiment investigate the following four research questions:

%\begin{itemize}
%\Item can our approach detect real bugs in multithreaded GUI applications?

%\Item is the reflection-aware call graph construction algorithm useful?

%\Item how effective is the proposed heuristic? is the reflection-aware call graph construction algorithm useful?
%\end{itemize}

Our experiment objective is three-fold: to demonstrate the effectiveness
of our approach in detecting real bugs in multithreaded GUI applications, to 
compare the reflection-aware call graph construction algorithms
with existing ones , and to evaluate the usefulness of the proposed 
 heuristics.  

First, we describe our subject programs (Section~\ref{sec:subjects}) and the experimental procedural (Section~\ref{sec:procedural}).
We then show that our technique detects bugs in real-world GUI applications on
four supported frameworks (Section~\ref{sec:errors}). We also compare our technique with
a straightfoward approach in Section~\ref{sec:straightforward},  compare the reflection-aware call graph
construction algorithms with existing algorithms (Section~\ref{sec:reflectionaware}), and
evaluate different graph search strategies in error detection (Section~\ref{sec:search}).
 Finally, we show that the set of proposed heuristics are effective
in removing redundant warnings and eliminate false positivies (Section~\ref{sec:filters}). 



\subsection{Subject Programs}
\label{sec:subjects}

%library code: 1.4 million

We use a set of open-source projects from Sourceforge, Google code and Eclipse plugin
market as evaluation subjects. To choose subjects for each supported GUI framework,
we first search the framework keywords (e.g., ``Java Swing'' or ``Java SWT'')
in the above three source repositories, and then select subjects based on the following
criterias. First, the subject must be a multithreaded GUI application, not
an open library. Second, the subject must undergone at least two years of active development,
and is listed in the first 3 results pages based on popularity. This permits us
to exclude many pre-mature subjects that may contain obvious errors.

The subjects used in our experiment are shown in Table~\ref{table:subjects}. Our
subjects include end-user applications, programming tools, and games. Specifically:

\begin{itemize}
\Item \textbf{SWT desktop applications.}
%VirgoFTP~\cite{virgo} and FileBunker~\cite{filebunker} are two useful SWT desktop applications.
VirgoFTP~\cite{virgo} implements a simple FTP client based on Java and SWT UI library, and
FileBunker~\cite{filebunker} is a file backup application which uses one or more GMail accounts as its backup repository.

\Item \textbf{Eclipse plugins.}
HudsonEclipse~\cite{hudson} adds several new tools to  monitor Hudson build status from Eclipse.
EclipseRunner~\cite{eclipserunner} extends capability of running launch configurations in Eclipse IDE.

\Item \textbf{Swing applications.} %S3dropbox~\cite{s3dropbox} and SudokuSolver~\cite{sudokusolver} are two Swing desktop applications.
S3dropbox~\cite{s3dropbox} implements a client that allows users
to drag and drop files, which are then uploaded to their Amazon S3 accounts. SodokuSolver~\cite{sudokusolver}
computes Sodoku solutions efficiently using mutlithreading execution. 

\Item  \textbf{Android applications.} %MyTracks~\cite{mytracks}, Fennec~\cite{fennec}, and SGTPuzzles~\cite{sgtpuzzles} are three Android applications.
MyTracks~\cite{mytracks}, developed by Google, records users' GPS tracks in outdoor activities, and provides
interfaces to visualize them on Google Maps. Fennec~\cite{fennec}, developed by Mozilla, is the Mozilla
Firefox web browser for mobile devices. SGTPuzzles~\cite{sgtpuzzles} is a small logic single-player game.

\end{itemize}


\begin{table}[t]
\begin{center}
 \fontsize{9pt}{\baselineskip}\selectfont
\hspace*{-0.2cm}
\setlength{\tabcolsep}{.70\tabcolsep}
\begin{tabular}{|p{4.4cm}||c|c|c|c|}
\hline
 Program (version) & LOC & Classes & Methods \\
\hline \hline
\multicolumn{4}{|l|}{SWT desktop applications}   \\
 \hline
 VirgoFTP (1.3.5) &  2293 &  20 &  165  \\
 \hline
 FileBunker (1.1.2)&  14237 &  150 &  1106  \\
 \hline
 \hline
\multicolumn{4}{|l|}{Eclipse plugins}   \\
 \hline
 EclipseRunner (1.1.0) &  3101 &  48 &  354\\
 \hline
 HudsonEclipse(1.0.9)&  11077 &  74 &  649 \\
 \hline
 \hline
\multicolumn{4}{|l|}{Swing desktop applications}   \\
 \hline
 S3dropbox (1.7) &  2353 &  42  &  224 \\
 \hline
 SudokuSolver (1.06)&  3555 &  10 &  62 \\
 \hline
 \hline
\multicolumn{4}{|l|}{Android mobile applications}   \\
 \hline
 SGTPuzzler (v9306.11)&  2220 &  16 &  148 \\
 \hline
 Fennec (10.0.a.rc3)&  8577 &  51 &  620 \\
 \hline
 MyTracks (1.1.13.rc4)&  20297 &  143 &  1374 \\
\hline
\hline
 Total &  64215 &  554 &  4702 \\
\hline
\multicolumn{4}{l}{}   \\
\hline
 GUI framework (version) & LOC & Classes & Methods  \\
\hline \hline
 SWT (3.6)&  129942 &  999 &  9643 \\
\hline
 Eclipse plugin development (3.6.2)&  460830 &  6630 &  37183 \\
\hline
Swing (1.6)&  167961 &  878 &  13159 \\
\hline
 Android (3.2)&  683289 &  5085 &  10584 \\
\hline
\hline
 Total &  1442022 &  13592 &  70569\\
\hline
\end{tabular}

\end{center}
\vspace{-15pt}
\Caption{{\label{table:subjects} Open source programs
used in our evaluation. Column ``LOC'' is the number of non-blank, non-comment lines
of code, as counted by LOCC~\cite{locc}.  Each program is analyzed
together with its dependent GUI framework, as listed in
the bottom table.} }
\end{table}


\subsection{Procedural}
\label{sec:procedural}

We run our tool on each subject with three call graph construction
algorithms: RTA~\cite{rta}, 0-CFA, and 1-CFA~\cite{kcfa}.  When running
each call graph construction algorithm on three Android applications, we
use two configurations: with and without the reflection-aware algorithm 
(Section~\ref{sec:cg}).  We did not use other expensive algorithms like $k$-CFA ($k >$ 1),
because they can not scale to large programs.

We inspect the source code of each subject to determine whether any
native methods or project-specific GUI interaction patterns are implemented.
We found SGTPuzzler is the only subject that uses native methods to interact with
underlying operating systems. Thus, we manually added 3 \CodeIn{@CalledByNative}
annotations for it. Also two subjects MyTracks and Fennec employ a self-defined
way to interact with the GUI framework. For example, in MyTracks, all non-UI
threads are initialized via using the library method \CodeIn{android.os.handler.handleCallback}.
% the \CodeIn{safeInvokeXX}
%method in the XXX subject first checks whether the current thread is the UI-thread, and
%then determines whether to direct access a GUI object or wrap the code with
%message passing. 
Thus, we add 5 user-defined heuristic filters for these two subjects.%to indicate
%that they are safe to be invoked from any thread. 
In this experiment, none of the paper
authors was familar with the subjects; but we found it is quite easy
to add extra annotations and heuristic filters. All these
manual parts are finished in less than 60 minutes.

After provided with necessary annotations and heuristic filters,
our tool works in a fully-automatic, push-bottom manner. For each output
warning, we manually determine its validity by either searching the
bug repository to check whether the same error has been reported before,
or submitting a new bug report, or writing a test driver to reproduce
the error.



\begin{table*}[ht]
\begin{center}
 \fontsize{9pt}{\baselineskip}\selectfont
\hspace*{-0.2cm}
\setlength{\tabcolsep}{.35\tabcolsep}
\begin{tabular}{|l||c|c|c||c|c|c||c|c|c||c|c|}
\hline
 Subject&  \multicolumn{9}{|c||}{Our Technique} & \multicolumn{2}{|c|}{The Approach in}  \\
\cline{2-10}
 Program  &  \multicolumn{3}{|c|}{RTA / reflection aware}& \multicolumn{3}{|c|}{0-CFA / reflection aware} & \multicolumn{3}{|c||}{1-CFA / reflection aware} & \multicolumn{2}{|c|}{Section~\ref{sec:straightforward}}  \\
\cline{2-12}
 & CG Size & \#Warning & \#Bug & CG Size & \#Warning & \#Bug & CG Size & \#Warning & \#Bug & \#Warning & \#Bug\\
\hline \hline
\multicolumn{12}{|l|}{SWT desktop applications}   \\
 \hline
 VirgoFTP&  12401 &  2 &  2 & 10858 & 2 & 2 & 43598 & 2 & 2& 149 & 2 \\
 \hline
 FileBunker &  18951 &  1 &  0 & 15743 & 0 & 0 & 76088 & 2 & 1& 693 & 1 \\
 \hline
 \hline
\multicolumn{12}{|l|}{Eclipse plugins}   \\
 \hline
 EclipseRunner&  11248&  6 &  1 & 7201 & 6 & 1 & 26911 & 6 & 1& 202 & 1 \\
 %\hline
 %FileSync&  12132 &  xx &  xxx &8235 & xx & xx& 32565 & 1 & 1 & 331 & 1 \\
 \hline
 HudsonEclipse& 18473 &  2 &  1 & 15814 & 2 & 1& 56645 & 3 & 1 & 182 & 1 \\
 \hline
 \hline
\multicolumn{12}{|l|}{Swing desktop applications}   \\
 \hline
 S3dropbox & 37751 &  0 &  0 & 30609 & 0 & 0 & 115324 & 1 & 1 & 210 & 1 \\
 \hline
  SudokuSolver&  27730&  3 &  2 & 20907 & 3 & 2 & 39299 & 2 & 2 & 356 & 2 \\
 \hline
 \hline
\multicolumn{12}{|l|}{Android mobile applications}   \\
 \hline
 SGTPuzzler & 13631 / 13865&  12 / 16 &  0 / 0 & 9546 / 9682& 4 / 4& 1 / 1 & 35198 / 35756 & 1 / 1  & 1 / 1& 104 & 1 \\
 \hline
 Fennec & 14058 / 14387 &  1 / 1 &  0 / 0 & 8263 / 8898 & 1 / 1 & 0 / 0& 29125/ 31759 & 3 / 3 & 1 / 1& 433 & 1 \\
 \hline
 MyTracks & 24036 / 24036 &  161 / 220 & 0 / 0 & 10803 / 13645 & 119 / 119 & 0 / 0 & 39235 / 110977 & 1 / 1 & 0 / 1 & 1192 & 1 \\
\hline
\hline
 Total & 178279 / 178842 &  188 / 251 & 6 / 6 & 129744 / 133357 & 137 / 137& 7 / 7 & 461423 / 536358& 21 / 21 & 10 / 11 & 3521 & 11 \\
\hline
\end{tabular}
\end{center}
\vspace{-15pt}
\Caption{{\label{table:results}Experimental results in finding invalid thread access errors
in multithreaded GUI programs. Column ``CG Size'' shows the
call graph size in terms of node number. Column ``\#Warning''
shows the number of warnings issued by our tool. Column ``\#Bug'' shows
the actual bugs found. Columns ``RTA / reflection aware '', ``0-CFA / reflection aware'',
and ``1-CFA / reflection aware'' show the results of using different
call graph construction algorithms with (before ``/'') or without (after ``/'')
using the reflection-aware algorithm on Android applications. In each cell of the table
for Android applications, a slash ``/'' separates the result of
using normal call graph construction algorithm and the reflection-aware
call graph construction algorithm. The 6 errors found in VirgoFTP, FileBunker,
S3dropbox, and SudokuSolver are previously  unknown. As a comparison, the
results of a straightforward approach (Section~\ref{sec:straightforward}) are show at the far right.} }
\end{table*}

\subsection{Results}
\label{sec:results}

%Table~\ref{table:results} show the errors in
%our subject programs, and Table~\ref{table:filters} show
%the effect of applying each heuristic filter.

\subsubsection{Errors in Multithreaded GUI Applications}
\label{sec:errors}

Table~\ref{table:results} tabulates the results of running our
tool with three different call graph construction algorithms
on the subject programs. From these results, we can see our
tool found real errors from each subject on  four
supported GUI frameworks. The results vary slightly according
to the specific call graph algorithm used. In general, using
the 1-CFA algorithm finds most errors (11 in total) and least
warnings (21 in total). The tradeoffs of using other
algorithms will be explained in Section~\ref{sec:reflectionaware}.

We further examined the detailed results for each subject.
Among the 21 warnings issued by using the 1-CFA algorithm, 2
are false positives, and 8 are redundant:
they have the same causes as the rest 11 warnings. 

%The primary
%reason for the false positive is because the precision of call graph,
%while the reason for redundant warnings is because multiple
%methods ca
%Even a context-sensitive call graph construction algorithm is used,

%why finds a bug in SGTPuzzle, because the \CodeIn{SmallKeyBoard} has been
%explicitly created in this main class \CodeIn{SGTPuzzles} \CodeIn{setKeyboardVisibility},
%thus, even without reflection-aware


\begin{figure}[t]
\begin{CodeOut}
\begin{alltt}
/* In class: com.tomczarniecki.s3.gui.DeleteBucketAction */
59.private void deleteBucket() \{
60.    executor.execute(new Runnable() \{
61.        public void run() \{
62.            try \{
63.                controller.deleteCurrentBucket();
64.            \} catch (Exception e) \{
65.                logger.info("Delete failed", e);
66.                deleteError(); 
67.            \}
68.       \}
69.    \});
70.\}

77.private void deleteError() \{
78.    String text = "Cannot delete folder .....";
79.    display.showErrorMessage("Delete failed", 
        String.format(text, controller.getSelectedBucketName())); 
80.\}
\end{alltt}
\end{CodeOut}
\vspace*{-2.0ex} \Caption{{\label{fig:swingerror} An
invalid thread access error reported by our tool
in the S3dropbox Swing application. The error occurs when
the method \CodeIn{deleteCurrentBucket} invoked at line 63 throws
an exception, which causes method \CodeIn{deleteError} to
access a Swing GUI object \CodeIn{display} at line 79 from
a non-UI thread. This error is previously-unknown, and has been confirmed by the S3dropbox developers.
}} %\vspace{-5mm}
\end{figure}

Figures~\ref{fig:swingerror} amd~\ref{fig:pluginerror} show
two real errors our tool has found. Figure~\ref{fig:swingerror}
shows an invalid thread access error in the S3dropbox
program. This error happens when the \CodeIn{deterCurrentBucket}
method at line 63 throws an exception. This error can be hard to
detect by testing, since the test must  execute that specific exception-handling
path. We reported this error to the S3dropbox developers. Tom Czarniecki,
a key developer of S3dropbox confirmed this potential single-GUI-thread
violation. He mentioned that the S3dropbox project has already
used certain design patterns and practice to avoid such violations (e.g.,
actions for UI interaction are encapsulated into a \CodeIn{Worker} interface),
but they still overlooked  the reported code snippet. 

Another reason of missing this violation is because some GUI frameworks like Swing
do not provide any run-time checks for invalid thread accesses. Most
of the time, programmers can get away with an apparently ``well behaved'' Swing
GUI that actually breaks the single-GUI-thread rule. However, As clearly
indicated in the official documentation, Swing is a single-GUI-threaded GUI
toolkit, and is not thread-safe. All Swing code must be executed in the UI-thread and
invoking Swing code from multiple threads risks thread interference
or memory-consistency errors, particularly in the multi-core era.



\begin{figure}[t]
\begin{CodeOut}
\begin{alltt} 
/* com.eclipserunner.views.impl.RunnerView */
179.private void initializeResourceChangeListener() \{
180.  ResourcesPlugin.getWorkspace().addResourceChangeListener(
        new IResourceChangeListener() \{
181.      public void resourceChanged(IResourceChangeEvent event) \{
182.        refresh();
183.      \}
184.  \}, IResourceChangeEvent.POST\_CHANGE);
185.\}

414.public void refresh() \{
415.  getViewer().refresh();
416.\}
\end{alltt}
\end{CodeOut}
\vspace*{-2.0ex} \Caption{{\label{fig:pluginerror} An
invalid thread access error reported by our tool
for the EclipseRunner plugin. In Eclipse, the
callback method \CodeIn{resourceChanged} at line 181
is invoked by non-UI threads when a \CodeIn{ResourceChangeEvent}
happens. However, the \CodeIn{refresh} method directly accesses
GUI objects (to refresh the view at line 415) without any
protection and thus triggers the error.
This error has been reported by other users 13 months after the
buggy code was checked in, and fixed by developers.
}} %\vspace{-5mm}
\end{figure}

Figure~\ref{fig:pluginerror} shows an error found in the EclipseRunner
plugin. This error is event-related. It happens when a 
\CodeIn{ResourceChangeEvent} happen, which then invokes the \CodeIn{refresh}
method at line 415 to update the user interface. In EclipseRunner
, the \CodeIn{refresh} method has been
called by 6 different methods in non-UI threads. Thus, our
tool issues 6 separate warnings for each caller method to indicate
different ways to trigger this error, despite that all the 6 warnings
acutally have the same error cause.

As reflected in our experiments, we found GUI developers have already
used design patterns, runtime checks, and testing to avoid violating
the single-GUI-thread rule. However, due to the huge space of
possible UI interactions, potential
invalid thread access errors still exist in many corner cases and
can be hard to find. Thus, a static analysis as presented in this paper
would be useful.

\vspace{1mm}

\noindent \textbf{\textit{Summary.}}Our technique finds real errors
in multithreaded GUI applications on four different frameworks with
an acceptable false positive rate.

\subsubsection{Comparing to a Straightforward Approach}
\label{sec:straightforward}

As briefly mentioned in Section~\ref{sec:finding}, one way to avoid
invalid thread access errors is to wrap every GUI access code
with message passing methods (i.e., the safe methods). Thus, a
straightforward way to detect potential invalid error
is to check every method to determine whether all GUI-accessing code
is probably wrapping by safe methods. If not, this approach
issues a warning for every violation. 
%If not, this approach issues warnings.

We implemented this straightforward approach and experimentally compared it
with our technique.% presented in Section~\ref{sec:technique}.
The far right column in Table~\ref{table:results} shows the results.
The results show that this straightforward approach can identify every
error that our technique has found, but issue a huge number of false
positives and redundant warnings. The primary reason is that this simple approach
does not globally reason about calling relationship between
threads, UI-accessing methods, and safe methods, thus it often incorrectly
classifies GUI accessing code which will never be executed
in a non-UI thread as erroneous. Furthermore, our technique
outputs a method call chain in each reported error, which can help
developers understand how an invalid thread access error is triggered.



\vspace{1mm}

\noindent \textbf{\textit{Summary.}} Our technique provides
richer contextually information for a potential error, and 
is significantly more precise than the straightforward approach.

\subsubsection{Comparing Call Graph Construction Algorithms}
\label{sec:reflectionaware}

We next compare three call graph construction algorithms (RTA, 0-CFA, and 1-CFA)
used in our experiments. As shown in Table~\ref{table:results},  1-CFA found
more errors with less warnings than the other two algorithms. This
is because RTA and 0-CFA do not consider the calling context when
constructing a call graph, and it may
mix the same method calls by different callers into a single node, and
then introduce imprecision. 

%Although the call graphs
%built by 1-CFA are significantly larger, XXX, the algorithm still runs
%in a practical amouont of time. For example,
%our tool finished building the call graph for the largest subject (xxx)
%within xxx minutes.

The results in Table~\ref{table:results} also show that using a reflection-aware
call graph construction algorithm is useful in finding errors in Android
applications. One error from the MyTracks Android application
can only be found by using the reflection-aware call graph construction algorithm.
This is because the error-related GUI object (\CodeIn{msgTextView}) in MyTracks is created
reflectively as follows.

\begin{CodeOut}
\begin{alltt}
/*In class: com.google.android.apps.mytracks.StatsUtilities*/
97.  public void setLatLong(int id, double d) \{
98.     TextView msgTextView = (TextView) activity.\textbf{findViewById}(id);
99.     msgTextView.setText(LAT\_LONG\_FORMAT.format(d));
100.\}
\end{alltt}
\end{CodeOut}
 
In constrast, the error-related GUI objects in the other two Android applications (Fennec
and SGTPuzzler) are created explicitly, so that a normal call graph construction
algorithm can create a sufficiently complete call graph.

%Without
%replacing the reflection calls with explicit object creation instructions,
%a call graph construction algorithm will
%conclude that no GUI object has been created, and thus miss many UI-related
%edges in the resulting graph.

\vspace{1mm}

\noindent \textbf{\textit{Summary.}} Using the 1-CFA call graph construction
algorithm is more effective in finding errors, and our reflection-aware call
graph construction algorithm is critical to find errors on some
Android applications.

\subsubsection{Comparing Graph Search Strategies}
\label{sec:search}

Our error detection algorithm in Figure~\ref{fig:detectalgorithm} uses BFS to search potential
single-GUI-thread violations. However, other graph search strategies such as
Depth-First Search (DFS) and exhaustive path search can also be employed. To
understand trade-offs between using different search strategies, we 
implemented two variants of the algorithm in Figure~\ref{fig:detectalgorithm}, and
evaluated them on the same subjects with the identical environment settings.

The first variant uses DFS by changing line 7 of Figure~\ref{fig:detectalgorithm}
to \CodeIn{getReachableStartsByDFS(cg)}, changing line 9 of Figure~\ref{fig:detectalgorithm}
to \CodeIn{queue} $\leftarrow$ \CodeIn{an empty stack}, and changing queue operations
(\CodeIn{enqueue} and \CodeIn{dequeue}) to the corresponding stack operations (\CodeIn{push} and \CodeIn{pop}).
This variant missed one error in the FileBunker subject, and found 8 errors
in total.  That is because DFS searches deeper into the graph, and the path it finds
may not be the shortest path.

%Hundson: 3, 1
%EclipseRunner: 6, 1
%S3dropbox: ? , ?  too many warnings time: 31678 (45528 chains), running out of memory
%SudokuSolver:  2, 1 too many warnings: 1700, 2090
%VirgoFTP: 2, 2
%FileBunker:  2, 0  //very long method calls
%Fennec: 3, 1
%SGTPuzzler: 1, 1
%MyTracks: 1, 1

The second variant uses exhaustive search to find potential errors. This variant
enumerates all non-cyclic paths from the start nodes, and checks whether each
path spawns a new thread and accesses GUI object without using safe methods. As
a result, this variant run out of memory on all subjects. We further checked
the traversed paths before the algorithm terminated, and found none of the explored
path revealed a valid error. This is because the number of
non-cyclic paths in a graph is exponetially large to the graph node number.
For a realistic call graph which often contains thousands of nodes, it is impossible to enumerate all
paths in practice. 


\vspace{1mm}

\noindent \textbf{\textit{Summary.}} Using BFS finds more errors than DFS and
exhaustive search in our error detection algorithm. 


\subsubsection{Evaluating Heuristic Filters}
\label{sec:filters}

We measure the effectiveness of the proposed heuristic filters of
Section~\ref{sec:heuristic} in terms of the number of removed false positives and
redundant warnings.

Table~\ref{table:filters} tabulates the results.
Our static analysis outputs thousands of warnings for the evaluated
subjects, but the proposed heuristic filters effectively reduces
the number of likely invalid warnings. On average, five heuristic
filters remove 99.96\% of the reported warnings as false positives
or redundant warnings. Specifically, heuristic filters 1 and 2 identified
32.48\% of the warnings as likely false positives, and filters 3, 4,
and 5 identified 67.48\% of the warnings as redundant warnings.
We further manually check each filtered warning, and found no
real errors have been removed.

\vspace{1mm}

\noindent \textbf{\textit{Summary.}} Although a static analysis may
issue false postivies or redundant reports, a set of well-designed
heuristics can still remove most of them.
%Heuristic filter, effective.

\begin{table}[t]
\begin{center}
 \fontsize{9pt}{\baselineskip}\selectfont
\setlength{\tabcolsep}{.81\tabcolsep}
\hspace*{-0.2cm}
\begin{tabular}{|l||c|c|c|c|c|c|}
\hline
 Subject & Before & \#W & \#W & \#W & \#W& \#W \\
 Program & Filtering & $F_1$ & $F_2$ & $F_3$&$F_4$ & $F_5$\\
\hline \hline
\multicolumn{7}{|l|}{SWT desktop applications}   \\
 \hline
 VirgoFTP &  21 &  21 &  3 & 3 &  2 & 2\\
 \hline
 FileBunker &  4494 &  3210 &  3210 &  642 &  2 & 2\\
 \hline
 \hline
\multicolumn{7}{|l|}{Eclipse plugins}   \\
 %\hline
 %EclipseRunner &  xxxx &  xx &  xxx & xx &  xxx & xx\\
 \hline
 EclipseRunner&  1621 &  1487 &  1487 & 1487 &  6 & 6\\
 \hline
 HundsonEclipse&  3192 &  2417 &  1696 & 9 &  3 & 3\\
 \hline
 \hline
\multicolumn{7}{|l|}{Swing desktop applications}   \\
 \hline
 S3dropbox&  45528 &  31978 &  30975 & 9 &  1 & 1\\
 \hline
 SudokuSolver &  58 &  58 &  58 & 2 &  2 & 2\\
 \hline
 \hline
\multicolumn{7}{|l|}{Android mobile applications}   \\
 \hline
 SGTPuzzler&  2 &  1 &  1 & 1 &  1 & 1\\
 \hline
 Fennec &  122 &  84 & 84 & 84 &  9 & 3\\
 \hline
 MyTracks &  1176 &  441 &  441 & 441 &  63 & 1 \\
\hline
 \hline
 Total &  56214 &  39708 &  37955 &  2678 &  89 & 21 \\
 \hline
\end{tabular}
\end{center}
\vspace{-15pt}
\Caption{{\label{table:filters}Number of warnings after applying a set of
heuristic filters (Section~\ref{sec:heuristic}). Column ``Before Filtering''
shows the output warning number using the reflection-aware 1-CFA algorithm.
Column ``\#W $F_i$'' represents the number of remaining warnings after applying
filters 1 -- $i$. Numbers in last column are taken
from Table~\ref{table:results}.
Results of using other algorithms show similar patterns, which are omitted for brevity.} }
\end{table}


\subsection{Discussions}

\noindent \textbf{\textit{Performance and scalability.}} Our tool
has been evaluated on 9 subjects with over 67 KLOC and dependent
libraries with over 1.4 MLOC, showing good scalability.
For each subject program, it runs in a practical amount of time. Our evaluations
were conducted on a 2.67GHz Intel Core PC with 4GB
physical memory (1GB is allocated for JVM), running Fedora 16.
For the most time-consuming subject, MyTracks, our tool finished the whole analysis
within 686 seconds using the most expensive call graph construction
algorithm: the reflection-aware 1-CFA algorithm. Analyzing
other subjects using different algorithms took less time.
The performance has not been tuned, and we believe
that it could be improved further.

%Our experiments also demonstrate that our approach has good scalability.
%Together with all dependent library code, our tool has analyzed
%XXX LOC in total  xxx the largest program.
%Our tool has analyzed xxx LOC in total, and together with all dependent libraries,
%the largest subject our approach can analyze and detect errors has
%over XXX LOC, xx classes, and xxx methods.


\vspace{1mm}

\noindent \textbf{\textit{Threats to validity.}}
There are two threats to validity in our evaluation. 
The major threat is the degree to which the subject programs
used in our experiment are representative of true practice.
In our evaluation, the studied subjects were selected with a bias towards popularity from
three open-source repositories. Another threat is that we only employed three
well-known call graph construction algorithms (i.e., RTA, 0-CFA, and 1-CFA) 
in our evaluation. It is still unknown whether using other 
call graph construction algorithms such as XTA~\cite{xta} or
an incremental algorithm~\cite{inccg} may achieve better results.


\vspace{1mm}

\noindent \textbf{\textit{Limitations.}}
Our technique is limited in two aspects. First, it only considers
non-UI threads that are spawned by the UI-thread after the GUI
is initialized, and ignores other possible non-UI threads
that are created during the pre-initialization GUI work. One way
to remedy this limitation is to design an analysis to identify
those non-UI threads created before a GUI is launched.
Second, our tool cannot compute call relationships
for inter-process communication between components. It also
requires users to manually add annotations to characterize
call relationships that involve native methods. This limitation
may lead to false negatives. Investigating the false positive
rate is ongoing work.

%Future work could evaluate our technique on more subjects, and
%investigate the false negative rate.

%our technique requires developers to manually
%add annotations to provide call relationships
%Second, 

%Third, for subject program
%that uses native methods, 

\vspace{1mm}

\noindent \textbf{\textit{Experimental Conclusions.}}
Invalid thread access errors  can be subtle to detect in many cases.
The technique presented in this paper offers a promising solution.
Our technique finds real-world errors, and issues
%Compared with the straightforward approach to check every
%GUI accessing code isolatedly, our technique produces
significantly less warnings when compared to a straightforward
approach. In addition, the reflection-aware call graph construction algorithm
is critical in detecting errors on Android applications, and our
proposed heuristic filtering rules are useful to reduce invalid warnings.

%in reducing programmers'
%burden to inspect the output result.

%In practice, programmers
%have designed practices and patterns to avoid such errors in the first
%place, but can easily overlook corner cases like exception-handling
%methods.


%Findings: programmers have noted this problem, but often overlook special cases like error handle, subtype

% Reporting on violations is fine but what is more important are the practices and patterns that need to be adopted to make it easier to avoid the issue in the first place. That is why there is a Worker interface in this project
