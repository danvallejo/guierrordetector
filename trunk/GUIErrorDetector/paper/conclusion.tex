\smallstep
\section{Conclusion and Future Work}

This paper presented a general technique to find invalid
thread access errors in multithreaded GUI applications. 
Our technique statically explores paths in a call graph to check
whether a non-UI thread can access a GUI object.
It uses a reflection-aware call graph construction algorithm
to build a good call graph, and employs a set of error
filters to filter likely false positives and redundant warnings.
%Since graphical user interfaces are often inadequately tested due
%to the enormously large space of UI interactions and
%extreme resource constraints during their development, we
%believe that providing a static analysis can allow developers to find
%more potential errors %before the application is released to customers (XXX),
% to improve software reliability.
We demonstrated the usefulness of our technique
by an evaluation on \subnum programs built on 4 popular GUI
frameworks. 

The source code of our tool implementation is available at:


\noindent \url{http://guierrordetector.googlecode.com}


\vspace{1mm}


Besides general issues such as performance and ease of use, our future
work will concentrate on the following topics:

%\begin{itemize}
%\item
\textbf{Integration with dynamic and symbolic analyses.} The technique 
presented in this paper is a call graph-based, pure static analysis. 
It suffers from false positives for many GUI applications,
due to the conservative nature of static analysis.
A possible way to reduce the number of false
positives is to integrate with
dynamic analyses~\cite{Jiang:2008:PPS:1453101.1453110}
or symbolic analyses~\cite{xie05:symstra, Pasareanu:2011, halfond09issta, BMF97}
by using more accurate information to guide call graph exploration.

%\item
\textbf{Unit testing multithreaded GUI programs.} Besides
a static analysis,  software testing is another
way to improve software quality.  Although many
testing techniques have been developed recently, few of them can be applied
to unit test multithreaded GUI programs to find potential errors \textit{earlier}. We
plan to investigate how to apply recent advance in automated
testing~\cite{Staats:2011:PTO:1985793.1985847, Jagannath:2011:IMU:2025113.2025145, Muccini_Bertolino_Inverardi_2004, Ricca:2001:ATW:381473.381476, Harman:2007}
to the context of multithreaded GUI applications.


%\item
\textbf{Fixing potential GUI errors.} After an error is revealed, fixing
it and verifying the patch is often time-consuming. Fixing concurrency
bugs has become especially critical in the multicore era.
Recently, work has been done
on automatically repairing test scripts for GUI applications~\cite{Daniel:2011:AGR:2002931.2002937, Huang:2010:RGT:1828417.1828465}. However, none of them focuses on repairing
the GUI program to patch a revealed error. Thus, we are interested in 
developing automated error fixing techniques for
multithreaded GUI applications.

%\end{itemize}
