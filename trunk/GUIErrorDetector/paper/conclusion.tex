\section{Conclusion and Future Work}

In this paper, we presented a general technique to find invalid
thread access errors in multithreaded GUI applications. 
Our technique statically explores paths in a call graph to check
whether a non-UI thread can access a GUI object.
It uses a combined RTA and k-CFA algorithm to construct
a good call graph, and employs a set of heuristics to
filter likely false positives and redundant reports.
Since graphical user interfaces are often inadequately tested due
to the enormously large space of UI interactions and
extreme resource constraints during their development, we
believe that providing a static analysis can allow developers to find
more potential errors %before the application is released to customers (XXX),
and thus encourage GUI developers to improve software reliability.

We have demonstrated that our technique can be both practical and useful
by an evaluation on 10 subjects acrosss 4 popular GUI
frameworks. Our experiments have shown that we can find bugs
in real-world programs.


Besides general issues such as performance or ease of use, our future
work will concentrate on the following topics:

%\begin{itemize}
%\item
\textbf{Integration with dynamic and symbolic analyses.} The technique 
presented in this paper is a pure static analysis. It
uses a combined RTA and k-CFA  algorithm to construct a relatively
complete and precise call graph. However, our technique still suffers from
a false positive rate for many GUI applications, due to the conservative
nature of static analysis. A possible way to reduce such false
positives is to integrate the current static analysis with
dynamic analyses~\cite{Jiang:2008:PPS:1453101.1453110, ZhangSBE2011}
or symbolic analyses~\cite{xie05:symstra, Pasareanu:2011, halfond09issta, BMF97}
by employing more accurate information to guide call graph exploration.

%\item
\textbf{Unit testing multithreaded GUI programs.} Besides
a static analysis,  software testing is another
way to improve software quality.  Although many
GUI testing techniques have been developed recently, few of them can be applied
to unit test multithreaded GUI programs to find potential errors \textit{earlier}. We
are interested to investigate how to apply recent advance in automated
software testing~\cite{Staats:2011:PTO:1985793.1985847, Jagannath:2011:IMU:2025113.2025145, Muccini_Bertolino_Inverardi_2004, Ricca:2001:ATW:381473.381476, Harman:2007}
to the context of multithreaded GUI applications.


%\item
\textbf{Fixing potential GUI errors.} After a potential error is revealed, fixing
it and verifing the patch has often been another important and time-consuming process. Fixing concurrency
bugs has become especially critical in the multicore era.
Recently, a few work has been done
on automatically repairing test script for GUI applications~\cite{Daniel:2011:AGR:2002931.2002937, Huang:2010:RGT:1828417.1828465}. However, none of them focuses on repairing
the GUI program to fix a revealed bug. Thus, we are interested in 
developing automated error fixing techniques for
multithreaded GUI applications.

%\end{itemize}
