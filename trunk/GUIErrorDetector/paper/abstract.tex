\begin{abstract}
To keep a Graphical User Interface (GUI) responsive and active, a GUI
application often has a main \textit{UI thread} (or \textit{event dispatching thread})
and spawns separate threads to handle lengthy operations in the background,
such as expensive computation, I/O tasks, and network requests.
Many GUI frameworks require all GUI objects to be accessed exclusively by the
UI thread. If a GUI object is accessed through a non-UI thread,
an \textit{invalid thread access} error occurs and the whole
application may abort. 

This paper presents a general technique to find such \textit{invalid thread access}
errors in multithreaded GUI applications. We formulate finding invalid
thread access errors as a call graph reachability problem with
thread spawning as the sources and GUI object accessing as the sinks. 
Standard call graph construction algorithms fail to build a good
call graph for some modern GUI applications, because of heavy use of reflection.
Thus, our technique builds reflection-aware call graphs.


We implemented our technique and instantiated it for four popular Java GUI
frameworks: SWT, the Eclipse plugin development framework, Swing, and
Android. In an evaluation on \subnum programs comprising \totaloc LOC, our technique
found 5 previously-known errors and 7 new ones.

\end{abstract}

