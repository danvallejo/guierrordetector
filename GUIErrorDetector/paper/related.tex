\section{Related Work}

Work related to this paper falls into two main categories; (1)
techniques and tools to analyze and test GUI programs; and (2)
bug finding techniques for general multithreaded programs.

\subsection{Analysis and Testing for GUI Applications}

Automated GUI testing is a challenging tasks~\cite{Bertolino:2007:STR:1253532.1254712,
Harrold:2000:TR:336512.336532}.
 Many GUI analysis and testing techniques have been proposed in the past decade; some have been
implemented as tools and adopted by practitioners. Existing techniques automate
some aspect(s) of GUI testing including model creation
(for model-based testing)~\cite{androidtesting, Xie:2006:MTC:1172962.1172990},
test-case generation~\cite{YuanCohenMemonTSE2011, YuanMemonICSE2007},
test oracle creation~\cite{MemonFSE2000}, test execution~\cite{MemonFSE2001},
and test script repairing~\cite{Huang:2010:RGT:1828417.1828465, Daniel:2011:AGR:2002931.2002937}.
For example,
Kervinen et al.~\cite{Kervinen06model-basedtesting} present a formal model and architecture for testing
concurrently running applications. The behavior of the system under test
needs to be specified manually as a labeled transition system. 
Guitar~\cite{YuanCohenMemonTSE2011, YuanMemonICSE2007, Xie:2006:MTC:1172962.1172990}
is a GUI testing framework for Java and Microsoft Windows applications. 
Yuan and Memon~\cite{YuanMemonICSE2007} generate event-sequence based test cases for GUI
programs. They proposed a model-based approach for testing GUI-based applications.
Their technique can generate test cases automatically using a structural event
generation graph. 
%Their approach targets Java desktop applications, which
%are quite different from the Android mobile environment (XXXX).
However, testing is often insufficient to ensure the correctness of a GUI program
due to the huge space of possible interactions
with a GUI. The large number of possible interactions
result in a large number of
test inputs requiring substantile human effort.
The second
problem is to determine the coverage of a set of test
cases~\cite{MemonFSE2001}. For conventional software, coverage is measured
using the amount and type of underlying code exercised.
These measures do not work well for GUI testing, because
what matters is not only how much of the code is tested,
but in how many different possible states of the software
each piece of code is tested. This makes it hard to determine
the completeness of a test suite. %All of the (XXXXX, so that GUI is often not well tested)
In contrast, as advocated in this paper, the primary advantage of
using a static analysis is that it can potentially explore
all paths to find possible bugs without actually executing the tested code. % (XXX use heuristic filtering)
Compared with software testing, a static analysis tool may
report false positives due to its conservative nature.
Nevertheless, as we demonstrated in our experiments, using a few simple heuristics
can dramatically reduce the number of false positives  to
an acceptable level.


Michail and Xie~\cite{michail05:helping} proposed a tool-based approach to help users avoid bugs
in GUI applications. Their approach monitors user's action in the background,
and gives a user a warning as well as the opportunity to abort the action, when
a user attempts an action that has led to problems in the past. 
As a comparison, our work aims to find unknown errors in a multithreaded
GUI applications, while their work aims to prevent an existing bug
from happening again.
%is significantly different than their


Recently, Payet and Spoto~\cite{Payet:2011:SAA:2032266.2032299} presented a static
analysis framework for Android programs based on  abstract
interpretation. Their framework focuses on the Android platform, and
 consists of 7 existing static analyses such as
nullness analysis, class analysis, and termination analysis.  However,
their framework does not support detecting the invalid thread access
errors, and uses a quite different abstraction than ours.
To the best of our knowledge, we are the first to identify the invalid
thread access problem on multithreaded GUI applications, and
our technique is a general one that is specifically designed to detect
the invalid thread access errors.

%To the best of our knowledge, there is no prior work to tackle this invalid thread access problem
% on multithreaded GUI applications.

\subsection{Finding Bugs in Multithreaded Programs}



A rich body of techniques have been developed to detect bugs in multithreaded programs~\cite{Huang:2011:PPC:2001420.2001438, Weeratunge:2010, Huang:2011:EST}.
Active testing frameworks like AssertFuzzer~\cite{Sen:2008} uses the trace information to detect
and create atomicity schedule violations. Static analysis tools such as Chord~\cite{Naik:2006}
 explore multithreaded bugs by statically analyzing the program. On the
other hand, runtime analysis tools such as Eraser~\cite{Savage:1997}  dynamically detect concurrency bugs using the lockset
algorithms or some criteria based automata. Bodden and Havelund~\cite{Bodden:2008} uses AOP for
race detection by providing three novel pointcuts to the AspectJ language. More recently,
Goldilock~\cite{Elmas:2007} uses the hybrid model that combines the happens-before and lock-based
approaches to identify data races based on an execution. By exploiting the insight
that a full generality of the happens-before is unnecessary in most data accesses,
FastTrack~\cite{Flanagan:2009} proposes an adaptive representation for the happen-before
relation to efficiently and precisely detect races at runtime. However, finding
invalid thread errors is quite different than detecting data races in
a multithreaded program. A data race occurs when two concurrent threads access
a shared variable and when at least one access is a write and the threads
use no explicit mechanism to prevent the accesses from being simultaneous. In contrast,
an invalid thread access error occurs when a non-UI thread accesses (reads or writes) a GUI object.
Unlike detecting potential data races, finding invalid thread access error does not require monitoring every shared-memory
reference and verify that consistent locking behavior is observed among different threads. Instead,
our technique only needs to track whether a non-UI thread can reach a GUI object or not.
How to adapt existing data race detection approaches to 
improve our technique is one of our future work.

%Significantly different from the concurrency bug finding work....

%They are often limited to detect errors manifested in a specific concrete execution.

An alternative way to find bugs in multithreaded programs is using model checking~\cite{Nori:2010:ESO, Inverardi:2000, Siegel:2008}.
By exhaustively exploring the thread scheduling space, a model checker can
report counter examples for the potential bugs with a counter example. For example,
CHESS~\cite{chess} dynamically explores the thread scheduling decisions to expose
concurrency bugs using a context-bounded approach. Shacham et al~\cite{Shacham:2007} uses
a model checker to construct the witness for data races reported by the lockset algorithm.
%(XXXX problem for applying to GUI).
Unfortunately, due to the exponential size of the search space,
 it is hard for model checking approaches to scale to a realistic multithreaded GUI application
 without compromising the detection capability. To our best knowledge,
we are not aware of any software model checking approach can scale to the subjects
as large as those used in our experiments. Compared with model checking, the static analysis
algorithm presented in this paper chooses call graph as the
abstraction level with a set of heuristics, and thus is much lightweighted. It is highly tailored to find the invalid thread
access errors instead of being a general property checking tool. For this reason,
our technique achieves good scalability with reasonable precision.



