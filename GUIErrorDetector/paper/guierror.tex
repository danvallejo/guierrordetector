

%\documentclass{acm_proc_article-sp}
\documentclass{sig-alternate}
\usepackage{multirow}
\usepackage{fancyheadings}
\usepackage{algorithmic}
\usepackage{amssymb}
\usepackage{xspace}
\usepackage{pslatex}
\usepackage{microtype}
\usepackage{subfigure}
%\usepackage{todonotes}
\input{macros}

% % Add line between figure and text
 \makeatletter
 \def\topfigrule{\kern3\p@ \hrule \kern -3.4\p@} % the \hrule is .4pt high
 \def\botfigrule{\kern-3\p@ \hrule \kern 2.6\p@} % the \hrule is .4pt high
 \def\dblfigrule{\kern3\p@ \hrule \kern -3.4\p@} % the \hrule is .4pt high
 \makeatother
 % If there is a line, you can get away with reducing the separation between
 % figures and text.  Don't do this without the line, though.
 \addtolength{\textfloatsep}{-.5\textfloatsep}
 \addtolength{\dbltextfloatsep}{-.5\dbltextfloatsep}
 \addtolength{\floatsep}{-.5\floatsep}
 \addtolength{\dblfloatsep}{-.5\dblfloatsep}


\begin{document}

\title{Finding Errors in Multithreaded Graphical User Interfaces}
%\subtitle{[Extended Abstract]
%\titlenote{This work is sponsored by}


\author{
\alignauthor Sai Zhang \quad Hao Lu \quad Michael D. Ernst\\
       \affaddr{Department of Computer Science \& Engineering}\\
       %\affaddr{1932 Wallamaloo Lane}\\
       \affaddr{University of Washington}\\
       \email{\{szhang, hlv, mernst\}@cs.washington.edu}
}

%Generating fault-revealing tests for object-oriented programs

\maketitle


\begin{abstract}
To keep a Graphical User Interface (GUI) responsive and active, a GUI
application often has a main \textit{UI thread} (or \textit{event dispatching thread})
and spawns separate threads to handle lengthy operations in the background,
such as expensive computation, I/O tasks, and network requests.
Many GUI frameworks require all GUI objects to be accessed exclusively by the
UI thread. If a GUI object is accessed through a non-UI thread,
an \textit{invalid thread access} error occurs and the whole
application may abort. 

This paper presents a general technique to find such \textit{invalid thread access}
errors in multithreaded GUI applications. We formulate finding invalid
thread access errors as a call graph reachability problem with
thread spawning as the sources and GUI object accessing as the sinks. 
Standard call graph construction algorithms fail to build a good
call graph for some modern GUI applications, because of heavy use of reflection.
Thus, our technique builds reflection-aware call graphs.


We implemented our technique and instantiated it for four popular Java GUI
frameworks: SWT, the Eclipse plugin development framework, Swing, and
Android. In an evaluation on \subnum programs comprising \totaloc LOC, our technique
found 5 previously-known errors and 7 new ones.

\end{abstract}



\section{Introduction}

Graphical User Interfaces (GUIs) are one of the most important parts of
software. Most software developed in recent years has a GUI, and the only
way for the end-user to interact with the software application is through
the GUI. Hence, end-user satisfaction and retention of a software application
 are largely determined by the usability and correctness of
its GUIs.

To make the GUIs more responsive and better utilize the increasingly available
computation power in the multi-core era, GUI applications often spawn separate
threads to handle time-consuming operations in the background, such as expensive
computations, I/O tasks, and network requests. Using threads for those lengthy tasks
makes a GUI active by allowing the timing of operations not delay unrelated parts of a 
GUI application.



\subsection{The Single-thread Rule}

Nowadays, many popular GUI frameworks such as Swing~\cite{swing}, SWT~\cite{swt}, Android~\cite{android},
Qt~\cite{qt}, MacOS Cocoa~\cite{macos}, and X Windows~\cite{xwindow} 
adapt the \textit{single-thread rule}:

\vspace{-2mm}

\begin{quote}
All GUI objects, including visual components and data models, must be
 accessed exclusively from the \textit{event dispatching thread}.
\end{quote}

\vspace{-2mm}

The \textit{event dispatching thread}, or called \textit{UI thread}, is a global special
thread initialized by the GUI framework, where all event-handling code
is executed. All code that interacts with GUI objects must also
execute on this thread.  There are several advantages to adapt the single-thread rule:

\begin{itemize}

\item GUI developers do not have to have an in-depth
understanding of concurrency programming. A GUI framework in which all components
must fully support multithreaded access can be difficult to extend, particularly
for developers who are not expert at threads programming.

\item Events are dispatched in a predictable order. Mouse and keyboard events, timer events, 
paint requests, and other user-defined events are dispatched from the same event queue.
In contrast, in a GUI framework without adapting the single-thread rule where components
support multithreaded access, any component changes can be interleaved with event
processing at the whim of the thread scheduler. This makes program behavior comprehension hard,
 and comprehensive testing difficult or impossible. 

\item Single-thread rule incurs less overhead.
A GUI framework that attempts to carefully lock critical sections can spend a substantial
amount of time and space managing locks. Whenever the framework calls a method that might
be implemented in client code (for example, any public or protected method in a public class),
the framework must save its state and release all locks so that the client code can grab locks
if necessary. When GUI objects return from the method, the framework must re-grab their locks and
restore states. Therefore, all applications using the framework will bear the cost of this, even though most
applications do not require concurrent access to the GUI.

\end{itemize}

%Single-threaded GUI frameworks are not unique to Java; ,
%and many others are also single-threaded.

\subsection{The Invalid Thread Access Error}

Despite the convenience that the single-thread rule brings to build a
GUI framework, it pushes the thread safety burden back onto the GUI application developers,
who must make sure all GUI objects are accessed by the UI thread.

In a multithreaded GUI application, the single-thread rule can be
easily violated since a spawned non-UI thread often needs to update
the GUI to display results after its computation task is finished.
For example, a server program can get requests from other programs
that might be running on different machines. These requests can come at any time,
and they result in one of the server's methods being invoked in some thread.
After handling the request, the GUI-update code must be executed on the UI-thread.
If not, an \textit{invalid thread access} will error occur. In practice, such
invalid thread access error is \textit{frequent}, \textit{pervasive}, \textit{severe}.
and \textit{difficult to avoid}.

Take the popular Standard Widget Toolkit (SWT)~\cite{swt} GUI framework as an example. 
In a recent survey~\footnote{\url{http://www.slideshare.net/lakshmip/top-3-swt-exceptions-3951224}}, the invalid thread access error
has been listed as the top 3 bugs in developing a SWT application. Searching
``SWTException:Invalid thread access'' in Google returns over 11,800 entries,
consisting of numerous bug reports, forum posts, and mailing list threads on
this problem. Eclipse~\cite{eclipse}, the de-facto standard IDE for Java development,
is built on top of SWT. As eclipse became open-sourced in 2001, the invalid thread access error has
been consistently reported during the past 11 years across different products, different components, and different releases.
Searching the same ``SWTException:Invalid thread access'' keyword in eclipse's bug repository
and discussion forum returns over 2732 bug reports and 351 distinct threads, respectively. 
We further manually studied all 156 \textit{confirmed} distinct bug reports, and 
found this error has been reported in at least 20 distinct eclipse projects
and 40 distinct eclipse components. Even after over 10-year's active development,
a recent release of eclipse still contains this error (bug id: 333533, reported in 01/2011).
 Furthermore, many reported invalid thread access errors are non-trivial. It is unintuitive
to fix them. Developers usually
need non-local reasoning to find the specific UI interactions that can trigger the bug; and
some reported bugs (e.g., bug id: 51757) even took developers 2 years to fix and verify the
patch. In addition, such invalid thread access error is UI-relevant and also user-perceivable;
it cannot be recovered by eclipse itself and often terminates the whole program.
In many circumstances (as described in the bug reports),
what users can only do is to restart the application to resort to an inconsistent status.




\begin{figure}[t]
\begin{CodeOut}
\begin{alltt}

     /* In class: org.mozilla.gecko.gfx.LayerView */
68.  public LayerView(Context context, LayerController controller) \{
69.     super(context);
        ....
73.     mRenderer = new LayerRenderer(this);
74.     setRenderer(mRenderer);
        ...
     \}

     /* In Android's lib class: android.opengl.GLSurfaceView */
272. public void setRenderer(Renderer renderer) \{
        ...
282.    mGLThread = new GLThread(renderer);
283.    mGLThread.start();   
     \}

     /* In class: org.mozilla.gecko.gfx.LayerRenderer */
220. public void onSurfaceChanged(GL10 gl, int width, int height) \{
221.    gl.glViewport(0, 0, width, height);
222.    mView.setViewportSize(new IntSize(width, height));
     \}

\end{alltt}
\end{CodeOut}
\vspace*{-2.0ex} \Caption{{\label{fig:androiderror} A real
bug reported in 11/17/2011 in Mozilla firefox release 10.0.a.rc3
for the Android platform (bugzilla ID: 703256). The \CodeIn{setRender}
method called inside \CodeIn{LayerView}'s constructor
spawns a new thread that violates the single-thread rule to inappropriately
access the GUI objects in line 222. Our tool
can find this bug and generate a warning as shown in
Figure~\ref{fig:report}.
}} %\vspace{-5mm}
\end{figure}

The invalid thread access error is not unique to the SWT framework. Other GUI frameworks
such as Android~\cite{android} also suffer from similar problems. For example, 
Figure~\ref{fig:androiderror} shows a recently reported bug on the Mozilla Firefox Android version.
In Figure~\ref{fig:androiderror}, \CodeIn{LayViewer}'s constructor calls the
\CodeIn{setRender} method (line 74), which spawns a new thread (line 283).
This newly created (non-UI) thread calls back the \CodeIn{onSurfaceChanged} method that
accesses the GUI objects at line 222, thus results in an invalid thread access error.
Compared to other reported errors, this bug is even more difficult to note, since the \CodeIn{setRender} method is in Android's
library whose source code is often not available to developers.

%It may work during development, but like most concurrent bugs, you'll start to see weird exceptions come up that seem completely unrelated, and occur non-deterministly - usually spotted AFTER you've shipped by real users. Not good.

%Also, you've got no confidence that your app will continue to work on future CPUs with more and more cores - which are more prone to encountering weird threading issues due to them being truely concurrent rather than just simulated by the OS.


\subsection{Finding Invalid Thread Access Errors}

The invalid thread access errors have significantly affect the
usability of GUIs, but have not been thoroughly studied in
the context of multithreaded GUI applications . To ensure GUIs
to behave correctly, effective
techniques must be devised to alleviate this problem.

%Effective techniques must be developed to
%find such errors.

Software testing, the most widely used method to ensure program correctness,
is often insufficient to detect such errors. That is because the space of possible interactions
with a GUI is enormous, in that each sequence of GUI events can result in
a difficult state, and each GUI event may need to be evaluated in all of
these states. It is almost impossible for testing to cover every possible
states in practice. In fact, for most reported bugs in eclipse, only very specific
UI interaction sequences can trigger the error. A large software system like eclipse
has a well-designed test suite that achieves fairly high coverage; but many
 bug-triggering UI sequences are still not covered.

Program refactoring~\cite{Mens:2004}, another popular way to improve software quality,
is also inadequate to eliminate this error.  A standard way to fix such
invalid thread access error is to replace direct GUI object accessing operations
with asynchronized message passing, to ensure GUI object are accessed in
the UI thread. For example, fixing the bug in Figure~\ref{fig:androiderror}
needs to wrap line 222 inside a \CodeIn{post} message passing method\footnote{The
\CodeIn{post} method is in class \CodeIn{android.widget.View}, providing
a standard utility to send messages to the UI-thread}, as follows:


\begin{CodeOut}
\begin{alltt}
     /* In class: org.mozilla.gecko.gfx.LayerRenderer */
220. public void onSurfaceChanged(GL10 gl, int width, int height) \{
221.     gl.glViewport(0, 0, width, height);
         \textbf{mView.post}(new Runnable() \{
             public void run() \{
222.             mView.setViewportSize(new IntSize(width, height));
             \}
         \});
     \}
\end{alltt}
\end{CodeOut}

One may wonder that a straightforward program transformation to wrap
every possible UI-accessing operation with asynchronized message passing
may solve this problem. However, such an approach is problematic,
because using asynchronized message passing has no timing
guarantee. It is entirely possible that, in some cases, a GUI object has already been
disposed before the message sent to arrives. Thus, programmers can not use
it everywhere; instead, they must choose carefully to put it in appropriate places
with extreme caution.

%Here are a few reasons, first, for the sake of efficiency, UI must use multi-thread to perform updates and processing user requests; second, there is no language level enforcement to prevent this error; and third, programmers often forget corner cases like an event listener that may init a non-UI thread to update a UI component.
%Furthermore, asyncExec has no timing guarantee, and programmers can not use it everywhere

In this paper, we choose to use static analysis to find such potential errors.
Static analysis has several advantages while comparing to dynamic approaches like testing for this problem.
First, a static analysis can explore every path of the program without executing the code, and
thus there is no need to construct non-trivial tests and harnesses.
Second, a static analysis can soundly find potentially errors: if it
reports no warnings, the code is guaranteed to be bug-free. 


\begin{figure}[t]
\begin{CodeOut}
\begin{alltt}

   org.mozilla.gecko.gfx.LayerView.<init>(Context;LayerController)
-> android.opengl.GLSurfaceView.setRenderer(GLSurfaceView\$Renderer;)
-> java.lang.Thread.start()
-> android.opengl.GLSurfaceView\$GLThread.run()
-> android.opengl.GLSurfaceView\$GLThread.guardedRun()
-> org.mozilla.gecko.gfx.LayerRenderer.onSurfaceChanged(GL10;II)
-> org.mozilla.gecko.gfx.LayerView.setViewportSize(IntSize;)
...
-> android.view.ViewRoot.checkThread()
\end{alltt}
\end{CodeOut}
\vspace*{-2.0ex} \Caption{{\label{fig:report} A report
generated by our tool to reveal the pontential bug in
Figure~\ref{fig:androiderror}. $\rightarrow$ represents the calling relations
between methods, and \CodeIn{checkThread} is an internal utility
method in Android's library that checks whether the current thread is the
event dispatching thread before accessing a GUI object.
}} %\vspace{-5mm}
\end{figure}

We formulate finding invalid thread access as a call graph reachability
problem. Our static analysis soundly constructs a call graph and starts
to traverse every possible paths from the entry points, checking whether
there exists a path to access a GUI object from a non-UI thread. If 
a suspicious path is found,
the static analysis issues a potential warning in the form of a method
call chain from the starting point, serving as the contextual information
for developers to understand the error.

As an example, Figure~\ref{fig:report} shows a report produced
by our static analysis for the buggy code in Figure~\ref{fig:androiderror}.
This report clearly indicates how a new, non-UI thread is spawned and
accesses GUI objects. 
The generated report is a great starting point for the developers, who can
inspect the method call chain, understand how the error
is triggered, and then determine if it is a real bug.

Our static analysis is independent of the call graph construction algorithms.
However, we find existing call graph construction algorithms~\cite{} are
insufficient to build a good call graph for GUI applications in the presence
of the heavy use of reflection. To alleviate this problem,
we present a combined RTA and k-CFA call graph analysis to balance the call
graph completeness and precision. (XXXX)

\subsection{Technique Instantiation and Evaluation}

(XXX)

We instantiate our technique on four popular GUI frameworks namely SWT,
eclipse plugin, Java Swing, and Android. Instantiating the general technique
for each framework only requires few human effort. The key issue to address
is to identify UI-thread



\subsection{Contributions}

This paper makes the following contributions:

\begin{itemize}
\item \textbf{Problem.} We explicitly identify the invalid thread
access error on multithreaded GUI programs, and formulate it
as a call graph reachability problem.

\item \textbf{Technique.} We present a general technique to find
such errors. In particular, to deal with the unique features in
multithreaded GUI programs, we propose a combined RTA and k-CFA
call graph construction algorithm

\item \textbf{Implementation.} We instantiate our technique for four
popular GUI platforms: SWT, eclipse plugin, Swing, and Android. Our
tool implementation is publicly available at:
\url{http://guierrordetector.googlecode.com}

\item \textbf{Evaluation.} We performed an experiment on XXX subjects
from 4 different platforms over XXX LOC. As a result, our technique
found XXX bugs 

\end{itemize}



\section{Technique}

\subsection{Problem Formulation}

formulate it as a graph reachability

\subsection{Error Detection Algorithm}


\begin{figure}[t]
\textbf{Input}: a Java program $P$\\
\textbf{Output}: a set of potential $errors$\\
\vspace{-5mm}
\begin{algorithmic}[1]
\STATE errors $\leftarrow$ $\emptyset$
\STATE cg $\leftarrow$ $constructCallGraph$($P$)
\STATE entryNodes $\leftarrow$ $getEntryNodes$(cg)
\STATE uiAccessNodes $\leftarrow$ $getUIAccessNodes$(cg)
\STATE safeNodes $\leftarrow$ $getSafeNodes$(cg)
\FOR{each entryNode in entryNodes}
\STATE threadStartNodes $\leftarrow$ $getReachableStartsByBFS$(cg)
\FOR{each threadStart in threadStartNodes}
\STATE queue $\leftarrow$ an empty queue
\STATE queue.enqueueAll(threadStart.$getAllSuccNodes$())
\WHILE{queue.isNotEmpty()}
\STATE nextNode $\leftarrow$ queue.dequeue()
\FOR{each succNode in cg.$getSuccNodes$(nextNode)}
\IF{succNode $\in$ uiAccessNodes}
\STATE errors $\leftarrow$ errors $\cup$ $createErrorReport$(succNode)
\ELSIF{succNode $\in$ safeNodes}
\STATE continue
\ELSE
\STATE queue.enqueue(succNode)
\ENDIF 
\ENDFOR
\ENDWHILE
\ENDFOR
\ENDFOR
\RETURN $errors$
%\ENDWHILE
\vspace{-2mm}
\end{algorithmic}
\caption{Algorithm for detecting errors in multithreaded GUI programs. 
Four utility methods in lines 2 -- 4 are customized for each GUI platforms
(not each program) as described in Section~\ref{sec:platforms}.
} \label{fig:detectalgorithm}
\end{figure}


\subsubsection{Call Graph Construction}

For desktop applications which have main method, it is straigtforward to apply
the above algorithm. However, for applications on platforms like eclipse plugin
and Android. presents new challenges. Two technical challenges must be solve.
First, use reflection extensively, Second, it is possible to interact with
native code.

Android features an extended event-based library and dynamic inflation of
graphical views from declarative XML layout files.
A static analyzer for Android programs must consider such features, for
correctness and precision.

\begin{figure}[t]
%\centering
\begin{CodeOut}
\begin{alltt}

<LinearLayout>
    <Button android:id="@+id/\textbf{button\_id}" android:text="A Button" />
</LinearLayout>

1. public class MyActivity extends Activity \{
2.    final List<View> cachedViews = new LinkedList<View>();
3.    @Override
4.    public void onCreate(Bundle savedInstanceState) \{
5.        super.onCreate(savedInstanceState);
6.        setContentView(R.layout.main);
7.        Button button = (Button) findViewById(R.id.\textbf{button\_id});
8.        button.setOnClickListener(new Button.OnClickListener() \{
9.            @Override
10.           public void onClick(View v) \{
11.               button.setText("Button Clicked.");
12.               cachedViews.add(v);
13.           \}
14.       \});
15.   \}
16. \}
\end{alltt}
\end{CodeOut}
\label{fig:sampleandroid}
\caption{A Sample Android Code ..}
\end{figure}


Give an example of how reflection affect call graph precision.

The Android button example.

Using RTA is not enough, since that maintains a global

So, we present a new call graph construction algorithm

reflected classes are known, use RTA.

reflection code

native code

\begin{figure}[t]
\textbf{Input}: a Java program $P$, a list of classes $cls$\\
\textbf{Output}: a call graph $cg$\\
\vspace{-5mm}
\begin{algorithmic}[1]
\STATE callSiteMap $\leftarrow$ new Map$\langle$CallSite, List$\langle$Type$\rangle$$\rangle$
\STATE rta\_cg $\leftarrow$ $constructCallGraphByRTA$($P$)
\FOR{each $method$ in $P$}
\FOR{each  $callSite$ in $method$}
\STATE type $\leftarrow$ $staticType$(callSite)
\IF{type $\in$ $cls$}
\FOR{each subType of $subType$(type)}
\IF{$hasObjectCreated$(subType, rta\_cg)}
\STATE callSiteMap[callSite].add(subType)
\ENDIF
\ENDFOR
\ENDIF
\ENDFOR
\ENDFOR
\STATE $cg$ $\leftarrow$ $constructCallGraphByKCFA$($P$, callSiteMap)
\RETURN $cg$
%\ENDWHILE
\vspace{-2mm}
\end{algorithmic}
\caption{Algorithm for constructing a call graph using a combination of the RTA~\cite{rta} and k-CFA~\cite{kcfa} algorithms.
$cls$ is a list of class XXXX, whose invocatioon is treated using RTA; while the remaining call graph
part is constructed using k-CFA.
} \label{fig:cgalgorithm}
\end{figure}




\subsubsection{Heuristic Filters and Annotations}
(XXX)

Static analysis exhaustively check possible error path; it may report
paths that indicate the same bug. The potentially huge volume of
reported warnings impose great burden for programmers to check
its validity. 

a few useful static heuristic filtering rules.

\begin{itemize}
\item \textbf{Method call chain sumbsumption}. Remove redundant
results

\item \textbf{Remove system calls}. Method call chain involves typical
system methods like \CodeIn{toString} are unlikely to reveal a real bug.

\item \textbf{Method call chain with the same entry}. such
call chains may reveal the same bug

\item \textbf{Method call chain with the same tail}.  different ways
to reach the same bug

\item \textbf{User-defined rules}.  to overcome the limitation

\end{itemize}

The above filtering rules are not sound, it may filter real bugs but works
well in practice.

We also provide annotations for native methods which are beyond the ability
of most static analyses

\subsection{Instantiation for Different Frameworks}
\label{sec:platforms}

We instantiate our error detection techniques for 4 popular GUI frameworks,
namely SWT, Eclipse plugin environment, Swing, and Android.
For each GUI platform, we customize the \textit{call graph entry nodes},
\textit{UI accessing nodes}, and \textit{safe methods} on lines 3 -- 5 of the
algorithm in Figure~\ref{fig:detectalgorithm}.

%For the sake of efficiency, they are all using the single thread model.

%Why these platforms? popularity? account for xxx\% of the GUI. 

%customizing for: entry points, starting points, error checking, thread
%safe methods, what is the event dispatching thread?

\subsubsection{SWT}

SWT is an open source widget toolkit for Java designed to provide efficient,
portable access to the GUI facilities of the operating systems on which it is implemented.

Like normal Java applications, a SWT desktop application starts from its main method.
By default, the thread that the main method executes is the UI thread. Thus, we instantiate
our technique as follows:

\begin{itemize}

\item \textbf{Call graph entry points: } the single main method

\item \textbf{UI accessing nodes: } any methods that calls \CodeIn{org.eclipse.swt.widgets.Display.checkWidget}
or \CodeIn{org.eclipse.swt.widgets.Display.checkDevice} methods, since these two methods are the
only place where the SWT framework checks whether the current thread is the UI thread or not. Thus, any
methods that calls the checking method must be invoked in the UI thread.

\item \textbf{Safe methods: } SWT provides two utility methods (\CodeIn{Display.asyncExec}
and \CodeIn{Display.syncExec}) that pass messages to the UI thread.
 Invoking these two methods are generally considered to be safe, and would not
cause an invalid thread access error.

\end{itemize}

\subsubsection{Eclipse plugin}

Unlike a SWT desktop application, an eclipse plugin, though is developed in SWT,
has no single main method. Instead, (XXX) it extends certain extension points
exposed by the eclipse framework, and eclipse will XXX. To instantiate our
technique for the Eclipse plugin environment, we need to redefine the
call graph entry points.

\begin{itemize}

\item \textbf{Call graph entry points: } all overriden methods of the classes inside
 a plugin's \CodeIn{ui} package.  Such methods are normally called by the eclipse framework
by the UI thread. XXX This rule, though quite heuristical, works quite well in practice.

\item \textbf{UI accessing nodes} and \textbf{Safe methods} are the same as SWT desktop applications.

\end{itemize}

\subsubsection{Swing}

%The collections classes from the original JDK (\CodeIn{Vector}
%and \CodeIn{Hashtable}) and the AWT library are designed as
%thread-safe. That is, two separate threads of execution can
%access the UI element at the same time without the developer having
%to worry about the threads interfering with one another. However, the
%safety comes at a cost. Because there is a great deal of overhead
%necessary to build thread-safe artifacts, the tend to be much slower
%than nonthread-safe alternatives. 

%The design decisions for Swing changed. This does not mean the controls
%can never be accessed from multiple threads, but the developer is now
%responsible for adding code to ensure that no ill effects occur.

A Swing application often contains three kinds of threads: \textit{initial thread}
that executes initial application code, the \textit{event dispatch thread},
where all GUI manipulation code is executed, and the \textit{worker thread} where
time-consuming background tasks are executed. After Swing UIs are launched,
the initial thread exits and the event dispatch thread starts to execute all
event-handling code or spawns new worker threads. XXXX Therefore, we can not directly
use the main method as the call graph entry point, since it is not invoked
by the UI thread. We customize our technique for Swing as follows:

\begin{itemize}

\item \textbf{Call graph entry points: } all overriden callback methods of the listerner classes.
Specifically, for each class, we check whether it is a Swing listerner class. If so,
we extract all its overriden callback methods as the call graph entry points.
 %Such methods are normally called by the eclipse framework
%by the UI thread. XXX This rule, though quite heuristical, works quite well in practice.

\item \textbf{UI accessing nodes:} unlike SWT, Swing does not explicitly check
whether the current thread is the UI thread or not, when a GUI object is accessed. Except
for three documented thread-safe method (\CodeIn{repaint}, XXXX), our technique
treat all methods defined in each GUI object as UI accessing nodes.

\item \textbf{Safe methods: } similar to SWT, Swing provides two utility methods
(\CodeIn{SwingUtilities.invokeLater} and \CodeIn{SwingUtilities.invokeDelay}) to pass messages to the UI thread. These two methods are regarded as safe methods.

\end{itemize}

%A key tasks in instantiating
%our technique to the Swing framework is to identify its event dispatch thread.
%We use

\subsubsection{Android}

Android is a Java-based platform for embedded or mobile devices. 
Android features an extended event-based library and dynamic inflation of
graphical views from declarative XML layout files. Android applications are
written in Java, running in their own process within their own virtual machine.
Unlike desktop applications, they do not have a single entry point but can
rather use parts of other Android applications on-demand and can require their
services by calling corresponding event handlers, directly or through the
operating system. In particular, Android applications use \textit{activities}
as code interacting with the user through a visual interface. Event handlers
are scheduled in no particular ordering, with some notable exceptions such as the
lifecycle of activities.

An XML \textit{manifest file} registers the components of an application. Other XML
fies describe the visual layout of the activities. Activities \textit{inflate}
layout files into
visual objects (a hierarchy of views), through an \textit{inflater} provided by the An-
droid library. This means that library or user-defined views are not explicitly
created by new statements but rather inflated through \textit{reflection}. Library meth-
ods such as \CodeIn{findViewById} access the inflated views.

We instantiate the technique for Android programs as follows.

\begin{itemize}

\item \textbf{Call graph entry points: } all overriden callback methods of the event
listerner classes, and all overriden callback methods of an Activity. 

\item \textbf{UI accessing nodes: } any methods that calls the \CodeIn{ViewRoot.checkThread}
method. This method is the only place where Android checks
whether the current thread is the UI thread or not. Thus, any
methods that calls the checking method must be invoked in the UI thread.

\item \textbf{Safe methods: } Android provides two utility methods (\CodeIn{View.post}
and \CodeIn{View.XXX}) that pass messages to the UI thread.
 Invoking these two methods are generally considered to be safe, and would not
cause an invalid thread access error.

\end{itemize}


\section{Implementation}
\label{sec:implementation}

We instantiated the proposed technique for four widely-used
GUI frameworks, namely SWT, Eclipse plugin development, Swing, and Android;
and implemented a prototype tool.
Our tool is built on top of the WALA framework~\cite{walatutorial}.
It takes Java bytecode as input, 
automatically detects potential invalid thread access
errors, and scales to realistic programs.  

%We implemented the proposed technique using the WALA framework~\cite{walatutorial}.
%Our prototype tool takes Java bytecode as input, 
%automatically detects potential invalid thread access
%errors, and scales to realistic programs.  




%Static Single Assignment (SSA) transformation to get limited flow sensitivity
\subsection{Instantiation for Different Frameworks}
\label{sec:platforms}

%We instantiated our error detection technique for four popular GUI frameworks:
%SWT, Eclipse plugin, Swing, and Android.
When instantiating our error detection technique for different
frameworks, the major framework-specific parts, corresponding to
lines 3 -- 5 in Figure~\ref{fig:detectalgorithm}, are identifying
\textbf{call graph entry nodes}, \textbf{UI-accessing nodes},
and \textbf{safe UI methods} for each framework.

%For each framework, we customize the \textit{call graph entry nodes},
%\textit{UI-accessing nodes}, and \textit{safe UI methods} on lines 3 -- 5 of the
%algorithm in 

%For the sake of efficiency, they are all using the single thread model.

%Why these platforms? popularity? account for xxx\% of the GUI. 

%customizing for: entry points, starting points, error checking, thread
%safe UI methods, what is the event dispatching thread?

\newcommand{\smallstep}{\vspace{-2mm}}
\subsubsection{SWT}

 We instantiated our technique for SWT applications as follows:

\begin{itemize}
\smallstep

\item \textbf{Call graph entry nodes: } the main method. Like a normal Java program,
a SWT desktop application has a single main method as the entry point. By default,
the main method is executed in the UI thread after the GUI
is initialized.

\smallstep

\item \textbf{UI-accessing nodes: } the \CodeIn{Display.checkWidget}
and \CodeIn{Display.} \CodeIn{checkDevice} methods. SWT uses the above
two methods to explicitly check thread accesses at runtime.
%Any methods for manipulating GUIs object must call (one of) these two methods
% to determine whether the current thread is the UI thread or not.
\smallstep

\item \textbf{Safe UI methods: } \CodeIn{Display.asyncExec}
and \CodeIn{Display.syncExec}. SWT provides these two methods
to execute code (a)synchro- nizedly on the UI thread.
%We treat these two methods as safe UI methods, since invoking
%them from any thread 
%will not cause an invalid thread access error.

\end{itemize}

\subsubsection{Eclipse plugin}

We instantiated our technique for the Eclipse plugin development framework as follows:

\smallstep

%Unlike a SWT desktop application, an eclipse plugin, though is developed in SWT,
%has no single main method. Instead, (XXX) it extends certain extension points
%exposed by the eclipse framework, and eclipse will XXX. To instantiate our
%technique for the Eclipse plugin environment, we need to redefine the
%call graph entry points.

\begin{itemize}

\item \textbf{Call graph entry nodes: } all user code methods that override
 SWT GUI event handling methods. Eclipse
calls back the overridden methods to handle the
events. All SWT GUI event handling methods (i.e., the overriden
methods in the class that implements \CodeIn{org.eclipse.swt.internal.SWTEventListener}) are
always called back from the UI thread, thus are used as call graph entry nodes.

\smallstep

\item \textbf{UI-accessing nodes} and \textbf{Safe UI methods} are the same as SWT.

\end{itemize}

\subsubsection{Swing}


A Swing application has a single main method, but contains three kinds of
threads: \textit{initial thread} that executes initial application code from the main method,
the \textit{UI thread}, where all GUI manipulation code is executed,
and the \textit{worker thread} where time-consuming background tasks are executed.
After a Swing program starts, its initial thread exits and the UI thread takes charge
of the application and starts to execute all event-handling code or spawns new worker threads. 
We instantiated our technique for Swing as follows:
%XXXX Therefore, we can not directly
%use the main method as the call graph entry point, since it is not invoked
%by the UI thread. We customize our technique for Swing as follows:

\begin{itemize}

\item \textbf{Call graph entry nodes:} all overridden Swing GUI event handling
methods in user code. Those event handling methods are always
called back from the UI thread.

\smallstep

\item \textbf{UI-accessing nodes:} %as stated in Swing's documentation,
all methods defined in each Swing GUI class except for three thread-safe
methods: \CodeIn{repaint()}, \CodeIn{revalidate()}, and \CodeIn{invalidate()}.

\smallstep

\item \textbf{Safe UI methods: } 
\CodeIn{SwingUtilities.}\CodeIn{invokeLater} and \CodeIn{Swing-\\Utilities.invokeAndWait} that execute code on the UI thread.

\end{itemize}

\smallstep

\subsubsection{Android}

Android is a Java-based platform for embedded or mobile devices. 
An Android program does not have a single entry point but can
rather use parts of other Android applications on-demand and can require their
services by calling corresponding event handlers, directly or through the
operating system. In particular, Android applications use \textit{activities}
(i.e., instances of the \CodeIn{Activity} class)
to interact with users through a visual interface and handle GUI events.
We instantiated our technique for Android programs as follows:
%, with some notable exceptions such as the
%lifecycle of activities.

%An XML \textit{manifest file} registers the components of an application. Other XML
%fies describe the visual layout of the activities. Activities \textit{inflate}
%layout files into
%visual objects (a hierarchy of views), through an \textit{inflater} provided by the An-
%droid library. This means that library or user-defined views are not explicitly
%created by new statements but rather inflated through \textit{reflection}. Library meth-
%ods such as \CodeIn{findViewById} access the inflated views.


\begin{itemize}

\item \textbf{Call graph entry nodes: }in an Android application,
an \CodeIn{Activity} object is created and manipulated by the UI thread. Thus, we treat
all public methods defined in the \CodeIn{Activity} class 
and any overriding definitions in its subclasses as call graph entry nodes.
We also add all overridden Android GUI event handling methods in user
code as entry nodes, since they are also called back from the UI thread.

\smallstep

\item \textbf{UI-accessing nodes: }the \CodeIn{ViewRoot.checkThread} method.
Like SWT, Android explicitly checks whether
the current thread is a UI thread or not before accessing a GUI object via
the \CodeIn{ViewRoot.checkThread} method.

\smallstep

\item \textbf{Safe UI methods: } \CodeIn{View.post}
and \CodeIn{View.postDelay} that execute code on the UI thread. 
%Invoking them on any thread will not cause an invalid thread access error.
%Invoking these two methods are generally considered to be safe, and would not
%cause an invalid thread access error.

\end{itemize}


Instantiating our general technique to  a specific framework
requires moderate human effort. We wrote around 500 lines of Java code in total to achieve
the above four instantiations.

\subsection{Implementation Details}

%Our current implementation supports four widely-used GUI frameworks:
%SWT, Eclipse plugin development, Swing, and Android.
We implemented the reflection-aware call graph construction
 algorithm (Section~\ref{sec:cg}) using WALA's \textit{bypass logic}.
Unlike similar tools~\cite{Payet:2011:SAA:2032266.2032299}, our tool
does not require a separate
pass for program instrumentation; instead, it
parses the configuration file in an Android application,
and then intercepts the call graph construction
process on-the-fly to replace all reflection calls with object creation statements.
Since Android applications are often fully encrypted and shipped in Dalvik
bytecode as a single apk file, our tool first uses
android-apktool~\cite{apktool} to
decrypt the apk file, and then uses the 
ded translator~\cite{ded} to convert
Dalvik bytecode to Java bytecode before feeding to WALA.  The Android system
library (i.e., \CodeIn{android.jar}) uses many ``stub'' classes as
placeholders for the sake of efficiency. We manually re-compiled 
\CodeIn{android.jar} from its source code, so it contains real
class files rather than stubs.


\section{Empirical Evaluation}
\label{sec:evaluation}

%Our experiment investigate the following four research questions:

%\begin{itemize}
%\Item can our approach detect real bugs in multithreaded GUI applications?

%\Item is the reflection-aware call graph construction algorithm useful?

%\Item how effective is the proposed heuristic? is the reflection-aware call graph construction algorithm useful?
%\end{itemize}

Our experiment objective is three-fold: to demonstrate the effectiveness
of our approach in detecting real bugs in multithreaded GUI applications, to 
compare the reflection-aware call graph construction algorithms
with existing ones , and to evaluate the usefulness of the proposed 
 heuristics.  

First, we describe our subject programs (Section~\ref{sec:subjects}) and the experimental procedural (Section~\ref{sec:procedural}).
We then show that our technique detects bugs in real-world GUI applications on
four supported frameworks (Section~\ref{sec:errors}). We also compare our technique with
a straightfoward approach in Section~\ref{sec:straightforward}. We further compare the reflection-aware call graph
construction algorithms with existing algorithms, showing its usefulness (Section~\ref{sec:reflectionaware}).
 Finally, we show that the set of proposed heuristics are effective
in removing redundant warnings and eliminate false positivies (Section~\ref{sec:filters}). 



\subsection{Subject Programs}
\label{sec:subjects}

We use a set of open-source projects from Sourceforge, Google code and Eclipse plugin
market as evaluation subjects. To choose subjects for each supported GUI framework,
we first search the framework keywords (e.g., ``Java Swing'' or ``Java SWT'')
in the above three source repositories, and then select subjects based on the following
criterias. First, the subject must be a multithreaded GUI application, not
an open library. Second, the subject must undergone at least two years of active development,
and is listed in the first 3 results pages based on popularity. This permits us
to exclude many pre-mature subjects that may contain obvious errors.

The subjects used in our experiment are shown in Table~\ref{table:subjects}. Our
subjects include end-user applications, programming tools, and games. Specifically:

\begin{itemize}
\Item VirgoFTP~\cite{virgo} and FileBunker~\cite{filebunker} are two useful
SWT desktop applications.
VirgoFTP implements a simple FTP client based on Java and SWT UI library, and
FileBunker is a file backup application which uses one or more GMail accounts as its backup repository.

\Item  HudsonEclipse~\cite{hudson}, and
EclipseRunner~\cite{eclipserunner} are two popular eclipse plugins.
HudsonEclipse adds several new tools to  monitor Hudson build status from Eclipse.
EclipseRunner extends capability of running launch configurations in Eclipse IDE.

\Item  S3dropbox~\cite{s3dropbox} and SudokuSolver~\cite{sudokusolver}
are two Swing desktop applications. S3dropbox implements a client that allows users
to drag and drop files, which are then uploaded to their Amazon S3 accounts. SodokuSolver
computes Sodoku solutions efficiently using mutlithreading execution. 

\Item  MyTracks~\cite{mytracks}, Fennec~\cite{fennec},
and SGTPuzzles~\cite{sgtpuzzles} are three Android applications.
MyTracks, developed by Google, records users' GPS tracks in outdoor activities, and provides
interfaces to visualize them on Google Maps. Fennec, developed by Mozilla, is the Mozilla
Firefox web browser for mobile devices. SGTPuzzles is a small logic single-player game.

\end{itemize}


\begin{table}[t]
\begin{center}
 \fontsize{9pt}{\baselineskip}\selectfont
\hspace*{-0.2cm}
\setlength{\tabcolsep}{.75\tabcolsep}
\begin{tabular}{|p{4.4cm}||c|c|c|c|}
\hline
 Program (version) & LOC & Classes & Methods \\
\hline \hline
\multicolumn{4}{|l|}{SWT desktop applications}   \\
 \hline
 VirgoFTP (1.3.5) &  2293 &  20 &  165  \\
 \hline
 FileBunker (1.1.2)&  14237 &  150 &  1106  \\
 \hline
 \hline
\multicolumn{4}{|l|}{Eclipse plugins}   \\
 \hline
 EclipseRunner (1.1.0) &  3101 &  48 &  354\\
 \hline
 HudsonEclipse(1.0.9)&  11077 &  74 &  649 \\
 \hline
 \hline
\multicolumn{4}{|l|}{Swing desktop applications}   \\
 \hline
 S3dropbox (1.7) &  2353 &  42  &  224 \\
 \hline
 SudokuSolver (1.06)&  3555 &  10 &  62 \\
 \hline
 \hline
\multicolumn{4}{|l|}{Android mobile applications}   \\
 \hline
 SGTPuzzler (v9306.11)&  2220 &  16 &  148 \\
 \hline
 Fennec (10.0.a.rc3)&  8577 &  51 &  620 \\
 \hline
 MyTracks (1.1.13.rc4)&  20297 &  143 &  1374 \\
\hline
\multicolumn{4}{l}{}   \\
\hline
 GUI framework (version) & LOC & Classes & Methods  \\
\hline \hline
 SWT (3.6)&  129942 &  999 &  9643 \\
\hline
 Eclipse plugin development (3.6.2)&  460830 &  6630 &  37183 \\
\hline
Swing (1.6)&  167961 &  878 &  13159 \\
\hline
 Android (3.2)&  683289 &  5085 &  10584 \\
\hline
\end{tabular}

\end{center}
\vspace{-15pt}
\Caption{{\label{table:subjects} Open source programs
used in our evaluation. Column ``LOC'' is the number of non-blank, non-comment lines
of code, as counted by LOCC~\cite{locc}.  Each program is analyzed
together with its dependent GUI framework, as listed in
the bottom table.} }
\end{table}


\subsection{Procedural}
\label{sec:procedural}

We run our tool on each subject with three call graph construction
algorithms: RTA~\cite{rta}, 0-CFA, and 1-CFA~\cite{kcfa}.  When running
each call graph construction algorithm on Android applications, we
use two configurations: with and without the reflection-aware transformation
(Section~\ref{sec:cg}).  We did not use other more expensive algorithms like $k$-CFA ($k >$ 1),
because they can not scale to large programs.

We inspect the source code of each subject to determine whether any
native methods or project-specific GUI interaction patterns are implemented.
We found SGTPuzzler is the only subject that uses native methods to interact with
underlying operating systems. Thus, we manually added XXX \CodeIn{@CalledByNative}
annotations for it. Also two subjects XXXX and XXX employ a self-defined
way to interact with the GUI framework. For example, the \CodeIn{safeInvokeXX}
method in the XXX subject first checks whether the current thread is the UI-thread, and
then determines whether to direct access a GUI object or wrap the code with
message passing. We add XX heuristic filtering rules for those methods, to indicate
that they are safe to be invoked from any thread. In this experiment, none of the paper
authors was familar with the subjects; but we found it is quite easy
to add extra those annotations and heuristic filters and finished these
manual parts in less than 20 minutes.

After provided with necessary annotations and heuristic filters,
our tool works in a fully-automatic, push-bottom manner. For each output
warning, we manually determine its validity by either searching the
bug repository to check whether the same bug has been reported before,
or submitting a new bug report, or writing a test driver to reproduce
the error.



\begin{table*}[ht]
\begin{center}
 \fontsize{9pt}{\baselineskip}\selectfont
\hspace*{-0.2cm}
\setlength{\tabcolsep}{.45\tabcolsep}
\begin{tabular}{|l||c|c|c||c|c|c||c|c|c||c|c|}
\hline
 Subject&  \multicolumn{9}{|c||}{Our Technique} & \multicolumn{2}{|c|}{The Approach in}  \\
\cline{2-10}
 Program  &  \multicolumn{3}{|c|}{RTA / reflection aware}& \multicolumn{3}{|c|}{0-CFA / reflection aware} & \multicolumn{3}{|c||}{1-CFA / reflection aware} & \multicolumn{2}{|c|}{Section~\ref{sec:straightforward}}  \\
\cline{2-12}
 & CG Size & \#Warning & \#Bug & CG Size & \#Warning & \#Bug & CG Size & \#Warning & \#Bug & \#Warning & \#Bug\\
\hline \hline
\multicolumn{12}{|l|}{SWT desktop applications}   \\
 \hline
 VirgoFTP&  12401 &  2 &  2 & 10858 & 2 & 2 & 43598 & 2 & 2& 149 & 2 \\
 \hline
 FileBunker &  18951 &  1 &  0 & 15743 & 0 & 0 & 76088 & 2 & 1& 693 & 1 \\
 \hline
 \hline
\multicolumn{12}{|l|}{Eclipse plugins}   \\
 \hline
 EclipseRunner&  11248&  6 &  1 & 7201 & 6 & 1 & 26911 & 6 & 1& 202 & 1 \\
 %\hline
 %FileSync&  12132 &  xx &  xxx &8235 & xx & xx& 32565 & 1 & 1 & 331 & 1 \\
 \hline
 HudsonEclipse& 18473 &  2 &  1 & 15814 & 2 & 1& 56645 & 3 & 1 & 182 & 1 \\
 \hline
 \hline
\multicolumn{12}{|l|}{Swing desktop applications}   \\
 \hline
 S3dropbox & 37751 &  0 &  0 & 30609 & 0 & 0 & 115324 & 1 & 1 & 210 & 1 \\
 \hline
  SudokuSolver&  27730&  3 &  2 & 20907 & 3 & 2 & 39299 & 2 & 2 & 356 & 2 \\
 \hline
 \hline
\multicolumn{12}{|l|}{Android mobile applications}   \\
 \hline
 SGTPuzzler & 13631 / 13865&  12 / 16 &  0 / 0 & 9546 / 9682& 4 / 4& 1 / 1 & 35198 / 35756 & 1 / 1  & 1 / 1& 104 & 1 \\
 \hline
 Fennec & 14058 / 14387 &  1 / 1 &  0 / 0 & 8263 / 8898 & 1 / 1 & 0 / 0& 29125/ 31759 & 3 / 3 & 1 / 1& 433 & 1 \\
 \hline
 MyTracks & 24036 / 24036 &  161 / 220 & 0 / 0 & 10803 / 13645 & 119 / 119 & 0 / 0 & 39235 / 110977 & 61 / 1 & 0 / 1 & 1192 & 1 \\
\hline
\end{tabular}
\end{center}
\vspace{-15pt}
\Caption{{\label{table:results}Experimental results in finding errors
in multithreaded GUI programs. Column ``CG Size'' shows the
call graph size in terms of node number. Column ``\#Warning''
shows the number of warnings issued by our tool. Column ``\#Bug'' shows
the actual bugs found. Columns ``RTA / reflection aware '', ``0-CFA / reflection aware'',
and ``1-CFA / reflection aware'' show the results of using different
call graph construction algorithms with (before ``/'') or without (after ``/'')
taming the reflection on Android applications. In each cell of the table
for Android applications, a slash ``/'' separates the result of
using normal call graph construction algorithm and the reflection-aware
call graph construction algorithm. As a comparison, the
results of a straightforward approach (Section~\ref{sec:straightforward}) are show at the far right.} }
\end{table*}

\subsection{Results}
\label{sec:results}

%Table~\ref{table:results} show the errors in
%our subject programs, and Table~\ref{table:filters} show
%the effect of applying each heuristic filter.

\subsubsection{Errors in Multithreaded GUI Applications}
\label{sec:errors}

Table~\ref{table:results} tabulates the results of running our
tool with three different call graph construction algorithms
on the subject programs. From these results, we can see our
tool found real errors from each subject on all the 
supported GUI frameworks. The results vary slightly according
to the specific call graph algorithm used. In general, using
the 1-CFA algorithm finds most errors (XXX in total) with least
warnings (XXX in total). The tradeoffs of using other
algorithms will be explained in Section~\ref{sec:reflectionaware}.

We further examined the detailed results for each subject.
Among xxx warnings issued by using the 1-CFA algorithm, xxx
are false positives, xxx are redundant:
they have the same causes as the rest xxx warnings. (XXX why there
are FP and redundant warnings?)

why finds a bug in SGTPuzzle, because the \CodeIn{SmallKeyBoard} has been
explicitly created in this main class \CodeIn{SGTPuzzles} \CodeIn{setKeyboardVisibility},
thus, even without reflection-aware

Why RTA, 0-CFA finds few, because they do not distinguish different
calling context.


\begin{figure}[t]
\begin{CodeOut}
\begin{alltt}
/* In class: com.tomczarniecki.s3.gui.DeleteBucketAction */
59.private void deleteBucket() \{
60.    executor.execute(new Runnable() \{
61.        public void run() \{
62.            try \{
63.                controller.deleteCurrentBucket();
64.            \} catch (Exception e) \{
65.                logger.info("Delete failed", e);
66.                deleteError(); 
67.            \}
68.       \}
69.    \});
70.\}

77.private void deleteError() \{
78.    String text = "Cannot delete folder .....";
79.    display.showErrorMessage("Delete failed", 
        String.format(text, controller.getSelectedBucketName())); 
80.\}
\end{alltt}
\end{CodeOut}
\vspace*{-2.0ex} \Caption{{\label{fig:swingerror} A
potential invalid thread access error reported by our tool
in the S3dropbox Swing application. The error occurs when
the method \CodeIn{deleteCurrentBucket} invoked at line 63 throws
an exception, which leads method \CodeIn{deleteError} to
access a Swing GUI object \CodeIn{display} at line 79 from
a non-UI thread. This error is previously-unknown, and has been confirmed by the S3dropbox developers.
}} %\vspace{-5mm}
\end{figure}

Figures~\ref{fig:swingerror} amd~\ref{fig:pluginerror} show
two real errors our tool has found. Figure~\ref{fig:swingerror}
shows a potential invalid thread access error in the S3dropbox
program. This error happens when the \CodeIn{deterCurrentBucket}
method at line 63 throws an exception. This error can be hard to
detect by testing, since the test must  execute that specific exception-handling
path. We reported this error to the S3dropbox developers. Tom Czarniecki,
a key developer of S3dropbox confirmed this potential single-thread
violation. He mentioned that the S3dropbox project has already
used certain design patterns and practice to avoid such violations (e.g.,
actions for UI interaction are encapsulated into a \CodeIn{Worker} interface),
but they still overlooked  the reported code snippet. 

Another reason of missing this violation is because some GUI frameworks like Swing
do not provide any run-time checks for invalid thread access. Most
of the time, programmers can get away with an apparently ``well behaved'' Swing
GUI that actually breaks the single-thread rule. However, As clearly
indicated in the official documentation, Swing is a single-threaded GUI
toolkit, and is not thread-safe. All Swing code must be executed in the UI-thread and
invoking Swing code from multiple threads risks thread interference
or memory-consistency errors, particularly in the multi-core era.



\begin{figure}[t]
\begin{CodeOut}
\begin{alltt} 
/* com.eclipserunner.views.impl.RunnerView */
179.private void initializeResourceChangeListener() \{
180.  ResourcesPlugin.getWorkspace().addResourceChangeListener(
        new IResourceChangeListener() \{
181.      public void resourceChanged(IResourceChangeEvent event) \{
182.        refresh();
183.      \}
184.  \}, IResourceChangeEvent.POST\_CHANGE);
185.\}

414.public void refresh() \{
415.  getViewer().refresh();
416.\}
\end{alltt}
\end{CodeOut}
\vspace*{-2.0ex} \Caption{{\label{fig:pluginerror} A
potential invalid thread access error reported by our tool
for the EclipseRunner plugin. In Eclipse, the
callback method \CodeIn{resourceChanged} at line 181
is invoked by non-UI threads when a \CodeIn{ResourceChangeEvent}
happens. However, the \CodeIn{refresh} method directly accesses
GUI objects (refresh the view at line 415) without any
protection and thus triggers the error.
This error has been reported by other users 13 months after the
buggy code was checked in, and fixed by developers.
}} %\vspace{-5mm}
\end{figure}

Figure~\ref{fig:pluginerror} shows an error found in the EclipseRunner
plugin. This error is event-related. It happens when a 
\CodeIn{ResourceChangeEvent} happen, which then invokes the \CodeIn{refresh}
method at line 415 to update the user interface. In EclipseRunner
, the \CodeIn{refresh} method has been
called by XXX different methods in non-UI threads. Thus, our
tool issues XXX separate warnings for each caller method to indicate
different ways to trigger this error, despite that all these xxx warnings
acutally have the same error cause.

As reflected in our experiments, we found GUI developers have already
used design patterns, runtime checks, and testing to avoid violating
the single-thread rule. However, due to the huge space of
possible UI interactions, potential
invalid thread access errors still exist in many corner cases and
can be hard to find. Thus, a static analysis as presented in this paper
would be useful.

\vspace{1mm}

\noindent \textbf{\textit{Summary.}}Our technique finds real errors
in multithreaded GUI applications on four different frameworks with
an acceptable false positive rate.

\subsubsection{Comparing to a Straightforward Approach}
\label{sec:straightforward}

As we mentioned in Section~\ref{sec:finding}, a standard way to avoid
invalid thread access errors is to wrap every GUI access code
with message passing methods (i.e., the safe methods). Thus, a
straightforward way to detect potential invalid error
is to check every method to determine whether all GUI-accessing code
is probably wrapping by safe methods. If not, this approach
issues a warning for every violation. 
%If not, this approach issues warnings.

We implemented this straightforward approach and experimentally compared it
with the technique presented in Section~\ref{sec:technique}.
The far right column in Table~\ref{table:results} shows the results.
The results show that this straightforward approach can identified every
error as our technique does, but leads to a tremendous number of false
positives and redundant warnings. The primary reason is that this simple approach
does not globally reason about calling relationship between
threads, UI-accessing methods, and safe methods, thus it often incorrectly
classifies GUI accessing code which will never be executed
in a non-UI thread as errenous. Furthermore, our technique
outputs a method call chain in the error report, which can help
developers understand how an invalid thread access error is triggered.



\vspace{1mm}

\noindent \textbf{\textit{Summary.}} Our technique provides
richer contextually information for a potential error, and 
is significantly more precise than the straightforward approach.

\subsubsection{Comparing Call Graph Construction Algorithms}
\label{sec:reflectionaware}

We next compare three call graph construction algorithms (RTA, 0-CFA, and 1-CFA)
used in our experiments. As shown in Table~\ref{table:results},  XXX found
more errors with less warnings than the XXX and XXXX algorithms. This
is because RTA and 0-CFA do not consider the context XXX, and it may
mix the same methods called by different callers. Although the call graphs
built by 1-CFA are significantly larger, XXX, the algorithm still runs
in a practical amouont of time. For example,
our tool finished building the call graph for the largest subject (xxx)
within xxx minutes.

The results in Table~\ref{table:results} also show that using a reflection-aware
call graph construction algorithm is critical in finding errors in Android
applications. None of the errors from the Android applications
can be found by using normal call graph construction algorithms, due to
their limitation in handling platform-specific reflections. Without
replacing the reflection calls with explicit object creation instructions,
a call graph construction algorithm will
conclude that no GUI object has been created, and thus miss many UI-related
edges in the resulting graph.

\vspace{1mm}

\noindent \textbf{\textit{Summary.}} Using the 1-CFA call graph construction
algorithm is more effective in finding errors, and our reflection-aware call
graph construction algorithm is critical to find errors on
Android applications.

\subsubsection{Evaluating Heuristic Filters}
\label{sec:filters}

We measure the effectiveness of the proposed heuristic filters of
Section~\ref{sec:heuristic} via the number of removed false positives and
redundant warnings.

Table~\ref{table:filters} tabulates the results. Although
our static analysis outputs thousands of warnings for each
subject, the proposed heuristic filters effectively reduces
the number. On average, they
 reduce the false positive numbers and redundant warnings by up to a factor of
XXX. Specifically, filters $XX_i$ and $XX_j$  identified xx\% of
the warnings as likely false positives, and filters $YY_i$
and $YY_j$ identified YY\% of the warnings as redundant warnings.
We further manually check each filtered warning, and found no
real errors have been removed.

\vspace{1mm}

\noindent \textbf{\textit{Summary.}} Although a static analysis may
issue false postivies or redundant reports, a set of well-designed
heuristics can still remove most of them.
%Heuristic filter, effective.

\begin{table}[t]
\begin{center}
 \fontsize{9pt}{\baselineskip}\selectfont
\setlength{\tabcolsep}{.81\tabcolsep}
\hspace*{-0.2cm}
\begin{tabular}{|l||c|c|c|c|c|c|}
\hline
 Subject & Before & \#W & \#W & \#W & \#W& \#W \\
 Program & Filtering & $F_1$ & $F_2$ & $F_3$&$F_4$ & $F_5$\\
\hline \hline
\multicolumn{7}{|l|}{SWT desktop applications}   \\
 \hline
 VirgoFTP &  21 &  21 &  3 & 3 &  2 & 2\\
 \hline
 FileBunker &  4494 &  3210 &  3210 &  642 &  2 & 2\\
 \hline
 \hline
\multicolumn{7}{|l|}{Eclipse plugins}   \\
 %\hline
 %EclipseRunner &  xxxx &  xx &  xxx & xx &  xxx & xx\\
 \hline
 EclipseRunner&  1621 &  1487 &  1487 & 1487 &  5 & 1\\
 \hline
 HundsonEclipse&  3192 &  2417 &  1696 & 9 &  3 & 3\\
 \hline
 \hline
\multicolumn{7}{|l|}{Swing desktop applications}   \\
 \hline
 S3dropbox&  45528 &  31978 &  30975 & 9 &  1 & 1\\
 \hline
 SudokuSolver &  58 &  58 &  58 & 2 &  2 & 2\\
 \hline
 \hline
\multicolumn{7}{|l|}{Android mobile applications}   \\
 \hline
 SGTPuzzler&  2 &  1 &  1 & 1 &  1 & 1\\
 \hline
 Fennec &  122 &  84 & 84 & 84 &  9 & 3\\
 \hline
 MyTracks &  1176 &  441 &  441 & 441 &  63 & 1 \\
\hline
\end{tabular}
\end{center}
\vspace{-15pt}
\Caption{{\label{table:filters}Number of warnings after applying a set of
heuristic filters (Section~\ref{sec:heuristic}). Column ``Before Filtering''
shows the output warning number using the reflection-aware 1-CFA algorithm.
Column ``\#W $F_i$'' represents the number of remaining warnings after applying
filters 1 -- $i$, and the number in last column is taken
from Table~\ref{table:results}.
Results of using other algorithms show similar patterns, which are omitted for brevity.} }
\end{table}


\subsection{Comparing Graph Search Strategies}
\label{sec:search}

We next empirically compare different search strategies.

Exhaustive search, and dfs, we found exhaustive search strategy
can not terminte on any of the evaluated subject program, since the
path is expoentially large to the graph node. For a realistic, program
We implemented two alternative

We check the path explored by the exhaustive algorithm

Use DFS, avoid going too deep, under the setting of OneCFA. The time cost
is almost identical

The costs a tremedous more time than BFS, generates more warnings.


Hundson: 3, 1

EclipseRunner: 6, 1

S3dropbox: ? , ?  too many warnings time: 31678 (45528 chains), running out of memory

SudokuSolver:  2, 1 too many warnings: 1700, 2090

VirgoFTP: 2, 2

FileBunker:  2, 0  //very long method calls

Fennec: 3, 1

SGTPuzzler: 1, 1

MyTracks: 1, 1

\subsection{Discussions}

\textbf{what about other graph traverse algorithms}

We next discuss several important issues in our evaluation.

excluding the call graph construction time.

\subsubsection{Performance and Scalability}

Our tool runs in a practical amount of time. Our evaluations
were conducted on a 2.67GHz Intel Core PC with 4GB
physical memory (1GB is allocated for JVM), running Fedora 16.
For the largest subject, xxx, our tool finished the whole analysis
within 286 seconds using the most expensive call graph construction
algorithm (the reflection-aware 1-CFA analysis). Analyzing
other subjects using different algorithms took less time.
The performance has not been tuned, and we believe
that it could be improved further.

Our experiments also demonstrate that our approach has good scalability.
Together with all dependent library code, our tool has analyzed
XXX LOC in total  XXX LOC in the largest program.
%Our tool has analyzed xxx LOC in total, and together with all dependent libraries,
%the largest subject our approach can analyze and detect errors has
%over XXX LOC, xx classes, and xxx methods.



\subsubsection{Threats to Validity}

There are two threats to validity in our evaluation. 
The major threat is the degree to which the subject programs
used in our experiment are representative of true practice.
In our evaluation, the studied subjects were selected with a bias towards popularity from
three open-source repositories. Another threat is that we only employed three
well-known call graph construction algorithms (i.e., RTA, 0-CFA, and 1-CFA) 
in our implementation. It is still unknown whether using other 
call graph construction algorithms such as XTA~\cite{xta} may achieve better results.

\subsubsection{Limitations}

Our technique is limited in two aspects. First, it only considers
non-UI threads that are spawned by the UI-thread after the GUI
is initialized, and ignores other possible non-UI threads
that are created during the pre-initialization GUI work. One way
to remedy this limitation is to design an analysis to identify
those threads created before a GUI is launched.
Second, our tool cannot compute call relationships
for inter-process communication between components. It also
requires users to manually add annotations to characterize
call relationships that involve native methods. This limitation
may lead to false negatives. Investigating the false positive
rate is ongoing work.

%Future work could evaluate our technique on more subjects, and
%investigate the false negative rate.

%our technique requires developers to manually
%add annotations to provide call relationships
%Second, 

%Third, for subject program
%that uses native methods, 

\subsubsection{Experimental Conclusions}

Invalid thread access errors significantly affect the usability of
a GUI application, but can be subtle to find. In practice, programmers
have designed practices and patterns to avoid such errors in the first
place, but can easily overlook corner cases like exception-handling
methods.

The technique presented in this paper offers a promising solution.
Compared with the straightforward approach to check every
GUI accessing code isolatedly, our technique produces significantly
less warnings. The reflection-aware call graph construction algorithm
is critical in detecting errors for Android programs, and our
proposed heuristic filtering rules are useful in reducing programmers'
burden to inspect the output result.

%Findings: programmers have noted this problem, but often overlook special cases like error handle, subtype

% Reporting on violations is fine but what is more important are the practices and patterns that need to be adopted to make it easier to avoid the issue in the first place. That is why there is a Worker interface in this project



\tinystep
\section{Related Work}

Work related to this paper falls into three main categories; (1)
analyzing and testing GUI applications; (2)
bug finding techniques for multithreaded programs; and (3)
call graph construction algorithms.

\tinystep
\subsection{Analyzing and Testing GUI Applications}
%\vspace{1mm}

%\noindent \textit{\textbf{Analysis and Testing for GUI Applications}}
Automated GUI testing is a challenging task~\cite{Bertolino:2007:STR:1253532.1254712,
Harrold:2000:TR:336512.336532}.
Various techniques automate GUI testing including
%model creation
%(for model-based testing)~\cite{androidtesting}, %, Xie:2006:MTC:1172962.1172990},
test generation~\cite{YuanMemonICSE2007},
test oracle creation~\cite{MemonFSE2000}, test execution~\cite{YuanCohenMemonTSE2011},
and test script repairing~\cite{Huang:2010:RGT:1828417.1828465, Daniel:2011:AGR:2002931.2002937}.
For example,
Guitar~\cite{YuanCohenMemonTSE2011, YuanMemonICSE2007}
is a GUI testing framework for Java and Microsoft Windows applications. 
Yuan and Memon~\cite{YuanMemonICSE2007} generated event-sequence-based test cases for GUI
programs using a structural event generation graph. 
However, testing is often insufficient to detect many potential
errors in a GUI application due to the huge space of possible UI interactions.
%The large number of possible interactions
%requires a large number of test inputs that require substantial human effort.
In contrast, a static analysis can explore all paths to find potential errors missed by testing.
Compared to software testing, a static analysis tool such as ours may
report false positives and redundant warnings due to its conservative nature.
In our experiments, simple error filters reduced the number of warnings to
an acceptable level.


Michail and Xie~\cite{michail05:helping} proposed a tool-based approach to help users avoid bugs
in GUI applications. Their approach monitors a user's action in the background,
and gives a warning as well as the opportunity to abort the action, when
a user attempts an action that has led to problems in the past. 
Their work aims to prevent an existing bug from happening again.
By contrast, our work aims to find unknown errors.


Recently, Payet and Spoto~\cite{Payet:2011:SAA:2032266.2032299} presented a static
analysis framework for Android programs based on  abstract
interpretation. Their framework focuses on the Android platform, and
 consists of 7 existing static analyses such as
nullness analysis, class analysis, and termination analysis.  However,
their framework does not support detecting invalid-thread-access
errors, and uses a quite different abstraction than ours.
To the best of our knowledge, we are the first to address the invalid
thread access error detection problem for multithreaded GUI applications, and
our core technique has been tailored for four GUI frameworks.


\tinystep
\subsection{Finding Bugs in Multithreaded Programs}

%\vspace{1mm}

%\noindent \textit{\textbf{Finding Bugs in Multithreaded Programs}}
A rich body of techniques have been developed to detect bugs in multithreaded programs~\cite{Huang:2011:PPC:2001420.2001438, Weeratunge:2010, Huang:2011:EST}.
Static analysis tools such as Chord~\cite{Naik:2006}
explore multithreaded programs. Runtime
analysis tools such as Eraser~\cite{Savage:1997}  dynamically detect concurrency bugs using lockset
algorithms or some criteria-based automata. 
More recently, Goldilocks~\cite{Elmas:2007} uses a
hybrid model that combines the happens-before and lock-based
approaches to identify data races based on an execution.
However, finding invalid-thread-access errors is quite different than detecting data races.
A data race occurs when two concurrent threads access
a shared variable and when at least one access is a write and the threads
use no explicit mechanism to prevent the accesses from being simultaneous. In contrast,
an invalid thread access error occurs when a non-UI thread accesses (reads or writes) a GUI object.
Unlike detecting data races, finding an invalid-thread-access error does not require monitoring every shared-memory
reference to verify that consistent locking behavior is observed among different threads.
A technique only needs to track whether a non-UI thread can accesses a GUI object or not,
and is much cheaper. Leveraging data race detection to 
improve our technique is future work.


An alternative way to find bugs in multithreaded programs is using model checking~\cite{Nori:2010:ESO, Inverardi:2000, Siegel:2008}.
By exhaustively exploring the thread scheduling space, a model checker can
report counterexamples as bug reports.
Unfortunately, due to the exponential size of the search space,
it is hard for model checking approaches to scale to a realistic multithreaded GUI application
 without compromising the error detection capability. 
We are not aware of any software model checking approach that scales to programs
as large as those used in our experiments (including the library code).
The technique presented in this paper is specifically designed to find invalid thread
errors instead of being a general property checking tool. 
It chooses the call graph as a coarse-grained program representation with a
set of error filters, to achieve good scalability with reasonable accuracy.

%For this reason,
%our technique achieves good scalability with reasonable precision.

\tinystep

\subsection{Call Graph Construction Algorithms}
%\vspace{1mm}

%\noindent \textit{\textbf{Call Graph Construction Algorithms}}
We briefly mention some call graph construction
algorithms for Java. Grove et al.~\cite{kcfa} described a unified
framework for expressing call graph construction algorithms, and
studied different instantiations of the framework.
Tip and Palsberg~\cite{xta} quantitatively compared
several low-cost call graph construction algorithms for Java.
Sundaresan et al.~\cite{Sundaresan:2000} went beyond the
RTA~\cite{rta} approach and used type propagation
to build a more precise call graph.  However,
those algorithms do not build a complete call graph in the presence of reflection.
As reflected in our experiments, using standard call graph algorithms
misses errors in some Android applications.


Livshits et al.~\cite{Livshits:2005} presented a static analysis
to reason about reflective calls. The analysis
attempts to infer additional information stored in string constants to resolve
reflective calls statically. Their approach focuses on standard Java
reflection calls (e.g., \CodeIn{Class.forName}) instead of
framework-specific ones (e.g., \CodeIn{View.find-\\ViewById}).
The standard call graph algorithms implemented in WALA, which we used
in our experiments, actually handles standard Java
reflection calls as~\cite{Livshits:2005} does, but still fails to build
a sufficiently complete call graph for
Android applications. TamiFlex~\cite{Bodden:2011}, a pure dynamic
approach, records all reflectively-created class instances
by intercepting JVM system calls, and re-inserts those recorded 
class into a program. However, TamiFlex requires a set of representative
program executions and is only sound with respect to the given executions.
Perhaps the closest work to our call graph
construction algorithm is Payet and Spoto's Julia
system~\cite{Payet:2011:SAA:2032266.2032299}. The Julia system
needs to first instrument Android's library code that performs the XML inflation,
and then replaces the \CodeIn{findViewById} call with the corresponding
object creation expressions. In addition, the Julia system does not handle
native methods when building call graphs. In contrast, our technique provides
annotation support for native methods, and our tool does not need a separate
pass of off-line instrumentation. The reflection-aware call graph is created
online by intercepting the standard call graph construction process.





%There are few algorithms have been studied to construct call
%A recent paper by Hirzel, Diwan, and Hind addresses the issues of dynamic
%class loading, native methods, and reflection in order to deal with the full complexity
%of Java in the implementation of a common pointer analysis [5]. Their
%approach involves converting the pointer analysis [6] into an online algorithm:
%they add constraints between analysis nodes as they are discovered at runtime.
%Newly generated constraints cause re-computation and the results are propagated
%to analysis clients such as a method inliner and a garbage collector at
%runtime. Their approach leverages the class hierarchy analysis (CHA) to update
%the call graph. Our technique uses a more precise pointer analysis-based
%approach to call graph construction.


 


\smallstep
\section{Conclusion and Future Work}

This paper presented a general technique to find invalid
thread access errors in multithreaded GUI applications. 
Our technique statically explores paths in a call graph to check
whether a non-UI thread can access a GUI object.
It uses a reflection-aware call graph construction algorithm
to build a good call graph, and employs a set of error
filters to filter likely false positives and redundant warnings.
%Since graphical user interfaces are often inadequately tested due
%to the enormously large space of UI interactions and
%extreme resource constraints during their development, we
%believe that providing a static analysis can allow developers to find
%more potential errors %before the application is released to customers (XXX),
% to improve software reliability.
We demonstrated the usefulness of our technique
by an evaluation on \subnum programs built on 4 popular GUI
frameworks. 

The source code of our tool implementation is available at:


\noindent \url{http://guierrordetector.googlecode.com}


\vspace{1mm}


Besides general issues such as performance and ease of use, our future
work will concentrate on the following topics:

%\begin{itemize}
%\item
\textbf{Integration with dynamic and symbolic analyses.} The technique 
presented in this paper is a call graph-based, pure static analysis. 
It suffers from false positives for many GUI applications,
due to the conservative nature of static analysis.
A possible way to reduce the number of false
positives is to integrate with
dynamic analyses~\cite{Jiang:2008:PPS:1453101.1453110}
or symbolic analyses~\cite{xie05:symstra, Pasareanu:2011, halfond09issta, BMF97}
by using more accurate information to guide call graph exploration.

%\item
\textbf{Unit testing multithreaded GUI programs.} Besides
a static analysis,  software testing is another
way to improve software quality.  Although many
testing techniques have been developed recently, few of them can be applied
to unit test multithreaded GUI programs to find potential errors \textit{earlier}. We
plan to investigate how to apply recent advance in automated
testing~\cite{Staats:2011:PTO:1985793.1985847, Jagannath:2011:IMU:2025113.2025145, Muccini_Bertolino_Inverardi_2004, Ricca:2001:ATW:381473.381476, Harman:2007}
to the context of multithreaded GUI applications.


%\item
\textbf{Fixing potential GUI errors.} After an error is revealed, fixing
it and verifying the patch is often time-consuming. Fixing concurrency
bugs has become especially critical in the multicore era.
Recently, work has been done
on automatically repairing test scripts for GUI applications~\cite{Daniel:2011:AGR:2002931.2002937, Huang:2010:RGT:1828417.1828465}. However, none of them focuses on repairing
the GUI program to patch a revealed error. Thus, we are interested in 
developing automated error fixing techniques for
multithreaded GUI applications.

%\end{itemize}


The source code of our tool implementation is available at:

\vspace{1mm}

\noindent \url{http://guierrordetector.googlecode.com}


\vspace{2mm}

%\noindent \textbf{Acknowledgement.} We thank XXX, and the %anonymous
%reviewers for their feedback. This work was supported in
%part by ABB Corporation and NSF under grant CCF-0963757.

%\vspace{-2mm}

\bibliographystyle{abbrv}
\footnotesize{
\bibliography{issta2012}
}
\end{document}
