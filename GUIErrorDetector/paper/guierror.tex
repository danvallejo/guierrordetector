

%\documentclass{acm_proc_article-sp}
\documentclass{sig-alternate}
\usepackage{multirow}
\usepackage{fancyheadings}
\usepackage{algorithmic}
\usepackage{amssymb}
\usepackage{xspace}
\usepackage{pslatex}
\usepackage{microtype}
\usepackage{subfigure}
%\usepackage{todonotes}
\usepackage{times}
\usepackage{graphicx}
\usepackage{epsf}
\usepackage{verbatim}
%\usepackage{psfig}
\usepackage{cite}
\usepackage{url}
\usepackage{color}
\usepackage{alltt}

\newcommand{\StateProjection}{static analysis}
\newcommand{\CurAspectJSubjectCount}{12}

\newcommand{\Add}{\CodeIn{add}}
\newcommand{\AVTree}{\CodeIn{AVTree}}
\newcommand{\Assignment}[3]{$\langle$ \Object{#1}, \Object{#2}, \Object{#3} $\rangle$}
\newcommand{\BinaryTreeRemove}{\CodeIn{BinaryTree\_remove}}
\newcommand{\BinaryTree}{\CodeIn{BinaryTree}}
\newcommand{\Caption}{\caption}
\newcommand{\Char}[1]{`#1'}
\newcommand{\CheckRep}{\CodeIn{checkRep}}
\newcommand{\ClassC}{\CodeIn{C}}
\newcommand{\CodeIn}[1]{{\small\texttt{#1}}}
\newcommand{\CodeOutSize}{\scriptsize}
\newcommand{\Comment}[1]{}
\newcommand{\Ensures}{\CodeIn{ensures}}
\newcommand{\ExtractMax}{\CodeIn{extractMax}}
\newcommand{\FAL}{field-ordering}
\newcommand{\FALs}{field-orderings}
\newcommand{\Fact}{observation}
\newcommand{\Get}{\CodeIn{get}}
\newcommand{\HashSet}{\CodeIn{HashSet}}
\newcommand{\HeapArray}{\CodeIn{HeapArray}}
\newcommand{\Intro}[1]{\emph{#1}}
\newcommand{\Invariant}{\CodeIn{invariant}}
\newcommand{\JUC}{\CodeIn{java.\-util.\-Collections}}
\newcommand{\JUS}{\CodeIn{java.\-util.\-Set}}
\newcommand{\JUTM}{\CodeIn{java.\-util.\-TreeMap}}
\newcommand{\JUTS}{\CodeIn{java.\-util.\-TreeSet}}
\newcommand{\JUV}{\CodeIn{java.\-util.\-Vector}}
\newcommand{\JMLPlusJUnit}{JML+JUnit}
\newcommand{\Korat}{Korat}
\newcommand{\Left}{\CodeIn{left}}
\newcommand{\Lookup}{\CodeIn{lookup}}
\newcommand{\MethM}{\CodeIn{m}}
\newcommand{\Node}[1]{\CodeIn{N}$_#1$}
\newcommand{\Null}{\CodeIn{null}}
\newcommand{\Object}[1]{\CodeIn{o}\ensuremath{_#1}}
\newcommand{\PostM}{\MethM$_{post}$}
\newcommand{\PreM}{\MethM$_{pre}$}
\newcommand{\Put}{\CodeIn{put}}
\newcommand{\Remove}{\CodeIn{remove}}
\newcommand{\RepOk}{\CodeIn{repOk}}
\newcommand{\Requires}{\CodeIn{requires}}
\newcommand{\Reverse}{\CodeIn{reverse}}
\newcommand{\Right}{\CodeIn{right}}
\newcommand{\Root}{\CodeIn{root}}
\newcommand{\Set}{\CodeIn{set}}
\newcommand{\State}[1]{2^{#1}}
\newcommand{\TestEra}{TestEra}
\newcommand{\TreeMap}{\CodeIn{TreeMap}}

\newenvironment{CodeOut}{\begin{scriptsize}}{\end{scriptsize}}
\newenvironment{SmallOut}{\begin{small}}{\end{small}}

\newcommand{\pairwiseEquals}{PairwiseEquals}
\newcommand{\monitorEquals}{MonitorEquals}
%\newcommand{\monitorWField}{WholeStateW}
\newcommand{\traverseField}{WholeState}
\newcommand{\monitorSMSeq}{ModifyingSeq}
\newcommand{\monitorSeq}{WholeSeq}

\newcommand{\IntStack}{\CodeIn{IntStack}}
\newcommand{\UBStack}{\CodeIn{UBStack}}
\newcommand{\BSet}{\CodeIn{BSet}}
\newcommand{\BBag}{\CodeIn{BBag}}
\newcommand{\ShoppingCart}{\CodeIn{ShoppingCart}}
\newcommand{\BankAccount}{\CodeIn{BankAccount}}
\newcommand{\BinarySearchTree}{\CodeIn{BinarySearchTree}}
\newcommand{\LinkedList}{\CodeIn{LinkedList}}

\newcommand{\Book}{\CodeIn{Book}}
\newcommand{\Library}{\CodeIn{Library}}

\newcommand{\Jtest}{Jtest}
\newcommand{\JCrasher}{JCrasher}
\newcommand{\Daikon}{Daikon}
\newcommand{\JUnit}{JUnit}

\newcommand{\trie}{trie}

\newcommand{\Perl}{Perl}


\newcommand{\SubjectCount}{11}
\newcommand{\DSSubjectCount}{two}

\newcommand{\Equals}{\CodeIn{equals}}
\newcommand{\Pairwise}{PairwiseEquals}
\newcommand{\Subgraph}{MonitorEquals}
\newcommand{\Concrete}{WholeState}
\newcommand{\ModSeq}{ModifyingSeq}
\newcommand{\Seq}{WholeSeq}
\newcommand{\Aeq}{equality}

\newcommand{\Meaning}[1]{\ensuremath{[\![}#1\ensuremath{]\!]}}
\newcommand{\Pair}[2]{\ensuremath{\langle #1, #2 \rangle}}
\newcommand{\Triple}[3]{\ensuremath{\langle #1, #2, #3 \rangle}}
\newcommand{\SetSuch}[2]{\ensuremath{\{ #1 | #2 \}}}
%\Comment{
%\newtheorem{definition}{Definition}
%\newtheorem{theorem}[definition]{Theorem}
%}
\newcommand{\Equiv}[2]{\ensuremath{#1 \EquivSTRel{} #2}}
\newcommand{\EquivME}{\Equiv}
\newcommand{\EquivST}{\Equiv}
\newcommand{\EquivSTRel}{\ensuremath{\cong}}
\newcommand{\Redundant}[2]{\ensuremath{#1 \lhd #2}}
\newcommand{\VB}{\ensuremath{\mid}}
\newcommand{\MES}{method-entry state}

\newcommand{\Small}[1]{{\small{#1}}}

\newcommand{\CenterCell}[1]{\multicolumn{1}{c|}{#1}}
\newcommand{\Fix}[1]{{\large\textbf{FIX}}#1{\large\textbf{FIX}}}

\newcommand{\CodeInS}[1]{{\scriptsize\texttt{#1}}}
\newcommand{\CodeInFN}[1]{{\footnotesize\texttt{#1}}}
\newcommand{\CodeOutFN}{\footnotesize}

\newcommand{\SmallSpace}{\vspace*{-1.4ex}}
\newcommand{\Item}{\SmallSpace\item}
\newenvironment{Itemize}{\begin{itemize}}{\end{itemize}\SmallSpace}
\newenvironment{Enumerate}{\begin{enumerate}}{\end{enumerate}\SmallSpace}

\newtheorem{definition}{Definition}
\newtheorem{theorem}[definition]{Theorem}

%\newcommand{\Item}{\vspace*{-0.5ex}\item\vspace*{-0.5ex}}
%\newenvironment{Itemize}{\begin{itemize}\vspace*{-1ex}}{\end{itemize}\vspace*{-1ex}}
%\newenvironment{Enumerate}{\begin{enumerate}\vspace*{-1ex}}{\end{enumerate}\vspace*{-1ex}}
\newenvironment{Definition}{\begin{definition}\vspace*{-1.5ex}}{\end{definition}\vspace*{-1.5ex}}

% Local Variables:
% mode:latex
% tex-main-file:"ase04.tex"
% End:


% % Add line between figure and text
 \makeatletter
 \def\topfigrule{\kern3\p@ \hrule \kern -3.4\p@} % the \hrule is .4pt high
 \def\botfigrule{\kern-3\p@ \hrule \kern 2.6\p@} % the \hrule is .4pt high
 \def\dblfigrule{\kern3\p@ \hrule \kern -3.4\p@} % the \hrule is .4pt high
 \makeatother
 % If there is a line, you can get away with reducing the separation between
 % figures and text.  Don't do this without the line, though.
 \addtolength{\textfloatsep}{-.5\textfloatsep}
 \addtolength{\dbltextfloatsep}{-.5\dbltextfloatsep}
 \addtolength{\floatsep}{-.5\floatsep}
 \addtolength{\dblfloatsep}{-.5\dblfloatsep}


\begin{document}

\title{Finding Errors in Multithreaded GUI Applications}
%\subtitle{[Extended Abstract]
%\titlenote{This work is sponsored by}


\author{
\alignauthor Sai Zhang \quad Hao Lu \quad Michael D. Ernst\\
       \affaddr{Department of Computer Science \& Engineering}\\
       %\affaddr{1932 Wallamaloo Lane}\\
       \affaddr{University of Washington}\\
       \email{\{szhang, hlv, mernst\}@cs.washington.edu}
}

%Generating fault-revealing tests for object-oriented programs

\maketitle


\begin{abstract}
To keep a Graphical User Interface (GUI) responsive and active, a GUI
application often has a main \textit{UI thread} (or \textit{event dispatching thread})
and spawns separate threads to handle lengthy operations in the background,
such as expensive computation, I/O tasks, and network requests.
Many GUI frameworks require all GUI objects to be accessed exclusively by the
UI thread. If a GUI object is accessed through a non-UI thread,
an \textit{invalid thread access} error occurs and the whole
application may abort. 

This paper presents a general technique to find such \textit{invalid thread access}
errors in multithreaded GUI applications. We formulate finding invalid
thread access errors as a call graph reachability problem with
thread spawning as the sources and GUI object accessing as the sinks. 
Standard call graph construction algorithms fail to build a good
call graph for some modern GUI applications, because of heavy use of reflection.
Thus, we use a method to build reflection-aware call graphs.


We implemented our technique and instantiated it for four popular Java GUI
platforms: SWT, the Eclipse plugin development framework, Swing, and
Android. In an evaluation on \subnum programs comprising \totaloc LOC, our technique
found both previously-known and unknown errors.

\end{abstract}



\section{Introduction}
\label{sec:introduction}

End-user satisfaction depends in part on the responsiveness and
robustness of a software application's GUI.

To make the GUI more responsive, 
GUI applications often spawn separate threads to handle time-consuming
operations in the background, such as expensive computation, I/O tasks,
and network requests. This permits the GUI to respond to new events
even before the lengthy task completes. However, the use of multiple threads
enables new types of errors that may compromise robustness.
We now discuss a standard programming rule for multithreaded GUIs, the consequences
of violating it, and a technique for statically detecting such violations.



%\subsection{The Single-thread Rule for GUI Object Access}
%ugly hack
\subsection{\hspace{-1.1ex}\mbox{The}~\mbox{Single-thread}~\mbox{Rule}~\mbox{for}~\mbox{GUI}~\mbox{Object}~\mbox{Access}}

Many popular GUI frameworks such as Swing~\cite{swing}, SWT~\cite{swt}, Eclipse plugin~\cite{eclipse},
Android~\cite{android}, Qt~\cite{qt}, and MacOS Cocoa~\cite{macos}
adopt the \textit{single-GUI-thread rule}:

\vspace{-2mm}

\begin{quote}
All GUI objects, including visual components and data models, must be
 accessed exclusively from the \textit{event dispatching thread}.
\end{quote}

\vspace{-2mm}

The \textit{event dispatching thread}, also called the \textit{UI thread}, is a single
special thread initialized by the GUI framework, where all event-handling code
is executed. All code that interacts with GUI objects must also
execute on that thread.  There are several advantages to the single-GUI-thread rule:

\begin{itemize}

\item Concurrency errors, such as races and deadlocks, never occur on GUI objects. 
GUI developers need not become experts at concurrent programming. Programming the
framework itself is also simpler and less error-prone.

\tinystep

\item The single-GUI-thread rule incurs less overhead.
Otherwise, whenever the framework calls a method that might
be implemented in client code (e.g., any non-final public or protected method in a public class),
the framework must save its state and release all locks so that the client code can grab locks
if necessary. When GUI objects return from the method, the framework must re-grab their locks and
restore states.  Even applications that do not require concurrent access to the GUI
must bear this cost.

\tinystep

\item GUI events are dispatched in a predictable order from a single event queue.
If the thread scheduler could arbitrarily interleave component changes, then event processing,
program comprehension, and testing would be more difficult.
\end{itemize}

%Single-threaded GUI frameworks are not unique to Java; ,
%and many others are also single-GUI-threaded.

\subsection{The Invalid Thread Access Error}

The single-GUI-thread rule requires GUI application developers to
 ensure that all GUI objects are accessed only by the UI thread.
If not, an \textit{invalid thread access error} will occur. This may
terminate the application, but it is considered preferable
to nondeterministic concurrency errors.

The single-GUI-thread rule can be easily violated; since a spawned non-UI thread often needs to update
the GUI  after its  task is finished.
In practice, invalid thread access errors are \textit{frequent}, \textit{severe}, and \textit{hard to debug}.
%and \textit{difficult to avoid}.

Take the popular Standard Widget Toolkit (SWT) GUI framework as an example. 
Invalid thread access is one of the top 3 bugs in developing a SWT application, and is
the source of many concurrency bugs~\cite{top3bugs}.
A Google search for ``SWTException:Invalid thread access''  returns over 11,800 entries,
consisting of numerous bug reports, forum posts, and mailing list threads on
this problem. Eclipse, the IDE for Java development, is built on top of SWT\@.
Searching for ``SWTException:Invalid thread access'' in Eclipse's bug repository
and discussion forum returns over 2732 bug reports and 351 distinct threads, respectively. 
We manually studied all 156 distinct \textit{confirmed} bug reports, and 
found this error has been confirmed in at least 20 distinct Eclipse projects
and 40 distinct Eclipse components. Even after over 10 years of active development,
a recent release of Eclipse still contains this error (bug id: 333533, reported in January 2011).
 In addition, the invalid thread access error
is severe. It is user-perceivable, it cannot be recovered by the program itself,
and it often terminates the whole application. In many circumstances as described in the bug reports,
users must restart the application to recover from the error.
Furthermore, many reported  errors are non-trivial to diagnose and fix.
Developers usually need non-local reasoning to find the specific
UI interactions that can trigger the bug;  it took developers 2 years
to fix Eclipse bug 51757 and verify the patch.

\begin{figure}[t]
\hspace{6mm}\small{In class: org.mozilla.gecko.gfx.LayerView}
\vspace{-2mm}
\begin{CodeOut}
\begin{alltt}
68.  public LayerView(Context context, LayerController controller) \ttlcb
69.     super(context);
        ....
73.     mRenderer = new LayerRenderer(this);
74.     setRenderer(mRenderer);
        ...
     \ttrcb
\end{alltt}
\end{CodeOut}
\hspace{6mm} \small{In Android library class: android.opengl.GLSurfaceView}
\vspace{-2mm}
\begin{CodeOut}
\begin{alltt}
272. public void setRenderer(Renderer renderer) \ttlcb
        ...
282.    mGLThread = new GLThread(renderer);
283.    mGLThread.start();   
     \ttrcb
\end{alltt}
\end{CodeOut}
 \hspace{6mm}\small{In class: org.mozilla.gecko.gfx.LayerRenderer}
\vspace{-2mm}
\begin{CodeOut}
\begin{alltt}
220. public void onSurfaceChanged(GL10 gl, int width, int height) \ttlcb
221.    gl.glViewport(0, 0, width, height);
222.    mView.setViewportSize(new IntSize(width, height));
     \ttrcb
\end{alltt}
\end{CodeOut}
\tinystep
\vspace*{-3.0ex} \Caption{{\label{fig:androiderror} 
Bug 703256 reported on 11/17/2011 in Fennec (Mozilla Firefox for Android)
release 10.0.a.rc3. On line 74, \CodeIn{LayerView}'s constructor
calls method \CodeIn{setRenderer} which 
spawns a new thread on line 283. This newly created, non-UI thread calls back method
\CodeIn{onSurfaceChanged} that accesses GUI objects on line 222,
causing an invalid thread access error. Our tool
finds this error and generates a report as shown in
Figure~\ref{fig:report}.
}} %\vspace{-5mm}
\end{figure}

The invalid thread access error is not unique to the SWT framework. Other recent GUI frameworks
like Android suffer from similar problems. For example, 
Figure~\ref{fig:androiderror} shows a recently-reported bug in the Android version of Mozilla Firefox.
This bug is particularly difficult to diagnose, 
since the code that spawns a new thread inside the \CodeIn{setRenderer} method is
in an Android library.

%It may work during development, but like most concurrent bugs, you'll start to see weird exceptions come up that seem completely unrelated, and occur non-- usually spotted AFTER you've shipped by real users. Not good.

%Also, you've got no confidence that your app will continue to work on future CPUs with more and more cores - which are more prone to encountering weird threading issues due to them being truely concurrent rather than just simulated by the OS.


\subsection{Finding Invalid Thread Access Errors}
\label{sec:finding}

To ensure that GUIs behave correctly, developers must prevent or detect invalid thread
access errors. Current techniques are not effective. %must be developed.% to alleviate this problem.

%Software testing, the most widely used method to ensure program correctness,
%is insufficient to detect such errors. 
It is infeasible for testing to cover the enormous space of possible interactions
with a GUI. Each sequence of GUI events can result in
a different state, and each GUI event may need to be evaluated in all of
these states. A  software system like Eclipse
often has a test suite that achieves fairly high statement coverage,
but many paths executed by bug-triggering UI sequences are still not covered.

Stylized coding patterns are also inadequate. One possible rule is to always
access GUI objects via asynchronous message passing, to ensure a GUI object is accessed in
the UI thread. For example, a developer could have prevented the bug in Figure~\ref{fig:androiderror}
by wrapping line 222 inside a \CodeIn{post} message-passing method\footnote{The
\CodeIn{post} method in class \CodeIn{android.widget.View} is a
a standard way to send asynchronized messages to the UI-thread.}, as follows:

{\vspace{2mm}
\hspace{3mm}\small{In class: org.mozilla.gecko.gfx.LayerRenderer}
\vspace{-2mm}
\begin{CodeOut}
\begin{alltt}
220. public void onSurfaceChanged(GL10 gl, int width, int height) \ttlcb
221.     gl.glViewport(0, 0, width, height);
         \textbf{mView.post(new Runnable() \ttlcb}
             \textbf{public void run() \ttlcb}
222.             mView.setViewportSize(new IntSize(width, height));
             \textbf{\ttrcb}
         \textbf{\ttrcb);}
     \ttrcb
\end{alltt}
\end{CodeOut}}

Such an approach is desirable for accesses from non-UI threads, but it is unnecessary.
In our evaluation on real-world programs, we found a simple analysis that
requires GUI operations to be in a message issues
an unacceptable number of warnings. %false positives and redundant warnings.
Furthermore, this approach is dangerous for accesses from the UI thread. Asynchronized message passing offers
no timing guarantee, so a GUI object may have already been
disposed before the message sent to it arrives, causing other bugs. 


%program transformation to wrap
%every possible UI-accessing operation with asynchronized message passing
%may avoid the invalid thread error. However, such an approach is problematic,
%because using asynchronized message passing has no timing
%guarantee. It is entirely possible that, in some cases, a GUI object has already been
%disposed before the message sent to arrives, causing other bugs. Thus, programmers can not use
%it everywhere; instead, they must choose carefully to put it in appropriate places
%with extreme caution. This makes invalid thread errors even harder to avoid,
%and demands a proactive technique to detect them.

%Here are a few reasons, first, for the sake of efficiency, UI must use multi-thread to perform updates and processing user requests; second, there is no language level enforcement to prevent this error; and third, programmers often forget corner cases like an event listener that may init a non-UI thread to update a UI component.
%Furthermore, asyncExec has no timing guarantee, and programmers can not use it everywhere

\smallskip

\textbf{Our approach: static analysis.}
This paper uses static analysis to find potential GUI errors.
Static analysis has two advantages compared to dynamic approaches such as
 testing. First, a static analysis can explore paths of the program without
executing the code, and without the need for a test suite.
Second, a static analysis can verify the code: if a sound static analysis
reports no warnings, the code is guaranteed to be bug-free. 


\begin{figure}[t]
\begin{CodeOut}
\begin{alltt}
   org.mozilla.gecko.gfx.LayerView.<init>(Context;LayerController)
-> android.opengl.GLSurfaceView.setRenderer(GLSurfaceView\$Renderer;)
-> java.lang.Thread.start()
-> android.opengl.GLSurfaceView\$GLThread.run()
-> android.opengl.GLSurfaceView\$GLThread.guardedRun()
-> org.mozilla.gecko.gfx.LayerRenderer.onSurfaceChanged(GL10;II)
-> org.mozilla.gecko.gfx.LayerView.setViewportSize(IntSize;)
   ... 
-> android.view.ViewRoot.recomputeViewAttributes(View;)
-> android.view.ViewRoot.checkThread()
\end{alltt}
\end{CodeOut}
%\hspace{5mm}\CodeIn{\scriptsize{org.mozilla.gecko.gfx.LayerView.<init>(Context;LayerController)}}\\
%$\rightarrow$\CodeIn{ \scriptsize{android.opengl.GLSurfaceView.setRenderer(GLSurfaceView\$Renderer;)}}\\
%$\rightarrow$\CodeIn{ \scriptsize{java.lang.Thread.start()}}\\
%$\rightarrow$\CodeIn{ \scriptsize{android.opengl.GLSurfaceView\$GLThread.run()}}\\
%$\rightarrow$\CodeIn{ \scriptsize{android.opengl.GLSurfaceView\$GLThread.guardedRun()}}\\
%$\rightarrow$\CodeIn{ \scriptsize{org.mozilla.gecko.gfx.LayerRenderer.onSurfaceChanged(GL10;II)}}\\
%$\rightarrow$\CodeIn{ \scriptsize{org.mozilla.gecko.gfx.LayerView.setViewportSize(IntSize;)}}\\
%\hspace{5mm}\CodeIn{...}\\
%$\rightarrow$\CodeIn{ \scriptsize{android.view.ViewRoot.checkThread()}}
\tinystep
\vspace*{-3.0ex} \Caption{{\label{fig:report} Our tool reports
a method call chain that reveals the potential error in
Figure~\ref{fig:androiderror}. $\rightarrow$ represents
the call relationship
between methods, and \CodeIn{checkThread} is an Android library
method  that checks whether the current thread is the
event dispatching thread before accessing a GUI object.
8 more methods in the call chain, shown as
``\CodeIn{\small{...}}" above, are omitted for brevity.
}} %\vspace{-5mm}
\end{figure}

Our static analysis formulates finding invalid thread access as a call graph reachability
problem. Given a call graph, our technique traverses
paths from its entry nodes, checking whether
any path accesses a GUI object from a non-UI thread. If 
a suspicious path is found, the static analysis warns of a potential error.
The warning is in the form of a method
call chain from the starting point.
As an example, Figure~\ref{fig:report} shows a report produced
by our static analysis for the buggy code in Figure~\ref{fig:androiderror}.
This report clearly indicates how a new, non-UI thread is spawned and
accesses GUI objects. The generated report allows  developers to
inspect the method call chain, understand how the error
could be triggered, and fix it if it is a real bug.

Our static analysis is independent of the call graph construction algorithm.
%Given a sound call graph, our analysis is sound in that it does
%not miss true positives.
However, modern GUI applications tend to use reflection, and
in the presence of reflection, existing call graph construction algorithms such as RTA~\cite{rta}
and k-CFA~\cite{kcfa} fail to build a complete call graph.
To alleviate this problem, we present an algorithm to build a reflection-aware call graph, and
also compare its usefulness with existing call graph construction algorithms in
our experiments.

Static analysis may report false positives due to its
conservative nature, or multiple warnings that actually correspond
to the same error. To address such limitations, we devise a set
 of error filters to remove likely false positives and redundant warnings.
The filters introduce potential unsoundness to our algorithm, but in practice
they work well and make our technique usable.

\subsection{Technique Instantiation and Evaluation}


We implemented an invalid-thread-error detection tool that supports
four popular GUI frameworks: SWT, Eclipse plugin environment,
Swing, and Android. Swing and SWT are the
two dominant GUI frameworks for desktop Java applications.
Eclipse is the de-facto standard IDE for Java. Android is the \#1 platform for
mobile applications with market share 56\% as of Sep 2011.
Although our technique is applicable to any GUI
framework with the single-GUI-thread rule, each framework
has its own definition of \textit{UI thread} and \textit{program 
entry points}. Thus, our implementation is parameterized with respect to
those framework-specific parts (see Section~\ref{sec:platforms}).


Our tool works in an automatic manner,
and scales to realistic programs.
We evaluated our implementation on \subnum programs comprising \totaloc LOC.
The experimental results demonstrate
that: our technique is effective (it found \bugs real-world errors
 and produced only \falses false positive warnings);
our reflection-aware call graph construction algorithm helps in
finding errors in Android applications;
and our proposed filters significantly reduce the number of warnings.
%false positives and redundant warnings.


\subsection{Contributions}

This paper makes the following contributions:

\tinystep

\begin{itemize}
\item \textbf{Problem.} To the best of our knowledge, we are the first to address
the invalid thread access error detection problem for multithreaded GUI applications.

\tinystep

\item \textbf{Technique.} We formulate finding
the invalid thread access error as a call graph reachability problem,
and present a general error detection technique.
In addition, we use a reflection-aware
call graph construction algorithm (Section~\ref{sec:technique}).

\tinystep

\item \textbf{Implementation.} We implemented our technique and
instantiated it for four
popular GUI platforms: SWT, Eclipse plugin, Swing, and Android (Section~\ref{sec:implementation}). Our
tool implementation is publicly available at
\url{http://guierrordetector.googlecode.com}.

\tinystep

\item \textbf{Evaluation.} We applied our tool to \subnum programs 
from 4 different frameworks, comprising \totaloc LOC. The results
show the usefulness of the proposed technique (Section~\ref{sec:evaluation}).

\end{itemize}



\section{Technique}
\label{sec:technique}

We first give a high-level formulation of the problem, then present
the error detection algorithm. Finally, we show how to instantiate
the error algorithm for different GUI frameworks.


\subsection{Problem Formulation}

This section formulates the problem. We first define the
notations of \textit{UI thread}, \textit{non-UI threads},
\textit{UI-accessing methods},
 \textit{safe UI methods}, and \textit{invalid thread access error}, and
state two assumptions we make with regard to error detection.

\noindent {\textsc{\textbf{Definition 1 (UI Thread).}}} {The UI thread
is a special thread created by the GUI framework during
GUI initialization. After the GUI becomes visible, the UI thread
takes charge of the application to handle events from the GUI,
and spawns new threads to process lengthy operations in the background. }\vspace{1mm}

\noindent {\textsc{\textbf{Assumption 1.}}} {We assume that each multithreaded
GUI application has a single global UI thread. This is true for applications
built on top of GUI frameworks adopting the single-GUI-thread rule. The
only exception is that 
 an application may fork a new process to launch another
application with its own UI thread. In that case, we require the
launched application to be analyzed separately.}\vspace{1mm}

\noindent {\textsc{\textbf{Definition 2 (non-UI Thread).}}} {Any other
threads except for the UI thread in a multithreaded GUI application
 are called non-UI threads.}\vspace{1mm}

\noindent {\textsc{\textbf{Assumption 2.}}} { We assume that each non-UI
thread is spawned by the UI thread. Under this assumption,
we ignore all non-UI threads created by the GUI framework
before the UI thread has been initialized. That is, all post-initialization
GUI work naturally occurs in the UI thread. Once the GUI is visible, the
application is driven by events, which are always handled in the UI thread.
We believe this assumption is reasonable, since if a non-UI thread 
spawned during pre-initialization GUI work accesses a GUI object, an exception becomes
immediately apparent and the whole application may abort even before the
GUI is visible. This is highly unlikely for realistic GUI
applications.
}\vspace{1mm}

\noindent {\textsc{\textbf{Definition 3 (UI-Accessing Methods).}}} { A method
whose execution may read or write a UI object is called a UI-accessing method.}\vspace{1mm}

\noindent  {\textsc{\textbf{Definition 4 (Safe UI Methods).}}} {GUI frameworks that
adopt the single-GUI-thread rule must provide methods to permit non-UI threads
to run code in the UI thread. We call such methods \textit{safe UI methods}, since
they can be invoked safely by any thread.}\vspace{1mm}

\noindent  {\textsc{\textbf{Definition 5 (Invalid Thread Access).}}} {An invalid
thread access error occurs when the UI thread may spawn a non-UI thread, and there
exists a path from the non-UI thread's \CodeIn{start} method to any UI-accessing method
without going through any safe UI method. }\vspace{2mm}

At a high level, to detect an invalid thread access error, an analysis needs to track all
non-UI threads spawned by the UI thread, and check whether those non-UI threads
may invoke a UI-accessing method.

\subsection{Error Detection Algorithm}

The algorithm for detecting potential invalid thread access errors
is shown in Figure~\ref{fig:detectalgorithm}. Our algorithm uses a
static call graph as the program representation for a multithreaded
GUI application. A Java call graph represents calling relationships
between methods. Specifically, each node represents a method and each
edge ($f$, $g$) indicates that method $f$ may call method $g$.
A theoretically ideal call graph can be defined as the union of the
dynamic call graphs over all possible executions of the program. 
A conservative, or sound, static call graph is a superset of
the ideal call graph; it over-approximates the
dynamic call graph of every possible execution. 
In an object-oriented programming language like
Java, a method is the basic logic unit in a program. Reporting
potential errors at the method level can be sufficient in most cases.
Thus, we believe using the call graph is well suited for this problem while
compared with other program representations like control flow graph or
program dependence graph. 


Our algorithm in Figure~\ref{fig:detectalgorithm} %works as follows. It
first constructs a static call graph for the tested program (line 2),
then specifies entry nodes, UI-accessing nodes, and safe UI nodes
for it on line 3, 4, and 5, respectively. Lines 3--5 are GUI framework-specific.
As an example, for a SWT desktop application, the entry nodes include
the single main method, UI-accessing nodes include all methods that
call method \CodeIn{Display.checkWidget} or \CodeIn{Display.checkDevice},
and safe UI methods include two SWT helper methods \CodeIn{Display.asyncExec}  and
\CodeIn{Display.syncExec} for message passing. The detailed instantiation
for each supported framework is explained in Section~\ref{sec:platforms}.

The algorithm first finds all reachable \CodeIn{Thread.start()}
nodes from the entry nodes (line 6).
%For each entry node, the algorithm uses standard Breadth First Search (BFS)
%to find all reachable \CodeIn{Thread.start()} nodes (line 7).
Each \CodeIn{Thread.start()} node in a call graph indicates that a new,
non-UI thread is spawned.  After that, from each \CodeIn{Thread.start()}
node, the algorithm uses BFS to search for reachable UI-accessing
methods (lines 9 -- 26).
Specifically, it uses a queue to keep track of the nodes to be visited (line 7),
and stores all visited nodes to avoid infinite loop (line 8).
The algorithm checks every successive node to be visited (line 16):
it reports an error if that node is a UI-accessing method (lines 18, 19), or skips
it if it is a safe UI method (lines 20, 21), or simply keeps traversing otherwise (line 23).

Note that when the algorithm determines that a UI-accessing method may be invoked
by a non-UI thread, it computes a method call chain from the entry node to
the UI-accessing node as the error report.
The method \CodeIn{createErrorReport} on line 18 does that. 
Figure~\ref{fig:androiderror} shows a method call chain from the entry
node \CodeIn{LayerView.<init>} to the UI-accessing node \CodeIn{ViewRoot}.\CodeIn{recomputeViewAttributes}
(for the sake of clarification, we also show the checking method \CodeIn{checkThread} in the reported warning).




\begin{figure}[t]
\textbf{Input}: a Java program $\mathit{P}$\\
\textbf{Output}: a set of potential invalid thread access errors\\
%\textbf{Auxiliary Methods:}\\
%getSuccNodes($\mathit{g}$, $\mathit{n}$): return node $\mathit{n}$'s successive
%nodes in the graph $\mathit{g}$\\
\vspace{-4mm}
\begin{algorithmic}[1]
\STATE $\mathit{errors}$ $\leftarrow$ $\emptyset$ 
\STATE $\mathit{cg}$ $\leftarrow$ constructCallGraph($\mathit{P}$)
\STATE $\mathit{entryNodes}$ $\leftarrow$ getEntryNodes($\mathit{cg}$)
\STATE $\mathit{uiAccessingNodes}$ $\leftarrow$ getUIAccessNodes($\mathit{cg}$)
\STATE $\mathit{safeUINodes}$ $\leftarrow$ getSafeNodes($\mathit{cg}$)\\
\COMMENT{get all reachable \CodeIn{thread.start()} nodes from entry nodes}
\STATE $\mathit{threadStarts}$ $\leftarrow$ $\bigcup_{entry \in entryNodes}$ getReachableStarts($\mathit{entry}$)
\COMMENT{keep the nodes to be visited next in a fringe}
\STATE $\mathit{fringe}$ $\leftarrow$ an empty queue\\
\COMMENT{keep all visited nodes to avoid infinite loop}
\STATE $\mathit{visited}$ $\leftarrow$ $\emptyset$
\STATE $\mathit{fringe}$.enqueueAll($\mathit{threadStarts}$)
\WHILE{$\mathit{fringe}$.isNotEmpty()}
\STATE $\mathit{node}$ $\leftarrow$ $\mathit{fringe}$.dequeue()
\IF{$\mathit{node}$ $\in$ $\mathit{visited}$}
\STATE continue
\ENDIF
\STATE $\mathit{visited}$ $\leftarrow$ $\mathit{visited}$ $\cup$ $\mathit{node}$
\FOR{each $\mathit{succNode}$ in getSuccNodes($\mathit{cg}$, $\mathit{node}$)}
\IF{$\mathit{succNode}$ $\in$ $\mathit{uiAccessingNodes}$}
\STATE $\mathit{newError}$ $\leftarrow$ createErrorReport($\mathit{entryNode, succNode}$)
\STATE $\mathit{errors}$ $\leftarrow$ $\mathit{errors}$ $\cup$ $\mathit{newError}$
\ELSIF{$\mathit{succNode}$ $\in$ $\mathit{safeUINodes}$}
\STATE continue
\ELSE
\STATE $\mathit{fringe}$.enqueue($\mathit{succNode}$)
\ENDIF 
\ENDFOR
\ENDWHILE
\RETURN $errors$
%\ENDWHILE
\vspace{-2mm}
\end{algorithmic}
\caption{Algorithm for detecting invalid thread access errors in multithreaded GUI programs. 
Any call graph construction algorithm can be used (line 2). The algorithm
is parameterized by the three methods in lines 3--5 which are specific to each GUI framework
 as described in Section~\ref{sec:platforms}.
} \label{fig:detectalgorithm}
\end{figure}

The algorithm in Figure~\ref{fig:detectalgorithm} uses BFS to search
for potential error-revealing paths, since BFS always returns the
shortest path to the UI-accessing node, permitting
to produce smaller error reports. However,
other graph search strategies such as Depath-First Search (DFS) or
exhaustive path search can also be employed. In our experiment (Section~\ref{sec:search}),
we empirically compared three different graph search strategies, and demonstrated that
using BFS, algorithm found more errors than DFS and
exhaustive path search.

Given a sound call graph, our algorithm is sound in that it does not
miss true positives. We note, however, that computing a sound
call graph in the presence of reflection is non-trivial. 
To alleviate this problem, we next proposed a reflection-aware call graph
construction algorithm in Section~\ref{sec:cg}.

\subsubsection{Reflection-aware Call Graph Construction}
\label{sec:cg}

%In OO program, translate k-CFA as:
%k-call-site sensitive interprocedural pointer
%analysis with a k-context-sensitive heap and onthe-
%fly call-graph construction
% 

\begin{figure}[t]
%\centering
\begin{CodeOut}
\begin{alltt}

<LinearLayout>
    <Button android:id="@+id/\textbf{button\_id}" android:text="A Button" />
</LinearLayout>

1. public class MyActivity extends Activity \{
2.    @Override
3.    public void onCreate(Bundle savedInstanceState) \{
4.        super.onCreate(savedInstanceState);
5.        setContentView(R.layout.main);
6.        Button button = (Button) findViewById(R.id.\textbf{button\_id});
7.        button.setOnClickListener(new Button.OnClickListener() \{
8.            @Override
9.            public void onClick(View v) \{
10.               button.setText("Button Clicked.");
11.           \}
12.       \});
13.   \}
14. \}
\end{alltt}
\end{CodeOut}
\caption{Sample GUI application code on the Android platform. The layout
XML file (the top) specifies a Button object declaratively, 
and the Java code (the bottom)
first loads the XML file (line 6) and then uses reflection to create
a Button object by its ID (line 7, the \CodeIn{findViewById} method).}
\label{fig:sampleandroid}
\end{figure}



\begin{figure}[t]
\textbf{Input}: a Java program $\mathit{P}$\\
\textbf{Output}: a call graph $\mathit{cg}$\\
\vspace{-5mm}
\begin{algorithmic}[1]
\FOR{each $\mathit{statement}$ in $\mathit{P}$}
\IF{isReflectionCall($statement$)}
\STATE $\mathit{objectSet}$ $\leftarrow$ getAllObjectsThatMaybeCreated($\mathit{statement}$)
\STATE $\mathit{newStmt}$ $\leftarrow$ createObjectCreationStatement($\mathit{objectSet}$)
\STATE replace $\mathit{statement}$ with $\mathit{newStmt}$
\ENDIF
\ENDFOR
\STATE $\mathit{cg}$ $\leftarrow$ constructCallGraph($\mathit{P}$)
\RETURN $\mathit{cg}$
%\ENDWHILE
\vspace{-2mm}
\end{algorithmic}
\caption{A reflection-aware call graph construction algorithm. Lines
1--7 show a simple program transformation to replace reflection calls
with object creations, and line 8 builds the call graph using
the existing call graph construction algorithms. This algorithm
is parameterized by three methods on line 2, 3, and 8. How to
instantiate it for the Android framework is presented in Section~\ref{sec:cg}.
%This algorithm is parameterized by the call graph construction
%algorithm on line 8, in which any existing algorithm can be used.
} 
\label{fig:cgalgorithm}
\end{figure}

GUI applications tend to use reflection intensively. In particular, GUI applications built on 
top of the Android framework often use configuration files and reflection to specify GUI layout.
The heavy use of reflection makes many existing call graph construction algorithms such as RTA~\cite{rta}
and k-CFA~\cite{kcfa} fail to build a sufficiently complete call graph.
The example code in Figure~\ref{fig:sampleandroid} from an Android application
 illustrates the limitations.
In Figure~\ref{fig:sampleandroid}, line 6 uses reflection to create a \CodeIn{Button}
object by looking up its id declared in the associated XML file. When this button
is clicked, its event handling code (lines 9--11) updates the text.
%d caches
%the current view to a list declared at line 2.

When analyzing the code in Figure~\ref{fig:sampleandroid}, existing call graph algorithms
fail to conclude that variable \CodeIn{button} declared at line 7
point to a \CodeIn{Button} object due to its limitation in deal with the
reflection call \CodeIn{findViewById}. As a result, the built 
call graph will miss the calling edge corresponding to the
\CodeIn{setText} method call on line 10. 

%Existing call graph algorithms failed either suffer from the problem of incompleteness
%or imprecision. The k-CFA algorithm performs points-to analysis with call
%graph construction interleavingly, and maintains a value set for every expression.
%For example, it concludes that variable \CodeIn{cachedViews} can only point
%to a \CodeIn{LinkedList} object and the method call \CodeIn{add} at line 12
%can only correspond to \CodeIn{LinkedList.add}. However, k-CFA fails to
%conclude that variable \CodeIn{button} declared at line 7 points to
%a \CodeIn{Button} object due to its limitation in dealing with reflection. Thus, k-CFA
%may miss the call graph edge corresponding to the method call \CodeIn{setText} at line 11. 
%On the other hand, using a coarse-grained algorithm like RTA~\cite{} can not still
%fully solve this problem. RTA adds an edge for a method invocation 
%to the call graph if an object of the receiver's type (or subtype)
%has been created.  By explicitly specifying that all GUI classes declared in the XML file will be created,
%RTA maintains a global value set for those GUI object type, and  adds a
%call edge \CodeIn{setText} to the call graph. However, RTA introduces 
%unnecessary precision loss: it may
%may conclude that the \CodeIn{cachedViews} variable at line 2
%point to any created \CodeIn{List} objects created somewhere else in the program, despite the
%fact that \CodeIn{cachedViews} can only be a \CodeIn{LinkedList} object.


To address this limitation, we present an algorithm to construct
a reflection-aware call graph based on simple program transformation.
The basic idea to replace reflection calls with explicit object
creation statements, pretending that the corresponding concrete object
has been created.  The algorithm is shown in Figure~\ref{fig:cgalgorithm}.
It consists of two steps. The first step is a simple program
transformation to replace reflection calls with object creation statements, and
the second step is to use an existing call graph construction algorithm
to build the graph on the transformed program. In our context,
a reflection call represents a framework-specific helper method invocation
that uses Java reflection to create desirable objects, such as
the \CodeIn{findViewById} method in Android applications, instead
of the methods in the \CodeIn{java.lang.reflection} package.
In Figure~\ref{fig:cgalgorithm}, lines 1--7 show
the first step of program transformation. The algorithm iterates through every statement
in an analyzed program. When it sees a reflection call, the algorithm
determines a set of possible objects that might be created.
After that, the algorithm creates a statement which non-deterministically
returns a new object from the object set. Line 8 is the second step
to build a call graph on the transformed program. 

This algorithm is parameterized in three places:
line 2, line 3, and line 8. Currently, we instantiated it
for Android applications as follows. On line 2,
the predicate isReflectionCall returns true
if the $statement$ is a \CodeIn{findViewById(id)} method call. On line 3, method
\CodeIn{getAllObjectsThatMaybeCreated} parses the associated XML configuration
file in an Android application to extract the class declaration
 correspondinig to the given \CodeIn{id} value. If the \CodeIn{id}
value is dynamically generated, the method will conservatively
return instances of all subclasses of the declared type.
On line 8, any existing call graph construction algorithm can be applied.

Take the code in Figure~\ref{fig:sampleandroid} as an example.
When the algorithm sees the reflection call
\CodeIn{findViewById(R.id.button\_id)}, it
parses the XML configuration file to determine that the \CodeIn{button\_id}
value is mapped to a \CodeIn{Button} instance. Then, it
replaces the reflection call
with an explicit object creation statement: \CodeIn{new Button(null)}.
After that, the algorithm employs existing call graph construction
algorithms to analyze the transformed program, and permits them
to add the call edge \CodeIn{setText} to the resulting graph.

As demonstrated in our experiments, using this reflection-aware call
graph construction algorithm helps in detecting errors
in Android applications.

% combined RTA and k-CFA
%call graph construction algorithm as shown in Figure~\ref{fig:cgalgorithm}.
%The algorithm shares the same spirit as k-CFA to maintain a value set
%for each expression. However, the only difference is that our
%algorithm uses a RTA-style treatment for constructing the value set for
%GUI classes: it parses the XML configuration file and explicitly
%adds a raw object to each declared GUI variable's value set. Thus,
%For the example in Figure~\ref{fig:sampleandroid}, this
%combined algorithm concludes that a \CodeIn{Button} object has been
%created at line 7 and then adds the call edge \CodeIn{setText} to the
%graph. It also concludes that the \CodeIn{cachedViews} variable can only
%point to a \CodeIn{LinkedList} object, avoiding unnecessary precision loss.


\subsubsection{Annotation for Native Methods}
\label{sec:annotation}

A GUI application may also use native methods to interact with the underlying
operating system or platforms. However, native methods are often
beyond the ability of a static analysis but should be considered to make
the call graph more complete. To do so, we provide an annotation \CodeIn{@CalledByNativeMethods}
for users to specify which native methods may call the current method. For example,
the following code snippet indicates that native methods \CodeIn{native1()} and \CodeIn{native2()}
may call method \CodeIn{javaMethod()}.

%\noindent 
\CodeIn{@CalledByNativeMethods(callers=\{"native1", "native2"\})}

%\noindent
\CodeIn{public void javaMethod() \{ ... \}}


Our static analysis takes the call relationship specified by this
annotation into consideration when traversing the call graph. 
Adding annotations for native methods is optional and requires manual effort.
In our experiments, 1 out of 9 programs uses native methods to interact with
the underlying operating system. We manually
searched the source files, inspected the code to determine possible target methods
that may be called by a native method, and added 7 annotations for this program.
We found such an annotation, was useful: one real-world bug
can only be reported with using the user-provided annotations.

\subsubsection{Heuristic Filtering}
\label{sec:heuristic}

A static analysis can exhaustively check possible error paths; but may report
paths that do not actually exist (false positives) or multiple paths
 that have the same error cause (redundant warnings). We devise
5 heuristic filters below to remove likely false positives and redundant warnings.
The first two are for eliminating false positives, while the
other three are for reducing redundant warnings.

%To remove those 
%The potentially huge volume of
%reported warnings impose great burden for programmers to check
%its validity. 


%\begin{enumerate}

%\item
\textbf{1. Filter Reports Containing Library Calls}. A
method call chain containing certain library calls
like \CodeIn{Runtime.}\CodeIn{shutDown} are unlikely to
be buggy. For example, the \CodeIn{Runtime.}\CodeIn{shutDown}
method is called when JVM terminates, and uses multithreading
to dispose all GUI objects when the program exits.
We encoded a list of common library calls by proposing
a basic set of library calls that seemed natural and
generally applicable, based on our programming experience.
We later added other calls we found helpful in analyzing
programs and that we believed would be generally useful.
%We encode a list of common library calls, and remove
%a reported warning if it contains one of them.
%Method call chain involves typical
%system methods like \CodeIn{toString} are unlikely to reveal a real bug.

%\item
\textbf{2. Filter Reports With User-Annotated Methods}. Besides general
library calls, users are also permitted to explicitly annotate methods
that will never trigger an error.
For example, the Android GUI framework provides a utility method 
\CodeIn{runOnUIThread} to check whether the current thread is the UI thread
before executing the code. If the current thread is not the UI thread,
this method will be executed on the UI thread via message passing.
Thus, a reported method call chain containing an annotated 
can be safely removed. We produced the list of annotated methods
during analyzing programs on different GUI frameworks; we did this
only between experiments rather than biasing experiments by tuning
our tool to specific programs.

%\item
\textbf{3. Filter Lexically Redundant Reports}. One reported method call
chain can lexically subsume another one. For example, suppose that two
reported method call chains: \CodeIn{a()} $\rightarrow$ \CodeIn{b()}
$\rightarrow$ \CodeIn{c()} and 
\CodeIn{d()} $\rightarrow$ \CodeIn{a()} $\rightarrow$ \CodeIn{b()} $\rightarrow$ \CodeIn{c()}
both lead to a potential error, since $\CodeIn{d()}$ and \CodeIn{a()}
are two distinct entry methods. The second call chain should
be removed and would not reduce the error detection capability, since
the first chain reveals the same error and is shorter 
for programmers to interpret.


%\item
\textbf{4. Filter Report With the Same Head Methods from The Entry Node to \CodeIn{Thread.start()}}. A method can call
multiple methods that access GUI objects, such as:
\vspace{-1mm}
\begin{CodeOut}
\begin{alltt}
     public void m() \{
         accessUIObject1();
         accessUIObject2();
     \}
\end{alltt}
\end{CodeOut}
If method \CodeIn{m()} is invoked by a non-UI thread, method call chains
only different at the last few method nodes may be reported separately, such as:

\CodeIn{a()} $\rightarrow$ ...\CodeIn{Thread.start()} ... $\rightarrow$ \CodeIn{m()} $\rightarrow$ \CodeIn{accessUIObject1()} ...

\CodeIn{a()} $\rightarrow$ ...\CodeIn{Thread.start()} ... $\rightarrow$ \CodeIn{m()} $\rightarrow$ \CodeIn{accessUIObject2()} ...

In fact, these two chains may have the same error root: knowning that
method \CodeIn{m()} is called by a non-UI thread is sufficient to understand
the error.
This filter compares two reported chains that have the same
head nodes from the entry node to \CodeIn{Thread}.\CodeIn{start()},
and removes the longer one.
%$k$ ($k$ is user-settable with a default value 5 as used in our
%experiments) head nodes, and removes the longer one.

%\item
\textbf{5. Filter Report With the Same Tail Methods from \CodeIn{Thread}.
\CodeIn{start()}
to the UI-accessing node}. A method
can have multiple callers, so that method call chains with the same tail are likely
to just represent different ways to trigger the same error. This filter compares 
two reported method call chains that share the same 
tail nodes from \CodeIn{Thread.start()} to the UI-accessing node, and
removes the longer one.


%\end{enumerate}
\vspace{1mm}

Among the above 5 filters, filters 2 and 3 are sound
in that they will not filter real bugs, and other 3 filters
are based on heuristics. However, as we will show in
our experiments, these filters work remarkably well in practice.


\subsection{Instantiation for Different Frameworks}
\label{sec:platforms}

We instantiated our error detection techniques for four popular GUI frameworks,
namely SWT, Eclipse plugin environment, Swing, and Android.
The major framework-specific parts, corresponding to
lines 3 -- 5 in Figure~\ref{fig:detectalgorithm}, are identifying
\textbf{call graph entry nodes}, \textbf{UI-accessing nodes},
and \textbf{safe UI methods} for each framework.

%For each framework, we customize the \textit{call graph entry nodes},
%\textit{UI-accessing nodes}, and \textit{safe UI methods} on lines 3 -- 5 of the
%algorithm in 

%For the sake of efficiency, they are all using the single thread model.

%Why these platforms? popularity? account for xxx\% of the GUI. 

%customizing for: entry points, starting points, error checking, thread
%safe UI methods, what is the event dispatching thread?

\subsubsection{SWT}

%SWT is an open source widget toolkit for Java designed to provide efficient,
%portable access to the GUI facilities of the operating systems
%on which it is implemented.
 We instantiate our technique for SWT applications as follows:

%starts from its main method.
%By default, the thread that the main method executes is the UI thread. Thus, we instantiate
%our technique as follows:

\begin{itemize}

\item \textbf{Call graph entry nodes: } the main method. Like a normal Java program,
a SWT desktop application has a single main method as the entry point. By default,
the main method is executed in the UI thread after the GUI
is initialized.

\item \textbf{UI-accessing nodes: } SWT provides explicit
runtime checking for illegal thread accesses. Any methods for manipulating
GUI object must call methods \CodeIn{Display.checkWidget}
or \CodeIn{Display.checkDevice} to check whether the current
thread is the UI thread or not.
Thus, we treat nodes (in both user code and framework code) corresponding to the methods
calling these two methods as UI-accessing nodes.
%must be invoked
%in the UI thread and as
%UI-accessing nodes.
 %methods to check whether the current thread is the UI thread or not. Thus,
%we treat all such methods that must be accessed in UI thread as the UI-accessing nodes.
% any
%methods that calls the checking method must be invoked in the UI thread.

\item \textbf{Safe UI methods: } SWT provides two helper methods (\CodeIn{Display.\\asyncExec}
and \CodeIn{Display.syncExec}) to execute code (a)synchro-
nizedly on the UI thread.
 We treat these two methods as safe UI methods, since invoking
them from any thread 
%are generally considered to be safe and
will not cause an invalid thread access error.

\end{itemize}

\subsubsection{Eclipse plugin}

We instantiated our technique for Eclipse plugin as follows:

%Unlike a SWT desktop application, an eclipse plugin, though is developed in SWT,
%has no single main method. Instead, (XXX) it extends certain extension points
%exposed by the eclipse framework, and eclipse will XXX. To instantiate our
%technique for the Eclipse plugin environment, we need to redefine the
%call graph entry points.

\begin{itemize}

\item \textbf{Call graph entry nodes: } all overridden SWT
GUI event handling methods in user code. An Eclipse plugin is built
on top of the SWT framework. However, unlike a SWT desktop application,
an Eclipse plugin does not have a main method as a single program entry point.
Instead, it extends extension points provided by the Eclispe framework to
implement functionalities. When a plugin starts, Eclipse
calls back the overridden extension points to handle the
upcoming events. All SWT GUI event handling methods (i.e., the overriden
methods in the class that implements \CodeIn{org.eclipse.swt.internal.\\SWTEventListener}) are
always called back from the UI thread, thus are used as call graph entry nodes.


\item \textbf{UI-accessing nodes} and \textbf{Safe UI methods} are the same as SWT.

\end{itemize}

\subsubsection{Swing}


A Swing application has a single main method, but contains three kinds of
threads: \textit{initial thread} that executes initial application code from the main method,
the \textit{UI thread}, where all GUI manipulation code is executed,
and the \textit{worker thread} where time-consuming background tasks are executed.
After a Swing program starts, its initial thread exits and the UI thread takes charge
of the application and starts to execute all event-handling code or spawns new worker threads. 
We instantiated our technique for Swing as follows:
%XXXX Therefore, we can not directly
%use the main method as the call graph entry point, since it is not invoked
%by the UI thread. We customize our technique for Swing as follows:

\begin{itemize}

\item \textbf{Call graph entry nodes:} all overridden Swing GUI event handling
methods in user code. Those event handling methods are always
called back from the UI thread, thus are used as call graph entry nodes.

\item \textbf{UI-accessing nodes:} %as stated in Swing's documentation,
all methods defined in each Swing GUI class except for three thread-safe
methods: \CodeIn{repaint()}, \CodeIn{revalidate()}, and \CodeIn{invalidate()}.

%methods in Swing GUI classes are not thread safe with three exceptions (\CodeIn{repaint()},
%\CodeIn{revalidate()}, and \CodeIn{invalidate()}). Thus, we
%treat nodes corresponding to those thread unsafe methods %in each GUI class
%as UI-accessing nodes.

\item \textbf{Safe UI methods: }  Swing's two helper methods
(\CodeIn{SwingUtilities.} \CodeIn{invokeLater} and \CodeIn{SwingUtilities.invokeAndWait}) that execute code in the UI thread.

\end{itemize}

\subsubsection{Android}

Android is a Java-based platform for embedded or mobile devices. 
An Android program does not have a single entry point but can
rather use parts of other Android applications on-demand and can require their
services by calling corresponding event handlers, directly or through the
operating system. In particular, Android applications use \textit{activities}
as code interacting with the user through a visual interface. GUI event handlers
are handled in an activity.
We instantiated our technique for Android programs as follows:
%, with some notable exceptions such as the
%lifecycle of activities.

%An XML \textit{manifest file} registers the components of an application. Other XML
%fies describe the visual layout of the activities. Activities \textit{inflate}
%layout files into
%visual objects (a hierarchy of views), through an \textit{inflater} provided by the An-
%droid library. This means that library or user-defined views are not explicitly
%created by new statements but rather inflated through \textit{reflection}. Library meth-
%ods such as \CodeIn{findViewById} access the inflated views.


\begin{itemize}

\item \textbf{Call graph entry nodes: }in an Android application,
an \CodeIn{Activity} object is created and manipulated by the UI thread. Thus, we treat
all public methods defined in the \CodeIn{Activity} class 
and any class in the user code that subclasses \CodeIn{Activity} as call graph entry nodes.
We also add all overridden Android GUI event handling methods in user
code as entry nodes, since they must be called back from the UI thread.

\item \textbf{UI-accessing nodes: }every method calling \CodeIn{ViewRoot.checkThread}.
Like SWT, Android explicitly checks whether
the current thread is a UI thread or not before accessing a GUI object via
the \CodeIn{ViewRoot.checkThread} method.
%method. This method is the only place where Android checks
%whether the current thread is the UI thread or not. Thus, any
%methods that calls the checking method must be invoked in the UI thread.

\item \textbf{Safe UI methods: } two helper methods (\CodeIn{View.post}
and \CodeIn{View.postDelay}) that execute code on the UI thread. 
Invoking them on any thread will not cause an invalid thread access error.
%Invoking these two methods are generally considered to be safe, and would not
%cause an invalid thread access error.

\end{itemize}


Instantiating our general technique to  a specific framework
requires moderate human effort. We wrote around 500 lines of Java code in total to achieve
the above four instantiations.



\tinystep
\tinystep
\section{Implementation}
\label{sec:implementation}

We implemented an error detection tool on top of the WALA framework~\cite{walatutorial}.
We instantiated the proposed technique for four widely-used
GUI frameworks, namely SWT, Eclipse plugin framework, Swing, and Android.

\tinystep


\subsection{Instantiation for Different Frameworks}
\label{sec:platforms}

When instantiating our error detection technique for different
frameworks, the major framework-specific parts, corresponding to
lines 3--5 in Figure~\ref{fig:detectalgorithm}, are identifying
\textbf{call graph entry nodes}, \textbf{UI-accessing nodes},
and \textbf{safe UI methods} for each framework.

\tinystep
\subsubsection{SWT}

 We instantiated our technique for SWT applications as follows:

\preitemizespace

\begin{itemize}

\item \textbf{Call graph entry nodes:}  the main method. 
% Like a normal Java program,
% a SWT desktop application has a single main method as the entry point. By default,
% the main method
It is executed in the UI thread after the GUI
is initialized.

\smallstep

\item \textbf{UI-accessing nodes:}  the \CodeIn{Widget.checkWidget}
and \CodeIn{Display\-.check\-Device} methods.
% SWT performs
% runtime checking before accessing a GUI object via these methods.
% If the current thread is a non-UI thread, the \CodeIn{Widget.checkWidget}
% and \CodeIn{Display.checkDevice} methods throw a \CodeIn{RuntimeException}.
If the current thread is a non-UI thread, these
methods throw a \CodeIn{RuntimeException}.


\smallstep

\item \textbf{Safe UI methods:}  \CodeIn{Display.asyncExec}
and \CodeIn{Display.syncExec}.
These methods
execute code (a)synchro\-nously on the UI thread.

\end{itemize}

\tinystep
\tinystep
\subsubsection{Eclipse plugin}

We instantiated our technique for the Eclipse plugin framework as follows:

\preitemizespace

\begin{itemize}

\item \textbf{Call graph entry nodes:}  all user code methods that override
 SWT GUI event handling methods. Eclipse
calls back the overridden methods to handle the
events. All SWT GUI event handling methods (i.e., the overridden
methods in a class that implements \CodeIn{org.eclipse.swt.internal.SWTEventListener}) are
always called back from the UI thread.
%, thus are used as call graph entry nodes.

\tinystep

\item \textbf{UI-accessing nodes} and \textbf{Safe UI methods} are the same as SWT\@.

\end{itemize}

\tinystep
\tinystep
\tinystep
\subsubsection{Swing}

A Swing application has a single main method, but contains three kinds of
threads: the \textit{initial thread} that executes initial application code from the main method,
the \textit{UI thread}, where all GUI manipulation code is executed,
and the \textit{worker thread} where time-consuming background tasks are executed.
After a Swing program starts, its initial thread exits and the UI thread takes charge
of the application and starts to execute event-handling code or spawn new worker threads. 
We instantiated our technique for Swing as follows:

\preitemizespace

\begin{itemize}

\item \textbf{Call graph entry nodes:} all user code methods that override
Swing GUI event handling methods. Those event handling methods are always
called back from the UI thread.

\tinystep

\item \textbf{UI-accessing nodes:} %as stated in Swing's documentation,
all methods defined in each Swing GUI class except for three thread-safe
methods: \CodeIn{repaint()}, \CodeIn{reval\-i\-date()}, and \CodeIn{invalidate()}.

\tinystep

\item \textbf{Safe UI methods:}  
\CodeIn{SwingUtilities.}\CodeIn{invokeLater} and \CodeIn{Swing\-Util\-ities.invokeAndWait}, which execute code on the UI thread.

\end{itemize}

\smallstep

\tinystep
\subsubsection{Android}

Android is a Java-based platform for embedded or mobile devices. 
An Android program does not have a single entry point.
%but can
%rather use parts of other Android applications on-demand and can require their
%services by calling corresponding event handlers, directly or through the
%operating system. In particular,
%Android applications 
It uses \textit{activities}
(i.e., instances of the \CodeIn{Activity} class)
to interact with users through a visual interface and handle GUI events.
We instantiated our technique for Android programs as follows:


\preitemizespace

\begin{itemize}

\item \textbf{Call graph entry nodes:} in an Android application,
an \CodeIn{Activity} object is created and manipulated by the UI thread. Thus, we treat
all public methods defined in the \CodeIn{Activity} class 
and any overriding definitions in its subclasses as call graph entry nodes.
We also add all user code methods that override Android GUI event handling methods
as entry nodes, since they are called back from the UI thread.

\tinystep

\item \textbf{UI-accessing nodes:} the \CodeIn{ViewRoot.checkThread} method.
% Like SWT, Android explicitly performs runtime checking before accessing
% a GUI object via the \CodeIn{ViewRoot.checkThread} method.
If the current
thread is a non-UI thread, the \CodeIn{ViewRoot.checkThread} method throws
a \CodeIn{RuntimeException}.

\tinystep

\item \textbf{Safe UI methods:}  \CodeIn{View.post}
and \CodeIn{View.postDelay}, which execute code on the UI thread. 

\end{itemize}

\tinystep

Given a GUI application using a supported framework, our tool 
takes its Java bytecode as input, 
plugs in the corresponding instantiation parameters at runtime,
and automatically detects potential invalid thread access
errors.

\subsection{Android-specific Implementation Details}


We implemented the reflection-aware call graph construction
 algorithm (Section~\ref{sec:cg}) using WALA's \textit{bypass logic}.
Unlike other tools~\cite{Payet:2011:SAA:2032266.2032299}, our tool
does not require a separate pass for program instrumentation; instead, it
parses the configuration file in an Android application,
and then intercepts the call graph construction
process on-the-fly to replace all reflection calls with object creation expressions.
Since Android applications are often fully encrypted and shipped in Dalvik
bytecode as a single apk file, our tool first uses
android-apktool~\cite{apktool} to
decrypt the apk file, and then uses the 
ded translator~\cite{ded} to convert
Dalvik bytecode to Java bytecode before feeding to WALA\@.  The Android system
library (i.e., \CodeIn{android.jar}) uses many ``stub'' classes as
placeholders for the sake of efficiency. We manually re-compiled 
\CodeIn{android.jar} from its source code, so it contains real
class files rather than stubs.


%%% Local Variables: 
%%% mode: latex
%%% TeX-master: "guierror"
%%% TeX-command-default: "PDF"
%%% End: 

%  LocalWords:  WALA SWT plugin checkWidget RuntimeException asyncExec apk
%  LocalWords:  syncExec synchro nously reval SwingUtilities invokeLater
%  LocalWords:  Util ities invokeAndWait ViewRoot checkThread postDelay
%  LocalWords:  WALA's Dalvik apktool ded



\tinystep
\section{Empirical Evaluation}
\label{sec:evaluation}

Our experimental objective is three-fold: to demonstrate the effectiveness
of our approach in detecting real errors in multithreaded GUI applications, to 
compare our call graph construction algorithm
with existing ones, and to evaluate the usefulness of the proposed 
error filters.  

First, we describe our subject programs (Section~\ref{sec:subjects}) and the experimental procedure (Section~\ref{sec:procedural}).
We then show that our technique detects bugs in real-world GUI applications (Section~\ref{sec:errors}).
We also compare our technique with
a straightforward approach (Section~\ref{sec:straightforward}),  compare
different call graph construction algorithms (Section~\ref{sec:reflectionaware}), and
evaluate various graph search strategies in error detection (Section~\ref{sec:search}).
 Finally, we show that the proposed filters are effective
in removing warnings (Section~\ref{sec:filters}). 



\subsection{Subject Programs}
\label{sec:subjects}

We used \subnum open-source projects from SourceForge, Google Code,
and the Eclipse plugin marketplace as evaluation subjects. 

Five subjects (EclipseRunner, HudsonEclipse, SGTPuzzler, Fennec, and MyTracks)
are selected because they have known invalid thread access errors (1 error per subject) and
all errors have been fixed in later revisions.
We used the buggy versions to check whether our tool
can correctly identify those known errors. 
For the other four subjects, we selected them
by first searching for the
framework keywords (e.g., ``Java Swing'' or ``Java SWT'')
in the above source repositories, and then choosing subjects based on the following
criteria. First, the subject must be a Java application, not
an open library. Second, the subject must use multithreading in its implementation.
Third, the subject is listed in the first 5 result pages.
This permits us to exclude immature subjects that may contain obvious errors.
For each selected subject, we ran our tool on the latest stable release to find new errors.
%of each project to check whether
%it can find any new errors.

The subjects used in our experiment (Table~\ref{table:subjects})
include end-user applications, programming tools, and games. 

\vspace{-5pt}

% \begin{myindentpar}{-2mm}
\begin{itemize}%\addtolength{\itemsep}{-0.5\baselineskip}
\item \textbf{SWT desktop applications.}
%VirgoFTP~\cite{virgo} and FileBunker~\cite{filebunker} are two useful SWT desktop applications.
%VirgoFTP~\cite{virgo} implements a simple FTP client based on Java and the SWT UI library.
FileBunker~\cite{filebunker} is a file backup application that uses one or more GMail
accounts as its backup repository. ArecaBackup~\cite{areca} offers a local
file backup solution for Linux and Windows.

% \vspace{0.5mm}
% \vspace{0.5mm}

\item \textbf{Eclipse plugins.}
HudsonEclipse~\cite{hudson} monitors Hudson build status from Eclipse.
EclipseRunner~\cite{eclipserunner} extends Eclipse's capability of running launch configurations.

% \vspace{0.5mm}

\item \textbf{Swing applications.} %S3dropbox~\cite{s3dropbox} and SudokuSolver~\cite{sudokusolver} are two Swing desktop applications.
S3dropbox~\cite{s3dropbox}  allows users
to drag and drop files to their Amazon S3 accounts. SudokuSolver~\cite{sudokusolver}
computes Sudoku solutions using mutlithreaded execution. 

% \vspace{0.5mm}

\item  \textbf{Android applications.} %MyTracks~\cite{mytracks}, Fennec~\cite{fennec}, and SGTPuzzles~\cite{sgtpuzzles} are three Android applications.
SGTPuzzler~\cite{sgtpuzzles} is a single-player logic game.
Fennec~\cite{fennec}, developed by Mozilla, is the Mozilla
Firefox web browser for mobile devices. 
MyTracks~\cite{mytracks}, developed by Google, records users' GPS tracks, and provides
interfaces to visualize them on Google Maps. 

\end{itemize}
% \end{myindentpar}


\begin{table}[t]
\begin{center}
 \fontsize{9pt}{\baselineskip}\selectfont
\hspace*{-0.2cm}
\setlength{\tabcolsep}{.6\tabcolsep}
\begin{tabular}{|p{4.4cm}||c|c|c|c|}
\hline
 Program (version) & LOC & Classes & Methods \\
\hline \hline
\multicolumn{4}{|l|}{SWT desktop applications}   \\
 %\hline
 %VirgoFTP (1.3.5) &  10815 &  116 &  1955\\
 \hline
 FileBunker (1.1.2)&  14237 &  150 &  1106  \\
 \hline
 ArecaBackup (7.2)&   23226 &  444 &  4729 \\
 \hline
 \hline
\multicolumn{4}{|l|}{Eclipse plugins}   \\
 \hline
 EclipseRunner (1.0.0) &  3101 &  48 &  354\\
 \hline
 HudsonEclipse(1.0.9)&  11077 &  74 &  649 \\
 \hline
 \hline
\multicolumn{4}{|l|}{Swing desktop applications}   \\
 \hline
 S3dropbox (1.7) &  2353 &  42  &  224 \\
 \hline
 SudokuSolver (1.06)&  3555 &  10 &  62 \\
 \hline
 \hline
\multicolumn{4}{|l|}{Android mobile applications}   \\
 \hline
 SGTPuzzler (v9306.11)&  2220 &  16 &  148 \\
 \hline
 Fennec (d7fa4814218d)&  8577 &  51 &  620 \\
 \hline
 MyTracks (01d5c1e1cd47)&  20297 &  143 &  1374 \\
\hline
\hline
 Total &  \totaloc &  978 &  9266 \\
\hline
\multicolumn{4}{l}{}   \\
\hline
 GUI framework (version) & LOC & Classes & Methods  \\
\hline \hline
 SWT (3.6)&  129942 &  999 &  9643 \\
\hline
 Eclipse plugin development (3.6.2)&  460830 &  6630 &  37183 \\
\hline
Swing (1.6)&  167961 &  878 &  13159 \\
\hline
 Android (3.2)&  683289 &  5085 &  10584 \\
\hline
\hline
 Total &  1442022 &  13592 &  70569\\
\hline
\end{tabular}

\end{center}
\vspace{-15pt}
\Caption{{\label{table:subjects} Open-source programs
used in our evaluation. Column ``LOC'' is the number of non-blank, non-comment lines
of code, as counted by LOCC~\cite{locc}.  Each program is analyzed
together with its GUI framework, as listed in
the bottom table.} }
\end{table}

\smallstep
\tinystep

\subsection{Experimental Procedure}
\label{sec:procedural}

We ran our tool on each subject with three call graph construction
algorithms: RTA~\cite{rta}, 0-CFA, and 1-CFA~\cite{kcfa}.  When running
each call graph construction algorithm on three Android applications, we
used two configurations: with and without our enhancements
(Section~\ref{sec:cgcon}).  We did not use more expensive algorithms like $k$-CFA ($k >$ 1),
because they do not scale to our subject programs.

Subject SGTPuzzler uses native methods to interact with
the underlying operating system. For it, we manually checked the possible
Java methods that a native method may call, and then added \annotationnum \CodeIn{@CalledByNativeMethods}
annotations for it. 
Two subjects (MyTracks and Fennec)
use a customized pattern to interact with the GUI framework.
For them, we added \filternum user-defined filters, described in Section~\ref{sec:heuristic}.
In this experiment, none of the paper
authors was familiar with the subjects, but we found it was quite easy
to add extra annotations and user-defined filters. All these
manual parts took less than a total of 60 minutes.

We manually determined the validity of each warning.
For a known error, we compared the generated report (i.e., an
error-revealing method call chain as shown in Figure~\ref{fig:report}) against the
actual bug fix to check whether the tool identified the
buggy method. For a previously-unknown error, we submitted a new bug
report to its developers, and wrote a test driver to reproduce it.
%wrote a test driver to 


\begin{table*}[ht]
\begin{center}
 \fontsize{9pt}{\baselineskip}\selectfont
\hspace*{-0.2cm}
\setlength{\tabcolsep}{.14\tabcolsep}
\begin{tabular}{|l||c|c|c||c|c|c||c|c|c||c|}
\hline
 Subject&  \multicolumn{9}{|c||}{Our Technique} & Requiring Wrappers  \\
\cline{2-10}
 Program  &  \multicolumn{3}{|c||}{RTA }& \multicolumn{3}{|c||}{0-CFA } & \multicolumn{3}{|c||}{1-CFA } & (Section~\ref{sec:straightforward})  \\
\cline{2-11}
 & CG Size & \#Warning & \#Bug & CG Size & \#Warning & \#Bug & CG Size & \#Warning & \#Bug & \#Warning \\
\hline \hline
\multicolumn{11}{|l|}{SWT desktop applications}   \\
 %\hline
 %VirgoFTP&  12401 &  1 &  1 & 10858 & 1 & 1 & 43598 & 2 & 2& 149  \\
 \hline
 FileBunker &  18951 &  1 &  0 & 15743 & 0 & 0 & 76088 & 2 & 1& 693  \\
 \hline
 ArecaBackup&  20882 &  1  &  0 & 19697 & 1 & 1  & 116398 & 1 & 1 & 3021\\
 \hline
 \hline
\multicolumn{11}{|l|}{Eclipse plugins}   \\
 \hline
 EclipseRunner&  11248&  6 &  1 & 7201 & 6 & 1 & 26911 & 6 & 1& 202  \\
 %\hline
 %FileSync&  12132 &  xx &  xxx &8235 & xx & xx& 32565 & 1 & 1 & 331 & 1 \\
 \hline
 HudsonEclipse& 18473 &  2 &  1 & 15814 & 2 & 1& 56645 & 3 & 1 & 182 \\
 \hline
 \hline
\multicolumn{11}{|l|}{Swing desktop applications}   \\
 \hline
 S3dropbox & 37751 &  0 &  0 & 30609 & 0 & 0 & 115324 & 1 & 1 & 210  \\
 \hline
  SudokuSolver&  27730&  3 &  2 & 20907 & 3 & 2 & 39299 & 2 & 2 & 356  \\
 \hline
 \hline
\multicolumn{11}{|l|}{Android mobile applications}   \\
 \hline
 SGTPuzzler & 13631 / 13865&  1 / 16 &  0 / 0 & 9546 / 9682& 0 / 4& 0 / 1 & 35198 / 35756 & 0 / 1  & 0 / 1& 104 \\
 \hline
 Fennec & 14058 / 14387 &  1 / 1 &  0 / 0 & 8263 / 8898 & 1 / 1 & 0 / 0& 29125/ 31759 & 3 / 3 & 1 / 1& 433 \\
 \hline
 MyTracks & 24036 / 24036 &  161 / 220 & 0 / 0 & 10803 / 13645 & 119 / 119 & 0 / 0 & 39235 / 110977 & 1 / 1 & 0 / 1 & 1192 \\
\hline
\hline
 Total & 186760 / 187323&  176 / 250 & 4 / 4 & 138583 / 142196& 132 / 136& 5 / 6 & 534223 / 609158& 19 / 20 & 8 / 10 & 6393\\
\hline
\end{tabular}
\end{center}
\vspace{-15pt}
\Caption{{\label{table:results}Experimental results in finding invalid-thread-access errors
in multithreaded GUI programs. Column ``CG Size'' is the number of nodes in the
call graph. Column ``\#Warning''
is the number of warnings issued by our tool. Column ``\#Bug'' shows
the actual bugs found. Column groups ``RTA'', ``0-CFA'',
and ``1-CFA'' show the results of using different
call graph construction algorithms.
For the Android applications, a slash ``/'' separates the result of
using standard call graph construction algorithm and our call
graph construction algorithm in Section~\ref{sec:cgcon} (dealing
with reflection calls and adding \annotationnum native method annotations
for SGTPuzzler). The \newbugs errors found in FileBunker, ArecaBackup,
S3dropbox, and SudokuSolver are previously  unknown. As a comparison, the
results of the ``Requiring Wrappers'' approach (Section~\ref{sec:straightforward}),
are shown at the far right.} }
\end{table*}

\tinystep
\subsection{Results}
\label{sec:results}

%Table~\ref{table:results} show the errors in
%our subject programs, and Table~\ref{table:filters} show
%the effect of applying each heuristic filter.
\smallstep

\subsubsection{Errors in Multithreaded GUI Applications}
\label{sec:errors}

As shown in Table~\ref{table:results}, our tool found
errors in each subject. Using the 1-CFA
call graph algorithm, our tool issued \warnings warnings, among
which \bugs warnings reveal \bugs distinct errors (\newbugs were previously unknown,
and the \oldbugs known errors were correctly identified),
2 warnings are false positives, and the remaining 8 warnings
are redundant. Section~\ref{sec:reflectionaware} compares with other call graph
construction algorithms.

We submitted all \newbugs new errors to the respective developers. As of May 2012, 
1 error S3dropbox has been confirmed, and we have reproduced other errors,
in which 2 errors in FileBunker and ArecaBackup
have not yet been verified by the developers. 
%2 errors in SudokuSolver can be reproduced by simply
%clicking its UIs, and the rest 4 errors can be reproduced
%by writing a test driver have not yet been verified
%by the developers.
All found bugs and our experimental results are
publicly available at: \url{http://www.cs.washington.edu/homes/szhang/guierror/}.




\begin{figure}[t]
\hspace{4mm}{In class: com.tomczarniecki.s3.gui.DeleteBucketAction}
\vspace{-2mm}
\begin{CodeOut}
\begin{alltt}
59.private void deleteBucket() \ttlcb
60.    executor.execute(new Runnable() \ttlcb
61.        public void run() \ttlcb
62.            try \ttlcb
63.                controller.deleteCurrentBucket();
64.            {\ttrcb} catch (Exception e) \ttlcb
65.                logger.info("Delete failed", e);
66.                deleteError(); 
67.            \ttrcb
68.       \ttrcb
69.    \ttrcb);
70.\ttrcb

77.private void deleteError() \ttlcb
78.    String text = "Cannot delete folder .....";
79.    display.showErrorMessage("Delete failed", 
        String.format(text, controller.getSelectedBucketName())); 
80.\ttrcb
\end{alltt}
\end{CodeOut}
\vspace*{-15pt}
\Caption{{\label{fig:swingerror} An
invalid thread access error reported by our tool
in the S3dropbox Swing application. The error occurs when
the method \CodeIn{deleteCurrentBucket} invoked on line 63 throws
an exception, which causes method \CodeIn{deleteError} to
access a Swing GUI object \CodeIn{display} on line 79 from
a non-UI thread. This error was previously unknown, and has been confirmed by the S3dropbox developers.
}} %\vspace{-5mm}
\end{figure}

%Figures~\ref{fig:swingerror} amd~\ref{fig:pluginerror} show
%two real errors our tool has found. 
Figure~\ref{fig:swingerror}
shows an invalid thread access error our tool found in S3dropbox.
This error happens when the \CodeIn{deleteCurrentBucket}
call on line 63 throws an exception, making it hard to
detect by testing. We reported this error to the S3dropbox developers. Tom Czarniecki,
a key developer of S3dropbox, confirmed this single-GUI-thread
violation. He mentioned that the S3dropbox project uses
certain design patterns to avoid such violations (e.g.,
actions for UI interaction are encapsulated into a \CodeIn{Worker} interface),
but the developers still overlooked  the error our tool found.
Another reason they overlooked this violation is because some GUI frameworks like Swing
do not provide any runtime checks for invalid thread accesses. The Swing
GUI does not exhibit user-visible faults on some erroneous executions.
However, as clearly stated in the official documentation~\cite{swing}, %Swing is a single-GUI-threaded GUI toolkit and
accessing Swing GUI objects from non-UI threads risks thread interference
or memory-consistency errors.



\begin{figure}[t]
\hspace{4mm}{In class: com.eclipserunner.views.impl.RunnerView}
\vspace{-2mm}
\begin{CodeOut}
\begin{alltt} 
179.private void initializeResourceChangeListener() \ttlcb
180.  ResourcesPlugin.getWorkspace().addResourceChangeListener(
        new IResourceChangeListener() \ttlcb
181.      public void resourceChanged(IResourceChangeEvent event) \ttlcb
182.        refresh();
183.      \ttrcb
184.  \ttrcb, IResourceChangeEvent.POST\_CHANGE);
185.\ttrcb

414.public void refresh() \ttlcb
415.  getViewer().refresh();
416.\ttrcb
\end{alltt}
\end{CodeOut}
\vspace*{-15pt}
\Caption{{\label{fig:pluginerror} An
invalid thread access error reported by our tool
for the EclipseRunner plugin. In Eclipse, the
callback method \CodeIn{resourceChanged} on line 181
is invoked by non-UI threads when a \CodeIn{ResourceChangeEvent}
happens. However, the \CodeIn{refresh} method directly accesses
GUI objects (to refresh the view on line 415) without any
protection and thus triggers the error.
This error was reported by other users 13 months after the
buggy code was checked in, and fixed by developers.
}} %\vspace{-5mm}
\end{figure}

Figure~\ref{fig:pluginerror} shows an error found in the EclipseRunner
plugin. This error is event-related. It happens when a 
\CodeIn{ResourceChangeEvent} happens, which then invokes the \CodeIn{refresh}
method on line 415 to update the user interface. In EclipseRunner,
the \CodeIn{refresh} method is
called by 6 different methods from the same non-UI thread. Thus, our
tool issues 6 separate warnings to indicate
6 different ways to trigger this error. 5 of the warnings are redundant.

Besides the above two examples, other errors our tool reported are also
subtle to find. For example, our tool found two new errors in SudokuSolver.
One error only happens when
the given Sudoku is unsolvable.  The other one happens
when the program fails to launch the mail-composing window of the
user's default mail client (i.e., the \CodeIn{java.awt.Desktop.mail()}
method throws an exception after a user clicks the ``eMail Me'' button).

We found GUI developers have already
used design patterns, runtime checks, and testing to avoid violating
the single-GUI-thread rule. However, due to the huge space of
possible UI interactions, hard-to-find
invalid thread access errors still exist.
%Thus, a static analysis as presented in this paper
%would be useful.


\vspace{1mm}

\noindent \textbf{\textit{Summary.}} Our technique can find real errors
in multithreaded GUI applications with
acceptable accuracy.

\subsubsection{Comparison to Requiring Wrappers}
\label{sec:straightforward}

As mentioned in Section~\ref{sec:finding}, one way to prevent
invalid thread access errors is to wrap every GUI-accessing operation
with message passing (i.e., via the safe UI methods).  A wrapper is not
always necessary, and indiscriminate wrapping
can give rise to other types of errors.  Nonetheless, a
straightforward and sound way to detect potential invalid thread access errors
is to issue a warning whenever a GUI-accessing operation
is not wrapped. 

This approach identifies every
error that our technique found.
However, requiring wrappers 
issues a huge number of warnings,
most of which are probably false positives (see the far right column of
Table~\ref{table:results}).
The primary reason is that this simple approach
does not globally reason about the calling relationship between
threads, UI-accessing methods, and safe UI methods, and thus it often incorrectly
classifies GUI accessing operations which will never be executed
in a non-UI thread as erroneous. Furthermore, our technique
outputs a method call chain with each reported error, which can help
developers understand how an invalid thread access error is triggered.



\vspace{1mm}

\noindent \textbf{\textit{Summary.}} Our technique provides
richer contextual information for the reported error, and 
is significantly more precise than requiring each GUI
access to be wrapped.

\subsubsection{Comparing Call Graph Construction Algorithms}
\label{sec:reflectionaware}

We next compare the 3 call graph construction algorithms (RTA, 0-CFA, and 1-CFA)
used in our experiments. As shown in Table~\ref{table:results},  1-CFA found
more errors than the other two algorithms. This
is because RTA and 0-CFA do not consider the calling context when
constructing a call graph:  they
mix calls to the same method from different callers into a single node, thus
introducing imprecision. Although the error paths exist in
the less precise graphs, our algorithm does not report them because of its heuristics
for outputting the shortest possible call chain. Figure~\ref{fig:ex} illustrates this point.

\begin{figure}[t]
  \centering
  \includegraphics[scale=0.44]{cgexample}
  \vspace*{-5.0ex}\caption {{\label{fig:ex} (a) shows example code, in which
\CodeIn{safe()} is a Safe UI method and \CodeIn{uiAcc()} is a UI-accessing method.
(b) shows a less precise call graph built by RTA or 0-CFA, and (c) shows a more
precise call graph built by 1-CFA\@. Nodes for constructors are omitted for brevity.
\newline
Using the less precise call graph, our error
detection algorithm (Figure~\ref{fig:detectalgorithm}) reports an invalid method
call chain: \CodeIn{entry()}
$\rightarrow$ \CodeIn{f()} $\rightarrow$ \CodeIn{start()} $\rightarrow$ \CodeIn{run()}
$\rightarrow$ \CodeIn{uiAcc()}. It does not report
the actual error path because it is longer: \CodeIn{entry()} $\rightarrow$ \CodeIn{g1()} $\rightarrow$
\CodeIn{g2()} $\rightarrow$ \CodeIn{start()} $\rightarrow$ \CodeIn{run()} $\rightarrow$ \CodeIn{uiAcc()}.
The algorithm outputs the actual error path when using the more precise call graph.
}}
\end{figure}



The results in Table~\ref{table:results} also show that using the
call graph construction algorithm in Section~\ref{sec:cgcon}
helps in finding errors in Android applications. One error from the MyTracks Android application
can only be found by using the reflection-aware call graph construction algorithm (Section~\ref{sec:cg}).
This is because the error-related GUI object (\CodeIn{msgTextView}) in MyTracks is created
reflectively as follows.

{\vspace{2mm}
\hspace{3mm}{In class: com.google.android.apps.mytracks.StatsUtilities}
\vspace{-2mm}
\begin{CodeOut}
\begin{alltt}
97.  public void setLatLong(int id, double d) \ttlcb
98.    TextView msgTextView = (TextView) activity.\textbf{findViewById}(id);
99.    msgTextView.setText(LAT\_LONG\_FORMAT.format(d));
100. \ttrcb
\end{alltt}
\end{CodeOut}}

Another error in SGTPuzzler is only reported when adding native method
annotations (Section~\ref{sec:annotation}). 

%The error in Fennec
%can be found by using a standard call graph construction algorithm.
%This is because the error-related GUI objects are created explicitly,
%so the standard call graph construction algorithm is sufficient.

%Without
%replacing the reflection calls with explicit object creation instructions,
%a call graph construction algorithm will
%conclude that no GUI object has been created, and thus miss many UI-related
%edges in the resulting graph.

\vspace{1mm}

\noindent \textbf{\textit{Summary.}} Using the 1-CFA 
algorithm finds more errors than 0-CFA and RTA, and our reflection-aware
call graph construction algorithm helps to find errors in some
Android applications.


\subsubsection{Comparing Graph Search Strategies}
\label{sec:search}

The error detection algorithm in Figure~\ref{fig:detectalgorithm} uses a
separate BFS for each entry node.  This section evaluates three variants:
a multi-source BFS; a separate DFS for each entry node; and exhaustive
search.

The first variant uses a single BFS, but starting at multiple sources.  It deletes
lines 6 and 24 in Figure~\ref{fig:detectalgorithm} and changes line
7 to:

\vspace{0.2mm}
$\mathit{worklist}$ $\leftarrow$ $\bigcup_{\mathit{entry} \in \mathit{entryNodes}}$ getReachableStarts($\mathit{entry}$)
\vspace{0.2mm}

\noindent This variant returns the shortest path from any \CodeIn{Thread.start()} node
to any UI-accessing node, rather than the shortest path from each
\CodeIn{Thread.start()} node to any UI-accessing node as the algorithm in
Figure~\ref{fig:detectalgorithm} does.
This variant reported 8 errors and 12 false positives.   It missed 1 error in S3dropbox and 1 error in
SudokuSolver.
It visits every node in the call graph only once, rather than potentially
once per entry node, but doing so prunes out real error paths.

The second variant uses one DFS per entry node, by changing
the $\mathit{worklist}$ on line 7 of Figure~\ref{fig:detectalgorithm} to
a stack, and changing queue operations dequeue and enqueueAll
on lines 10 and 21
to pop and pushAll.
This variant reported 9 errors and 10 false positives.  It missed 1 error in FileBunker.
DFS tends to search deeper into the graph and return longer paths that are
more likely to be infeasible or to be removed by the filters.
%For the FileBunker subject, paths
%returned by DFS are much longer than BFS, and do not actually exist.

The third variant uses exhaustive search to find potential errors. This variant
enumerates all non-cyclic paths from all reachable \CodeIn{Thread.start()} nodes
to each UI-accessing node, and then checks whether each
path spawns a new thread and accesses GUI objects without using safe UI methods.
We ran this variant on each subject for 1 hour, it explored 5.1$\times$$10^9$ paths
on average, but did not output any real errors before it terminated.
The number of non-cyclic paths in a graph is exponential in the graph size,
and it is infeasible to enumerate all paths for a realistic call graph.
In our experiments, the smallest call graph contains 7201 nodes and the average out-degree
of each node is 2.15. Thus, a rough estimation for the number of distinct
non-cyclic paths leads to an astronomically large number: 
$2.15^{7201}$ $\approx$ 1.07$\times$$10^{2107}$.

%for a realistic call graph consisting of thousands
%of nodes, 

%only a very few amount of paths can be enumerated within a practical amount of time.

Given a sound call graph, suppose that there exists some
error-revealing path between an entry node E and a UI-accessing node U\@.  A sound
search strategy is one that reports some error-revealing path between
E and U, while an unsound strategy might report a non-error-revealing
path between the nodes (Figure~\ref{fig:ex} shows an example).
Exhaustive search is sound because it does not
miss any possible (non-cyclic) paths.  However, it is impractical.
BFS and DFS are unsound because they visit
every node, but do not traverse every path to visit that node.
Using BFS or DFS can be viewed as a heuristic filtering
step, akin to the filters of Section~\ref{sec:heuristic}.
% The heuristics of Section~\ref{sec:heuristic} are
%more effective at retaining real errors than the search variants.
%Given a sound call graph, exhaustive search is sound because it does not
%miss any possible errors.  However, it is impractical.  Compared to
%exhaustive search, using BFS or DFS can be viewed as a heuristic filtering
%step, akin to the filters of Section~\ref{sec:heuristic}.
%The heuristics of Section~\ref{sec:heuristic} are
%more effective at retaining real errors than the search variants.

%   The algorithm of
% Figure~\ref{fig:detectalgorithm}, which does multiple BFS traversals, does
% less pruning, and runs more slowly, than the variant that does one
% multi-source BFS, but

\vspace{1mm}

\noindent \textbf{\textit{Summary.}} Our algorithm finds more errors than
using multi-source BFS, DFS, or exhaustive search.



\begin{table}[t]
\begin{center}
 \fontsize{9pt}{\baselineskip}\selectfont
\setlength{\tabcolsep}{.34\tabcolsep}
\hspace*{-0.2cm}
\begin{tabular}{|l||c|c|c|c|c|c|}
\hline
 & \multicolumn{6}{|c|}{Number of Warnings}  \\
\cline{2-7}
 Subject & Before & \multicolumn{2}{|c|}{Sound Filters}  &  \multicolumn{3}{|c|}{Heuristic Filters}  \\
\cline{3-7}
 Program &Filtering  & $F_1$ & $F_{1,2}$ & $F_{1,2,3}$&$F_{1,2,3,4}$ & $F_{1,2,3,4,5}$\\
\hline \hline
\multicolumn{7}{|l|}{SWT desktop applications}   \\
 %\hline
 %VirgoFTP &  21 &  21 &  21 & 7 &  2 & 2\\
 \hline
 FileBunker &  4494 &  4494 &  4494 &  3210 &  10  & 2\\
 \hline
 ArecaBackup &  6219 &  438 & 438 &  438  &  1 & 1\\
 \hline
 \hline
\multicolumn{7}{|l|}{Eclipse plugins}   \\
 %\hline
 %EclipseRunner &  xxxx &  xx &  xxx & xx &  xxx & xx\\
 \hline
 EclipseRunner&  1644 &  1644 &  1644 & 1644 &  6 & 6\\
 \hline
 HundsonEclipse&  1367 &  567 &  567 & 567 &  3 & 3\\
 \hline
 \hline
\multicolumn{7}{|l|}{Swing desktop applications}   \\
 \hline
 S3dropbox&  45528 &  31978 &  31978 & 30975 &  9 & 1\\
 \hline
 SudokuSolver &  58 &  58 &  58 & 58  &  2 & 2\\
 \hline
 \hline
\multicolumn{7}{|l|}{Android mobile applications}   \\
 \hline
 SGTPuzzler&  2 &  1 &  1 & 1 &  1 & 1\\
 \hline
 Fennec &  122 &  84 & 80 & 80 &  9 & 3\\
 \hline
 MyTracks &  1176 &  1176 &  483 & 441 &  69 & 1 \\
\hline
 \hline
 Total &  60610 &  40440 &  39753 &  37414 &  110 & \warnings \\
 \hline
\end{tabular}
\end{center}
\vspace{-15pt}
\Caption{{\label{table:filters}Number of warnings after applying a set of
sound and heuristic error filters. Column ``Before Filtering''
shows the number of warnings by the reflection-aware 1-CFA algorithm.
Other algorithms show similar patterns, which are omitted for brevity.
Column ``$F_{i,...,j}$'' represents the number of remaining warnings after applying
the $i$th to $j$th filters as defined in Section~\ref{sec:heuristic}.
The numbers in the last column are the same as the subcolumn ``\#Warning"
under column ``1-CFA'' in Table~\ref{table:results}.
} } \vspace{-2mm}
\end{table}


\subsubsection{Evaluating Error Filters}
\label{sec:filters}

Table~\ref{table:filters} measures the effectiveness of the
error filters of Section~\ref{sec:heuristic}.
The five error filters removed 99.96\% of the reported warnings as likely false positives
or redundant warnings. Specifically, the two sound filters 1 and 2 
removed 34.44\% of the warnings, and the three heuristic filters 3, 4, and 5
removed a further 65.52\% of the warnings. The most effective filter is
\#4, for removing reports with the same head methods from the entry node to \CodeIn{Thread.start()}.

\vspace{1mm}

\noindent \textbf{\textit{Summary.}} Our proposed error filters are 
effective in reducing the number of warnings.
%Although a static analysis may
%issue false positives or redundant reports, a set of well-designed
%error filters can remove most of them.
%Heuristic filter, effective.


\subsection{Discussion}

\label{sec:performance}

\noindent \textbf{\textit{Performance and scalability.}} Our tool
has been evaluated on \subnum subjects with \totaloc LOC and frameworks
with 1.4 MLOC, showing good scalability. Our evaluations
were conducted on a 2.67GHz Intel Core PC with 4GB
physical memory (1GB is allocated for the JVM), running Windows 7.
For the most time-consuming subject, S3dropbox, our tool finished the whole analysis
within 252 seconds using the most expensive 1-CFA call graph construction
algorithm. Analyzing other subjects or using different algorithms took less time.
Call graph construction took 33--91\% of the total time.


\vspace{1mm}

\noindent \textbf{\textit{Threats to validity.}}
There are two major threats to validity in our evaluation. 
One threat is the degree to which the subject programs
used in our experiment are representative of true practice.
In our evaluation, we only selected subjects from
open-source repositories. Another threat is that we only employed three
widely-used call graph construction algorithms (i.e., RTA, 0-CFA, and 1-CFA) 
in our evaluation. Using other 
call graph construction algorithms such as XTA~\cite{xta} or VTA~\cite{Sundaresan:2000} 
 might achieve different results.


\vspace{1mm}

\noindent \textbf{\textit{Limitations.}}
Our technique is limited in three aspects. First, it only considers
non-UI threads that are spawned by the UI-thread after the GUI
is initialized, and ignores other possible non-UI threads (quite unusual)
that are created during the pre-initialization GUI work. One way
to remedy this limitation is to design an analysis to identify
those non-UI threads created before a GUI is launched.
Second, like many bug-finding techniques, our technique
is neither sound nor complete. It may issue false positives
due to the conservative nature of a static analysis.
It may miss true positives due to the graph search strategy,
and it never reports cyclic paths. Furthermore, for the
sake of scalability, our tool implementation uses the default
configuration of WALA and ignores part of the AWT library. Thus,
it missed one error in S3dropbox which can be found by using a different
technique~\cite{JSR308-webpage-201110}.
Designing better call graph construction algorithms,
graph search strategies, and filtering heuristics may alleviate this limitation.
Third, our tool cannot compute call relationships
for inter-process communication between components. It also
requires users to manually add annotations to characterize
call relationships that involve native methods. This limitation
may lead to false negatives. Investigating the false negative
rate is ongoing work.


\vspace{1mm}

\noindent \textbf{\textit{Experimental Conclusions.}}
Invalid thread access errors  can be subtle to detect in many cases.
The technique presented in this paper offers a promising solution.
Our technique finds real-world errors and issues
few false positive warnings. Our
proposed filters are useful in reducing the number of warnings.


%%% Local Variables: 
%%% mode: latex
%%% TeX-master: "guierror"
%%% TeX-command-default: "PDF"
%%% End: 

%  LocalWords:  multithreaded SWT multithreading 2mm VirgoFTP UI GMail S3
%  LocalWords:  FileBunker ArecaBackup plugins HudsonEclipse EclipseRunner
%  LocalWords:  S3dropbox SudokuSolver Sudoku mutlithreaded SGTPuzzler 9pt
%  LocalWords:  Fennec MyTracks LOC v9306 plugin LOCC RTA CFA 9pt eMail g1
%  LocalWords:  CalledByNativeMethods deleteBucket deleteCurrentBucket g2
%  LocalWords:  deleteError showErrorMessage getSelectedBucketName uiAcc
%  LocalWords:  Czarniecki initializeResourceChangeListener getWorkspace
%  LocalWords:  ResourcesPlugin addResourceChangeListener resourceChanged
%  LocalWords:  IResourceChangeListener IResourceChangeEvent getViewer BFS
%  LocalWords:  ResourceChangeEvent msgTextView setLatLong TextView DFS
%  LocalWords:  findViewById setText getReachableStarts enqueueAll pushAll
%  LocalWords:  HundsonEclipse subcolumn MLOC 67GHz 4GB 1GB XTA VTA pre


\section{Related Work}

Work related to this paper falls into three main categories; (1)
analyzing and testing GUI applications; (2)
bug finding techniques for multithreaded programs; and (3)
call graph construction algorithms.

\tinystep
\subsection{Analyzing and Testing GUI Applications}
%\vspace{1mm}

%\noindent \textit{\textbf{Analysis and Testing for GUI Applications}}
Automated GUI testing is a challenging task~\cite{Bertolino:2007:STR:1253532.1254712,
Harrold:2000:TR:336512.336532}.
Various techniques automate GUI testing including model creation
(for model-based testing)~\cite{androidtesting}, %, Xie:2006:MTC:1172962.1172990},
test generation~\cite{YuanMemonICSE2007},
test oracle creation~\cite{MemonFSE2000}, test execution~\cite{YuanCohenMemonTSE2011},
and test script repairing~\cite{Huang:2010:RGT:1828417.1828465, Daniel:2011:AGR:2002931.2002937}.
For example,
Guitar~\cite{YuanCohenMemonTSE2011, YuanMemonICSE2007}
is a GUI testing framework for Java and Microsoft Windows applications. 
Yuan and Memon~\cite{YuanMemonICSE2007} generate event-sequence-based test cases for GUI
programs using a structural event generation graph. 
However, testing is often insufficient to detect many potential
errors in a GUI application due to the huge space of possible UI interactions.
%The large number of possible interactions
%requires a large number of test inputs that require substantial human effort.
In contrast, a static analysis can explore all paths to find potential errors missed by testing.
Compared to software testing, a static analysis tool such as ours may
report false positives and redundant warnings due to its conservative nature.
In our experiments, simple error filters reduced the number of warnings to
an acceptable level.


Michail and Xie~\cite{michail05:helping} proposed a tool-based approach to help users avoid bugs
in GUI applications. Their approach monitors a user's action in the background,
and gives a warning as well as the opportunity to abort the action, when
a user attempts an action that has led to problems in the past. 
Their work aims to prevent an existing bug from happening again.
By contrast, our work aims to find unknown errors.


Recently, Payet and Spoto~\cite{Payet:2011:SAA:2032266.2032299} presented a static
analysis framework for Android programs based on  abstract
interpretation. Their framework focuses on the Android platform, and
 consists of 7 existing static analyses such as
nullness analysis, class analysis, and termination analysis.  However,
their framework does not support detecting invalid-thread-access
errors, and uses a quite different abstraction than ours.
To the best of our knowledge, we are the first to address the invalid
thread access error detection problem for multithreaded GUI applications, and
our core technique has been tailored for four GUI frameworks.


\tinystep
\subsection{Finding Bugs in Multithreaded Programs}

%\vspace{1mm}

%\noindent \textit{\textbf{Finding Bugs in Multithreaded Programs}}
A rich body of techniques have been developed to detect bugs in multithreaded programs~\cite{Huang:2011:PPC:2001420.2001438, Weeratunge:2010, Huang:2011:EST}.
Static analysis tools such as Chord~\cite{Naik:2006}
explore multithreaded programs. Runtime
analysis tools such as Eraser~\cite{Savage:1997}  dynamically detect concurrency bugs using lockset
algorithms or some criteria-based automata. 
More recently, Goldilocks~\cite{Elmas:2007} uses a
hybrid model that combines the happens-before and lock-based
approaches to identify data races based on an execution.
However, finding invalid-thread-access errors is quite different than detecting data races.
A data race occurs when two concurrent threads access
a shared variable and when at least one access is a write and the threads
use no explicit mechanism to prevent the accesses from being simultaneous. In contrast,
an invalid thread access error occurs when a non-UI thread accesses (reads or writes) a GUI object.
Unlike detecting data races, finding an invalid-thread-access error does not require monitoring every shared-memory
reference to verify that consistent locking behavior is observed among different threads.
A technique only needs to track whether a non-UI thread can accesses a GUI object or not,
and is much cheaper. Leveraging data race detection to 
improve our technique is future work.


An alternative way to find bugs in multithreaded programs is using model checking~\cite{Nori:2010:ESO, Inverardi:2000, Siegel:2008}.
By exhaustively exploring the thread scheduling space, a model checker can
report counterexamples as bug reports.
Unfortunately, due to the exponential size of the search space,
it is hard for model checking approaches to scale to a realistic multithreaded GUI application
 without compromising the error detection capability. 
We are not aware of any software model checking approach that scales to programs
as large as those used in our experiments (including the library code).
The technique presented in this paper is specifically designed to find invalid thread
errors instead of being a general property checking tool. 
It chooses the call graph as a coarse-grained program representation with a
set of error filters, to achieve good scalability with reasonable accuracy.

%For this reason,
%our technique achieves good scalability with reasonable precision.

\tinystep

\subsection{Call Graph Construction Algorithms}
%\vspace{1mm}

%\noindent \textit{\textbf{Call Graph Construction Algorithms}}
We briefly mention some call graph construction
algorithms for Java. Grove et al.~\cite{kcfa} described a unified
framework for expressing call graph construction algorithms, and
studied different instantiations of the framework.
Tip and Palsberg~\cite{xta} quantitatively compared
several low-cost call graph construction algorithms for Java.
Sundaresan et al.~\cite{Sundaresan:2000} went beyond the
RTA~\cite{rta} approach and used type propagation
to build a more precise call graph.  However,
those algorithms do not build a complete call graph in the presence of reflection.
As reflected in our experiments, using standard call graph algorithms
misses errors in some Android applications.


Livshits et al.~\cite{Livshits:2005} presented a static analysis
to reason about reflective calls. The analysis
attempts to infer additional information stored in string constants to resolve
reflective calls statically. Their approach focuses on standard Java
reflection calls (e.g., \CodeIn{Class.forName}) instead of
framework-specific ones (e.g., \CodeIn{View.find-\\ViewById}).
The standard call graph algorithms implemented in WALA, which we used
in our experiments, actually handles standard Java
reflection calls as~\cite{Livshits:2005} does, but still fails to build
a sufficiently complete call graph for
Android applications. TamiFlex~\cite{Bodden:2011}, a pure dynamic
approach, records all reflectively-created class instances
by intercepting JVM system calls, and re-inserts those recorded 
class into a program. However, TamiFlex requires a set of representative
program executions and is only sound with respect to the given executions.
Perhaps the closest work to our call graph
construction algorithm is Payet and Spoto's Julia
system~\cite{Payet:2011:SAA:2032266.2032299}. The Julia system
needs to first instrument Android's library code that performs the XML inflation,
and then replaces the \CodeIn{findViewById} call with the corresponding
object creation expressions. In addition, the Julia system does not handle
native methods when building call graphs. In contrast, our technique provides
annotation support for native methods, and our tool does not need a separate
pass of off-line instrumentation. The reflection-aware call graph is created
online by intercepting the standard call graph construction process.





%There are few algorithms have been studied to construct call
%A recent paper by Hirzel, Diwan, and Hind addresses the issues of dynamic
%class loading, native methods, and reflection in order to deal with the full complexity
%of Java in the implementation of a common pointer analysis [5]. Their
%approach involves converting the pointer analysis [6] into an online algorithm:
%they add constraints between analysis nodes as they are discovered at runtime.
%Newly generated constraints cause re-computation and the results are propagated
%to analysis clients such as a method inliner and a garbage collector at
%runtime. Their approach leverages the class hierarchy analysis (CHA) to update
%the call graph. Our technique uses a more precise pointer analysis-based
%approach to call graph construction.


 


\section{Conclusion and Future Work}

In this paper, we presented a general technique to find invalid
thread access errors in multithreaded GUI applications. 
Our technique statically explores paths in a call graph to check
whether a non-UI thread can access any GUI object.
It uses a combined RTA and k-CFA algorithm to construct
a good call graph, and employs a set of heuristics to
filter likely false positives and redundant reports.
Since graphical user interfaces are often inadequately tested due
to the enormously large space of UI interactions and
extreme resource constraints during their development, we
believe that providing a static analysis can allow developers to find
more potential errors %before the application is released to customers (XXX),
and thus encourage GUI developers to improve software reliability.

We have demonstrated that our technique can be both practical and useful
by an evaluation on XXX subjects acrosss 4 popular GUI
frameworks. Our experiments have shown that we can find bugs
in real-world programs.


Besides general issues such as performance or ease of use, our future
work will concentrate on the following topics:

\begin{itemize}

\item \textbf{Integration with dynamic and symbolic analyses.} The technique 
presented in this paper is a pure static analysis. It
uses a combined RTA and k-CFA  algorithm to construct a relatively
complete and precise call graph. However, our technique still suffers from
a false positive rate for many GUI applications, due to the conservative
nature of static analysis. A possible way to reduce such false
positives is to integrate the current static analysis with
dynamic analyses~\cite{Jiang:2008:PPS:1453101.1453110, ZhangSBE2011}
or symbolic analyses~\cite{xie05:symstra, Pasareanu:2011, halfond09issta, BMF97}
by employing more accurate information to guide call graph exploration.

\item \textbf{Unit testing multithreaded GUI programs.} Besides
a static analysis,  software testing is another
way to improve software quality.  Although many
GUI testing techniques have been developed recently, few of them can be applied
to unit test multithreaded GUI programs to find potential errors \textit{earlier}. We
are interested to investigate how to apply recent advance in automated
software testing~\cite{Staats:2011:PTO:1985793.1985847, Jagannath:2011:IMU:2025113.2025145, Muccini_Bertolino_Inverardi_2004, Ricca:2001:ATW:381473.381476, Harman:2007}
to the context of multithreaded GUI applications.


\item \textbf{Fixing potential GUI errors.} After a potential error is revealed, fixing
it has often been another important and time-consuming process. Fixing concurrency
bugs has become especially critical in the multicore era.
Recently, a few work has been done
on automatically repairing test script for GUI applications~\cite{Daniel:2011:AGR:2002931.2002937, Huang:2010:RGT:1828417.1828465}. However, none of them focuses on repairing
the GUI program to fix a revealed bug. Thus, we are interested in further exploring 
along that line to investigate techniques to fix potential errors in
multithreaded GUI applications.

\end{itemize}


The source code of our tool implementation is available at:

\vspace{1mm}

\noindent \url{http://guierrordetector.googlecode.com}


\vspace{2mm}

\noindent \textbf{Acknowledgement.} We would like
to thank Stephen Fink and Manu Sridharan for
generously answering our questions about WALA.
%thank XXX
%reviewers for their feedback. This work was supported in
%part by ABB Corporation and NSF under grant CCF-0963757.
\vspace{-2mm}

\bibliographystyle{abbrv}
\footnotesize{
\bibliography{issta2012}
}
\end{document}
