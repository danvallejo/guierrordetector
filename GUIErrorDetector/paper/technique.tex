\section{Technique}

\subsection{Problem Formulation}

formulate it as a graph reachability

\subsection{Error Detection Algorithm}


\begin{figure}[t]
\textbf{Input}: a Java program $P$\\
\textbf{Output}: a set of potential $errors$\\
\vspace{-5mm}
\begin{algorithmic}[1]
\STATE errors $\leftarrow$ $\emptyset$
\STATE cg $\leftarrow$ $constructCallGraph$($P$)
\STATE entryNodes $\leftarrow$ $getEntryNodes$(cg)
\STATE uiAccessNodes $\leftarrow$ $getUIAccessNodes$(cg)
\STATE safeNodes $\leftarrow$ $getSafeNodes$(cg)
\FOR{each entryNode in entryNodes}
\STATE threadStartNodes $\leftarrow$ $getReachableStartsByBFS$(cg)
\FOR{each threadStart in threadStartNodes}
\STATE queue $\leftarrow$ an empty queue
\STATE queue.enqueueAll(threadStart.$getAllSuccNodes$())
\WHILE{queue.isNotEmpty()}
\STATE nextNode $\leftarrow$ queue.dequeue()
\FOR{each succNode in cg.$getSuccNodes$(nextNode)}
\IF{succNode $\in$ uiAccessNodes}
\STATE errors $\leftarrow$ errors $\cup$ $createErrorReport$(succNode)
\ELSIF{succNode $\in$ safeNodes}
\STATE continue
\ELSE
\STATE queue.enqueue(succNode)
\ENDIF 
\ENDFOR
\ENDWHILE
\ENDFOR
\ENDFOR
\RETURN $errors$
%\ENDWHILE
\vspace{-2mm}
\end{algorithmic}
\caption{Algorithm for detecting errors in multithreaded GUI programs. 
Four utility methods in lines 2 -- 4 are customized for each GUI platforms
(not each program) as described in Section~\ref{sec:platforms}.
} \label{fig:detectalgorithm}
\end{figure}


\subsubsection{Call Graph Construction}

For desktop applications which have main method, it is straigtforward to apply
the above algorithm. However, for applications on platforms like eclipse plugin
and Android. presents new challenges. Two technical challenges must be solve.
First, use reflection extensively, Second, it is possible to interact with
native code.

Android features an extended event-based library and dynamic inflation of
graphical views from declarative XML layout files.
A static analyzer for Android programs must consider such features, for
correctness and precision.

\begin{figure}[t]
%\centering
\begin{CodeOut}
\begin{alltt}

<LinearLayout>
    <Button android:id="@+id/\textbf{button\_id}" android:text="A Button" />
</LinearLayout>

1. public class MyActivity extends Activity \{
2.    final List<View> cachedViews = new LinkedList<View>();
3.    @Override
4.    public void onCreate(Bundle savedInstanceState) \{
5.        super.onCreate(savedInstanceState);
6.        setContentView(R.layout.main);
7.        Button button = (Button) findViewById(R.id.\textbf{button\_id});
8.        button.setOnClickListener(new Button.OnClickListener() \{
9.            @Override
10.           public void onClick(View v) \{
11.               button.setText("Button Clicked.");
12.               cachedViews.add(v);
13.           \}
14.       \});
15.   \}
16. \}
\end{alltt}
\end{CodeOut}
\label{fig:sampleandroid}
\caption{A Sample Android Code ..}
\end{figure}


Give an example of how reflection affect call graph precision.

The Android button example.

Using RTA is not enough, since that maintains a global

So, we present a new call graph construction algorithm

reflected classes are known, use RTA.

reflection code

native code

\begin{figure}[t]
\textbf{Input}: a Java program $P$, a list of classes $cls$\\
\textbf{Output}: a call graph $cg$\\
\vspace{-5mm}
\begin{algorithmic}[1]
\STATE callSiteMap $\leftarrow$ new Map$\langle$CallSite, List$\langle$Type$\rangle$$\rangle$
\STATE rta\_cg $\leftarrow$ $constructCallGraphByRTA$($P$)
\FOR{each $method$ in $P$}
\FOR{each  $callSite$ in $method$}
\STATE type $\leftarrow$ $staticType$(callSite)
\IF{type $\in$ $cls$}
\FOR{each subType of $subType$(type)}
\IF{$hasObjectCreated$(subType, rta\_cg)}
\STATE callSiteMap[callSite].add(subType)
\ENDIF
\ENDFOR
\ENDIF
\ENDFOR
\ENDFOR
\STATE $cg$ $\leftarrow$ $constructCallGraphByKCFA$($P$, callSiteMap)
\RETURN $cg$
%\ENDWHILE
\vspace{-2mm}
\end{algorithmic}
\caption{Algorithm for constructing a call graph using a combination of the RTA~\cite{rta} and k-CFA~\cite{kcfa} algorithms.
$cls$ is a list of class XXXX, whose invocatioon is treated using RTA; while the remaining call graph
part is constructed using k-CFA.
} \label{fig:cgalgorithm}
\end{figure}




\subsubsection{Heuristic Filters and Annotations}
(XXX)

Static analysis exhaustively check possible error path; it may report
paths that indicate the same bug. The potentially huge volume of
reported warnings impose great burden for programmers to check
its validity. 

a few useful static heuristic filtering rules.

\begin{itemize}
\item \textbf{Method call chain sumbsumption}. Remove redundant
results

\item \textbf{Remove system calls}. Method call chain involves typical
system methods like \CodeIn{toString} are unlikely to reveal a real bug.

\item \textbf{Method call chain with the same entry}. such
call chains may reveal the same bug

\item \textbf{Method call chain with the same tail}.  different ways
to reach the same bug

\item \textbf{User-defined rules}.  to overcome the limitation

\end{itemize}

The above filtering rules are not sound, it may filter real bugs but works
well in practice.

We also provide annotations for native methods which are beyond the ability
of most static analyses

\subsection{Instantiation for Different Frameworks}
\label{sec:platforms}

We instantiate our error detection techniques for 4 popular GUI frameworks,
namely SWT, Eclipse plugin environment, Swing, and Android.
For each GUI platform, we customize the \textit{call graph entry nodes},
\textit{UI accessing nodes}, and \textit{safe methods} on lines 3 -- 5 of the
algorithm in Figure~\ref{fig:detectalgorithm}.

%For the sake of efficiency, they are all using the single thread model.

%Why these platforms? popularity? account for xxx\% of the GUI. 

%customizing for: entry points, starting points, error checking, thread
%safe methods, what is the event dispatching thread?

\subsubsection{SWT}

SWT is an open source widget toolkit for Java designed to provide efficient,
portable access to the GUI facilities of the operating systems on which it is implemented.

Like normal Java applications, a SWT desktop application starts from its main method.
By default, the thread that the main method executes is the UI thread. Thus, we instantiate
our technique as follows:

\begin{itemize}

\item \textbf{Call graph entry points: } the single main method

\item \textbf{UI accessing nodes: } any methods that calls \CodeIn{org.eclipse.swt.widgets.Display.checkWidget}
or \CodeIn{org.eclipse.swt.widgets.Display.checkDevice} methods, since these two methods are the
only place where the SWT framework checks whether the current thread is the UI thread or not. Thus, any
methods that calls the checking method must be invoked in the UI thread.

\item \textbf{Safe methods: } SWT provides two utility methods (\CodeIn{Display.asyncExec}
and \CodeIn{Display.syncExec}) that pass messages to the UI thread.
 Invoking these two methods are generally considered to be safe, and would not
cause an invalid thread access error.

\end{itemize}

\subsubsection{Eclipse plugin}

Unlike a SWT desktop application, an eclipse plugin, though is developed in SWT,
has no single main method. Instead, (XXX) it extends certain extension points
exposed by the eclipse framework, and eclipse will XXX. To instantiate our
technique for the Eclipse plugin environment, we need to redefine the
call graph entry points.

\begin{itemize}

\item \textbf{Call graph entry points: } all overriden methods of the classes inside
 a plugin's \CodeIn{ui} package.  Such methods are normally called by the eclipse framework
by the UI thread. XXX This rule, though quite heuristical, works quite well in practice.

\item \textbf{UI accessing nodes} and \textbf{Safe methods} are the same as SWT desktop applications.

\end{itemize}

\subsubsection{Swing}

%The collections classes from the original JDK (\CodeIn{Vector}
%and \CodeIn{Hashtable}) and the AWT library are designed as
%thread-safe. That is, two separate threads of execution can
%access the UI element at the same time without the developer having
%to worry about the threads interfering with one another. However, the
%safety comes at a cost. Because there is a great deal of overhead
%necessary to build thread-safe artifacts, the tend to be much slower
%than nonthread-safe alternatives. 

%The design decisions for Swing changed. This does not mean the controls
%can never be accessed from multiple threads, but the developer is now
%responsible for adding code to ensure that no ill effects occur.

A Swing application often contains three kinds of threads: \textit{initial thread}
that executes initial application code, the \textit{event dispatch thread},
where all GUI manipulation code is executed, and the \textit{worker thread} where
time-consuming background tasks are executed. After Swing UIs are launched,
the initial thread exits and the event dispatch thread starts to execute all
event-handling code or spawns new worker threads. XXXX Therefore, we can not directly
use the main method as the call graph entry point, since it is not invoked
by the UI thread. We customize our technique for Swing as follows:

\begin{itemize}

\item \textbf{Call graph entry points: } all overriden callback methods of the listerner classes.
Specifically, for each class, we check whether it is a Swing listerner class. If so,
we extract all its overriden callback methods as the call graph entry points.
 %Such methods are normally called by the eclipse framework
%by the UI thread. XXX This rule, though quite heuristical, works quite well in practice.

\item \textbf{UI accessing nodes:} unlike SWT, Swing does not explicitly check
whether the current thread is the UI thread or not, when a GUI object is accessed. Except
for three documented thread-safe method (\CodeIn{repaint}, XXXX), our technique
treat all methods defined in each GUI object as UI accessing nodes.

\item \textbf{Safe methods: } similar to SWT, Swing provides two utility methods
(\CodeIn{SwingUtilities.invokeLater} and \CodeIn{SwingUtilities.invokeDelay}) to pass messages to the UI thread. These two methods are regarded as safe methods.

\end{itemize}

%A key tasks in instantiating
%our technique to the Swing framework is to identify its event dispatch thread.
%We use

\subsubsection{Android}

Android is a Java-based platform for embedded or mobile devices. 
Android features an extended event-based library and dynamic inflation of
graphical views from declarative XML layout files. Android applications are
written in Java, running in their own process within their own virtual machine.
Unlike desktop applications, they do not have a single entry point but can
rather use parts of other Android applications on-demand and can require their
services by calling corresponding event handlers, directly or through the
operating system. In particular, Android applications use \textit{activities}
as code interacting with the user through a visual interface. Event handlers
are scheduled in no particular ordering, with some notable exceptions such as the
lifecycle of activities.

An XML \textit{manifest file} registers the components of an application. Other XML
fies describe the visual layout of the activities. Activities \textit{inflate}
layout files into
visual objects (a hierarchy of views), through an \textit{inflater} provided by the An-
droid library. This means that library or user-defined views are not explicitly
created by new statements but rather inflated through \textit{reflection}. Library meth-
ods such as \CodeIn{findViewById} access the inflated views.

We instantiate the technique for Android programs as follows.

\begin{itemize}

\item \textbf{Call graph entry points: } all overriden callback methods of the event
listerner classes, and all overriden callback methods of an Activity. 

\item \textbf{UI accessing nodes: } any methods that calls the \CodeIn{ViewRoot.checkThread}
method. This method is the only place where Android checks
whether the current thread is the UI thread or not. Thus, any
methods that calls the checking method must be invoked in the UI thread.

\item \textbf{Safe methods: } Android provides two utility methods (\CodeIn{View.post}
and \CodeIn{View.XXX}) that pass messages to the UI thread.
 Invoking these two methods are generally considered to be safe, and would not
cause an invalid thread access error.

\end{itemize}
