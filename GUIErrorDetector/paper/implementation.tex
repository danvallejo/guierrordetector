\section{Implementation}
\label{sec:implementation}

We instantiated the proposed technique for four widely-used
GUI frameworks, namely SWT, Eclipse plugin development, Swing, and Android;
and implemented a prototype tool.
Our tool is built on top of the WALA framework~\cite{walatutorial}.
It takes Java bytecode as input, 
automatically detects potential invalid thread access
errors, and scales to realistic programs.  

%We implemented the proposed technique using the WALA framework~\cite{walatutorial}.
%Our prototype tool takes Java bytecode as input, 
%automatically detects potential invalid thread access
%errors, and scales to realistic programs.  




%Static Single Assignment (SSA) transformation to get limited flow sensitivity
\subsection{Instantiation for Different Frameworks}
\label{sec:platforms}

%We instantiated our error detection technique for four popular GUI frameworks:
%SWT, Eclipse plugin, Swing, and Android.
When instantiating our error detection technique for different
frameworks, the major framework-specific parts, corresponding to
lines 3 -- 5 in Figure~\ref{fig:detectalgorithm}, are identifying
\textbf{call graph entry nodes}, \textbf{UI-accessing nodes},
and \textbf{safe UI methods} for each framework.

%For each framework, we customize the \textit{call graph entry nodes},
%\textit{UI-accessing nodes}, and \textit{safe UI methods} on lines 3 -- 5 of the
%algorithm in 

%For the sake of efficiency, they are all using the single thread model.

%Why these platforms? popularity? account for xxx\% of the GUI. 

%customizing for: entry points, starting points, error checking, thread
%safe UI methods, what is the event dispatching thread?


\subsubsection{SWT}

 We instantiated our technique for SWT applications as follows:

\begin{itemize}
\smallstep

\item \textbf{Call graph entry nodes: } the main method. Like a normal Java program,
a SWT desktop application has a single main method as the entry point. By default,
the main method is executed in the UI thread after the GUI
is initialized.

\smallstep

\item \textbf{UI-accessing nodes: } the \CodeIn{Widget.checkWidget}
and \CodeIn{Display.} \CodeIn{checkDevice} methods. SWT uses the above
two methods to explicitly check thread accesses at runtime.
%Any methods for manipulating GUIs object must call (one of) these two methods
% to determine whether the current thread is the UI thread or not.
\smallstep

\item \textbf{Safe UI methods: } \CodeIn{Display.asyncExec}
and \CodeIn{Display.syncExec}. SWT provides these two methods
to execute code (a)synchro- nizedly on the UI thread.
%We treat these two methods as safe UI methods, since invoking
%them from any thread 
%will not cause an invalid thread access error.

\end{itemize}

\subsubsection{Eclipse plugin}

We instantiated our technique for the Eclipse plugin development framework as follows:

\smallstep

%Unlike a SWT desktop application, an eclipse plugin, though is developed in SWT,
%has no single main method. Instead, (XXX) it extends certain extension points
%exposed by the eclipse framework, and eclipse will XXX. To instantiate our
%technique for the Eclipse plugin environment, we need to redefine the
%call graph entry points.

\begin{itemize}

\item \textbf{Call graph entry nodes: } all user code methods that override
 SWT GUI event handling methods. Eclipse
calls back the overridden methods to handle the
events. All SWT GUI event handling methods (i.e., the overriden
methods in the class that implements \CodeIn{org.eclipse.swt.internal.SWTEventListener}) are
always called back from the UI thread, thus are used as call graph entry nodes.

\smallstep

\item \textbf{UI-accessing nodes} and \textbf{Safe UI methods} are the same as SWT.

\end{itemize}

\subsubsection{Swing}


A Swing application has a single main method, but contains three kinds of
threads: \textit{initial thread} that executes initial application code from the main method,
the \textit{UI thread}, where all GUI manipulation code is executed,
and the \textit{worker thread} where time-consuming background tasks are executed.
After a Swing program starts, its initial thread exits and the UI thread takes charge
of the application and starts to execute all event-handling code or spawns new worker threads. 
We instantiated our technique for Swing as follows:
%XXXX Therefore, we can not directly
%use the main method as the call graph entry point, since it is not invoked
%by the UI thread. We customize our technique for Swing as follows:

\begin{itemize}

\item \textbf{Call graph entry nodes:} all overridden Swing GUI event handling
methods in user code. Those event handling methods are always
called back from the UI thread.

\smallstep

\item \textbf{UI-accessing nodes:} %as stated in Swing's documentation,
all methods defined in each Swing GUI class except for three thread-safe
methods: \CodeIn{repaint()}, \CodeIn{revalidate()}, and \CodeIn{invalidate()}.

\smallstep

\item \textbf{Safe UI methods: } 
\CodeIn{SwingUtilities.}\CodeIn{invokeLater} and \CodeIn{Swing-\\Utilities.invokeAndWait} that execute code on the UI thread.

\end{itemize}

\smallstep

\subsubsection{Android}

Android is a Java-based platform for embedded or mobile devices. 
An Android program does not have a single entry point but can
rather use parts of other Android applications on-demand and can require their
services by calling corresponding event handlers, directly or through the
operating system. In particular, Android applications use \textit{activities}
(i.e., instances of the \CodeIn{Activity} class)
to interact with users through a visual interface and handle GUI events.
We instantiated our technique for Android programs as follows:
%, with some notable exceptions such as the
%lifecycle of activities.

%An XML \textit{manifest file} registers the components of an application. Other XML
%fies describe the visual layout of the activities. Activities \textit{inflate}
%layout files into
%visual objects (a hierarchy of views), through an \textit{inflater} provided by the An-
%droid library. This means that library or user-defined views are not explicitly
%created by new statements but rather inflated through \textit{reflection}. Library meth-
%ods such as \CodeIn{findViewById} access the inflated views.


\begin{itemize}

\item \textbf{Call graph entry nodes: }in an Android application,
an \CodeIn{Activity} object is created and manipulated by the UI thread. Thus, we treat
all public methods defined in the \CodeIn{Activity} class 
and any overriding definitions in its subclasses as call graph entry nodes.
We also add all overridden Android GUI event handling methods in user
code as entry nodes, since they are also called back from the UI thread.

\smallstep

\item \textbf{UI-accessing nodes: }the \CodeIn{ViewRoot.checkThread} method.
Like SWT, Android explicitly checks whether
the current thread is a UI thread or not before accessing a GUI object via
the \CodeIn{ViewRoot.checkThread} method.

\smallstep

\item \textbf{Safe UI methods: } \CodeIn{View.post}
and \CodeIn{View.postDelay} that execute code on the UI thread. 
%Invoking them on any thread will not cause an invalid thread access error.
%Invoking these two methods are generally considered to be safe, and would not
%cause an invalid thread access error.

\end{itemize}


Instantiating our general technique to  a specific framework
requires moderate human effort. We wrote around 500 lines of Java code in total to achieve
the above four instantiations.

\subsection{Implementation Details}

%Our current implementation supports four widely-used GUI frameworks:
%SWT, Eclipse plugin development, Swing, and Android.
We implemented the reflection-aware call graph construction
 algorithm (Section~\ref{sec:cg}) using WALA's \textit{bypass logic}.
Unlike similar tools~\cite{Payet:2011:SAA:2032266.2032299}, our tool
does not require a separate
pass for program instrumentation; instead, it
parses the configuration file in an Android application,
and then intercepts the call graph construction
process on-the-fly to replace all reflection calls with object creation statements.
Since Android applications are often fully encrypted and shipped in Dalvik
bytecode as a single apk file, our tool first uses
android-apktool~\cite{apktool} to
decrypt the apk file, and then uses the 
ded translator~\cite{ded} to convert
Dalvik bytecode to Java bytecode before feeding to WALA.  The Android system
library (i.e., \CodeIn{android.jar}) uses many ``stub'' classes as
placeholders for the sake of efficiency. We manually re-compiled 
\CodeIn{android.jar} from its source code, so it contains real
class files rather than stubs.
